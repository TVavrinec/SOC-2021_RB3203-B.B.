https://docs.google.com/presentation/d/1z_a-4hQFeqkxhdXX8CfWHh0_gHIVp0qsNtIAHn8cLWs/edit#slide=id.gc9a0d5e08f_2_101

1. Věta 1. slidu má být: Co je to blackbox? -> Představím na 1 minutu v kostce.  
ukážu ho v ruce, lze ho ukázat i jako součást nějaké skříňky? Ať je tam taky ten BOX  
a ne jenom jeho dveře, 
+ vysvětlit název -> reklama, nadsázka (např.: protože může pro účastníka kurzu nebo 
hry skrývat překvapivé možnosti (později rozvedu -> a navážu v předposledním slidu, 
např. hráč musí přijít na ovládání náklonem, aby se dostal dovnitř ... ).


2. slide - proč blackbox (výuka i hry)

3. slide - jak se blackbox vyvíjel ??? - možná ani ne ? 

4. slide (+ možná i pátý) - přehled elektroniky 

další slidy - jednotlivé části elektroniky podrobně 


předposlední slide/slidy - použití: počty lidí, .. varianty a nebo náměty her ... => KILLER-APP

poslední - závěr + Rohlínkovo programování  

úvod 30s závěr 30s 
co je to 1min 
proč 2 min 
tlakovka 1 min 
elektronika 2-3 min 
použití 1-2 min 


Slide 1

Dobrý den, jmenuji se Tomáš Vavrinec, jsem ze SPŠ a VOŠ Brno, Sokolská
a dnes vám budu prezentovat svojí práci SOČ s názvem BlackBox.

% Řekněte jim, co jim chcete říct. Řekněte jim to. Řekněte jim, co jste jim řekli. 

Slide 2
1. Věta 1. slidu má být: Co je to blackbox? -> Představím na 1 minutu v kostce.  
ukážu ho v ruce, lze ho ukázat i jako součást nějaké skříňky? Ať je tam taky ten BOX  
a ne jenom jeho dveře, 
+ vysvětlit název -> reklama, nadsázka (např.: protože může pro účastníka kurzu nebo 
hry skrývat překvapivé možnosti (později rozvedu -> a navážu v předposledním slidu, 
např. hráč musí přijít na ovládání náklonem, aby se dostal dovnitř ... ).


Toto téma jsem si zvolil z několika důvodů. Za prvé protože se již nějakou dobu zabývám 
výukou robotiky v zájmových kroužcích a na robotických táborech. 

Slide 3 

K čemu je dobrý Blackbox 



%Slide 3

Z toho důvodu potřebuji 
vhodné stavebnice, se kterými se účastníci mohou učit, a na táborech ideálně i hrát hry.
Stavebnice jsou pro výuku robotiky ideální, protože účastníkům umožňují práci jak 
s hárdwarem tak jeho programem.

Slide 4

Jednou z takovýchto stavebnic byl i robot SchoolBot, o kterém byla moje loňská práce SOČ.
SchoolBot byl roborické vozítko, podobných je však na trhu spoustu a i mi jsme jich již měli v kroužcích
a na táborech několik. Tyto vozítka jsou ale vždy vázány na hladký povrch a jinde se s nimi jezdit nedalo.
Jakékoli teréní úpravy těchto vozítek pak byly velmi drahé a ani tak nebyli sto projet všude.

Slide 5

Toto je BlackBox a jak vidíte tyto problémy u něj nehrozí. Není to totiž robotické vozítko ale zařízení které uživatel
sám nosí v ruce a tak BlackBox muže používat kdekoli a potřebuje k tomu jen volné ruce a nabitou baterii.



Slide 6

Dříve jsme také měli velmi podstatný problém s dálkovým ovládáním a obeceně bezdrátovým připojením.
To jsme totiž častokráte potřebovali a museli jsme to řešit externími modulů.

BlackBox je částečně i proto založen na čipu ESP32 který má výkoný dvou jádroví procesor integrovaný WiFi a Blutooth modul. 
Jeden BlackBox je tak schopen se například spojit s jiným BlackBoxem a sdílet libovolná data, 
nebo se třeba spojit s mobilem uživatele a nechat se jím dálkově ovládat. % fotku 



Slide 7

Asi víte o tom že existuje mnoho již navržených stavebnic jako je například známé Arduino, 
ke kterým jde připojit mnoho různých senzorů a jiných periferii pro vše co vás může napadnout, které jsou ke všemu 
často velmi levné. 
Proč jsem tedy jen nesestavil dohromady několik desek Arduina? Při sestavování různých desek
totiž ne vždy vím jak konkrétně jsou navrženy a něco mi tak může mezi sebou kolidovat.
Primárně jde však o fakt že takováto zařízení jsou obvykle velmi nepřehledná a obtížně 
se na nich hledají chyby. Často totiž vypadají a u dětí to platí dvojnásob, jako vrabčí hnízdo.
V podobné změti kabelů pak stačí aby jediný kabel ztratil kontakt a některý z vedoucích 
strávý půl dne hledáním problému na jedno zařízení, zatím co by měl pomáhat všem dětem v místnosti.

Slide 8

Z tohoto důvodu je BlackBox navržen tak aby k těmto problémům docházelo co nejméně. 
Vše je umístěno na dvou kruhových deská z nichž na hlavní desce je procesor, power management a většina
senzoriky. Na druhé desce je pak světelný kruh a snímání tlakové plochy, o obojím vám něco řeknu za chvíli.
Obě desky jsou pak přehledně spojeny jediným plochým kabelem a smontováni dohromady v kompaktní celek 
v mechanicky odolném a voděvzdorném pouzdře. %todo slide s pouzdrem a slidi s ledkruhem

Slide 9

BlackBox je taky díky svému světelnému kruhu schopen velmi dobře sloužit jako herní prvek při nočních hrách,
které jsou u táborníku velmi oblíbené.
Světelný kruh vévodí na přední straně trezoru a je složen z 60 inteligentních RGB ledek WS2812, které 
se dají řídit za sebe a tak jsem schopen celí kruh o 60 lledkách ovládat pouhým jedním pinem. Díky této jejich dovednosti si jich mohu dovolit 60,
jinak by jejich počet nemohl přesáhnout několik jednotek. Číslo 60 jsem zvolil v návaznosti na šedesát 
minut v hodině. Jedna z dřívějších variant pak měla ledek 12 v návaznosti 
na ciferníkové obyčejných hodiny,
to však bylo málo a tak jsem jejich počet zvedl na zmíněních 60.
Na rozdíl od dnes využívaných displejů jak alfanumerických tak grafických je tento displai dobře viditelný po tmě i na velké vzdálenosti 
a zárověn podporuje fantazii dětí při vymýšlení různých způsobů zobrazení a v neposlední řadě je poměrně levný a mechanicky odolný. % testování vzdálenost čitelnosti za šera za zamračena atd.



Slide 10

Jeden z dalších problémů které BlackBox řeší je nabíjení baterii. Na táborech jsme totiž vždy museli
řešit nabíjení externím zařízením a protože jsme jich neměli dostatek byl to vždy velmi zdlouhaví proces.
BlackBox má ale integrovanou nabíječku a tak jej stačí jen připojit kabelem stejně jako třeba mobil.

Slide 11

S dětmi jsme také naráželi na problémy se stavbou, primárně s pájením, které málokterý 
účastník zvládl bez podstatného zásahu vedoucího. Z tohoto důvodu jsem se snažil snížit na minimum 
nutnost pájení a naprostou většinu součástek tak strojně osazuji u firmy JLCPCB u které zároveň i vyrábím 
desky plošných spojů.



Slide 12

Zde můžete vidět blokové schéma rozvržení napájení.
Můžete si všimnout že BlackBox má tři různá napájecí napětí, za prvé vstupní které poskytují dva li-on články 18650, 
ze druhé napětí 5V pro světelný kruh a motor západky, a v neposlední řadě 3.3V pro napájení většiny desky.

Také si můžete všimnout že napětí zdroje není využíváno jen pro výroby 5 a 3.3V ale je na něj ještě před zapínáním 
napojeno RTC, které se nikdy nevypíná a za zapínáním IR příjmač který pro svou funkci vyžaduje o něco vyšší napětí.

Slide 13

Na tomto blokové schéma můžete vidět zapojení všech užitých čipů i jiných periferii.
Opět si můžete povšimnout několika věcí, např. že ESP je programováno pomocí převodníku USB UART a konektoru USB-C,
nebo že všechny sofistykovanější čipy komunikují s ESP po sběrnici I2C.

Pro maximální možnou interaktivitu má totiž systém senzory jako gyroskop, barometr nebo RTC. 
Asi nejzajímavějším senzorem ja však tlaková plocha která je založena čipu LDC1614, 

Slide 14

což je čip který 
pomocí čtyř cívek měří vzdálenost vodivého terčíku na principu vířivých (Foucaultův) proudů. tlaková plocha je pak vlastně 
jen destička která se při používání naklání podle toho kde jí uživatel stlačí. Uživatel je tak pomocí tlakové plochý
schopen komunikovat nejen pomocí pozice kde plochu stlačí ale i pomocí síli jakou na plochu tlačí. 



Slide 15

Elektroniku jsem původě navrhoval v programu Eagle, ale pozdeji jsem přešel na program KiCad, primárně kvůli 
modulu KiKitu. Což je nástroj, který zásadně zjednodušuje přípravu podkladů pro reálnou výrobu desek. 

Slide 16

Mechanické části jsem pak navrhoval v programu fusion 360, ve kterém již nekolik let pracuji a jsem s ním velmi spokojený.
První verze systému jsem pak vyráběl kombinací technologie laserové řezačky a 3D tisku, a to jak technologií FDM tak SLA.
Po odzkoušení konstrukce jsem u dříve tisknutých dílů přešel z 3D tisku na odlévání z polyuretanu. 
Tuto část v textu mé práce nenajdete, protože
to není elektronika a práce mi už tak přišla hodně dlouhá.

Slide 17

BlackBox by ale nikdy nefungoval bez vhodného programu, ten však není náplní mojí práce ale vytvořil jej v samostatně 
můj kolega Tomáš Rohlínek. Tuto práci by jste mohli najít v oboru informatiky pod názvem Software pro BlackBox.

Slide 18

A co jsem vlastně vytvořil? 
BlackBox je zařízení vhodné pro výuku robotiky, především programování a dá se velmi dobře použít i při návrhu rozsáhlých 
kolektivních her. Řeší řadu problémů které jsme řešili v zájmových kroužcích i na táborech a dodává i něco navrch.

%*******************
Vyvinul jsem úžasnou věc s množstvím aplikací. 


Slide 19

Na konec bych rád poděkoval svému školiteli Miroslavu Burdovi za vytrvalou pomoc, Jihomoravskému kraji za finanční 
podporu a vám děkuji za pozornost a rád zodpovím vaše dotazy, nazhledanou.


%********************
********************
u slidu důvody práce 

očekávám jako divák, že se u zde zastavíš a jednotlivé body okomentuješ 


slide blackbox 
u toho rendru to na první pohled vypadá jak vozítko + není dané měřítko - je větší než ty 
kromě vzhledu jsem se pořád nedozvěděl, co je a proč je blackbox 
(chce to např. něco jako vedle něj krabičku zápalek... ) 


slide ESP-32 - "my jsme potřebovali bezdrátovou komunikaci" 
NE 
Prezentuješ možným uživatelům: 
Např.: Jako organizátoři během hry v terénu potřebujete (nebo oceníte) 
možnost bezdrátové komunikace, protože ... 


zahození kabelů v 3:20 je velmi efektní -> klidně si do záběru dej i koš ... 
zlepšil bych formulaci "s tímhle můžeme udělat tohle" 

5:20 velká vzdálenost je kolik? 
cca 6:35 - moc pěkné odtlačování dalšího schématu 
napřed bych zařadil blokové schéma a až potom schéma napájení 
u blokového schématu by stálo za to jednou větou/více větami okomentovat schopnosti/důvod každého senzoru 
8:40  

-----------------
BtW: kolik takový blackbox stojí? 1.600,- Kč -> jestli není levnější 
pro městkou hru koupit 20x levný mobil a naprogramovat do něj appku.    
možná odpověď: mobil nemůže sloužit jako dveře od trezorku ...  + odolnost tlakové plochy 
