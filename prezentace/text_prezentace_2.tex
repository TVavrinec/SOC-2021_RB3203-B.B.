https://docs.google.com/presentation/d/1z_a-4hQFeqkxhdXX8CfWHh0_gHIVp0qsNtIAHn8cLWs/edit#slide=id.gc9a0d5e08f_2_101

1. Věta 2. slidu má být: Co je to blackbox? -> Představím na 1 minutu v kostce.  
ukážu ho v ruce, lze ho ukázat i jako součást nějaké skříňky? Ať je tam taky ten BOX  
a ne jenom jeho dveře, 
+ vysvětlit název -> reklama, nadsázka (např.: protože může pro účastníka kurzu nebo 
hry skrývat překvapivé možnosti (později rozvedu -> a navážu v předposledním slidu, 
např. hráč musí přijít na ovládání náklonem, aby se dostal dovnitř ... ).


3. slide - proč blackbox (výuka i hry)

4. slide - jak se blackbox vyvíjel ??? - možná ani ne ? 

5. slide (+ možná i pátý) - přehled elektroniky 

další slidy - jednotlivé části elektroniky podrobně 


předposlední slide/slidy - použití: počty lidí, .. varianty a nebo náměty her ... => KILLER-APP

poslední - závěr + Rohlínkovo programování  

úvod 30s závěr 30s 
co je to 1min 
proč 2 min 
tlakovka 1 min 
elektronika 2-3 min 
použití 1-2 min 


Slide 1

Dobrý den, jmenuji se Tomáš Vavrinec, jsem ze SPŠ a VOŠ Brno, Sokolská
a dnes vám budu prezentovat svojí práci SOČ s názvem BlackBox.

% 15 s

% Řekněte jim, co jim chcete říct. Řekněte jim to. Řekněte jim, co jste jim řekli. 

Slide 2
% 1. Věta 1. slidu má být: Co je to blackbox? -> Představím na 1 minutu v kostce.  
% ukážu ho v ruce, lze ho ukázat i jako součást nějaké skříňky? Ať je tam taky ten BOX  
% a ne jenom jeho dveře, 
% + vysvětlit název -> reklama, nadsázka (např.: protože může pro účastníka kurzu nebo 
% hry skrývat překvapivé možnosti (později rozvedu -> a navážu v předposledním slidu, 
% např. hráč musí přijít na ovládání náklonem, aby se dostal dovnitř ... ).

Toto je BlackBox, jeho koncept vychází z původní vize elektronického trezoru.
Z rané fáze mu zbyla schopnost v sobě zamykat primárně drobné předměty (ale dá se jednoduše upravit pro jednotlivé aplikace).
Postupem vývoje se na BlackBox přidávali různé senzory, dnešní varianta však bohužel nemá k dispozici senzorů nejvíce kůli 
pandemii jsem se totiž potýkal s obtížnou dostupností některých čipů. 
Senzoriku potřebuji pro maximální možnou interaktivitu. Přes to že BlackBox začal jako výuková stavebnice je také velmi dobře 
schopen sloužit jako herní prvek a právě k tomu využívá svou senzoriku. Díky té je totiž schopen například vydat odměnu, nebo
předmět potřební pro další čísti hri na základě řekněme správného natočení vůči zemi a zároveň vůči magnetickému pólu země.

A proč jméno BlackBox? no protože je bílej ne.

% 1minuta (prakticky přesně)

Slide 3 
Toto téma jsem si zvolil z několika důvodů. Za prvé protože se již nějakou dobu zabývám 
výukou robotiky v zájmových kroužcích a na robotických táborech. 

% K čemu je dobrý Blackbox 
    Slide 3.1

    Z toho důvodu potřebuji 
    vhodné stavebnice, se kterými se účastníci mohou učit, a na táborech ideálně i hrát hry.
    Stavebnice jsou pro výuku robotiky ideální, protože účastníkům umožňují práci jak 
    s hárdwarem tak jeho programem.

    Slide 3.2

    Jednou z takovýchto stavebnic byl i robot SchoolBot, o kterém byla moje loňská práce SOČ.
    SchoolBot byl roborické vozítko, podobných je však na trhu spoustu a i mi jsme jich již měli v kroužcích
    a na táborech několik. Tyto vozítka jsou ale vždy vázány na hladký povrch a jinde se s nimi jezdit nedalo.
    Jakékoli teréní úpravy těchto vozítek pak byly velmi drahé a ani tak nebyli sto projet všude.

% 1minuta (prakticky přesně)

Slide 4

Asi víte o tom že existuje mnoho již navržených stavebnic založeny například na známém Arduinu, 
ke kterým jde připojit celá řada modulů. Abi toho nebylo málo tyto moduly se dají často sehnat velmi levně. 
Proč jsem potom vlastně BlackBox vůbec stavěl? Proč jsem jen nesestavil dohromady několik desek Arduina? 
Při sestavování různých desek
totiž ne vždy vím jak konkrétně jsou navrženy a něco mi tak může mezi sebou kolidovat.
Primárně jde však o fakt že takováto zařízení jsou vetšinou velmi nepřehledná a obtížně.
Na obrázku vlevo si můžete udělat představu o tom ja nepřehledné může být zapojení pouhých dvou čipů, 
na obrázku vpravo je pak příklad zapojení vice těchto modulů. V podobné změti kabelu pak stačí abi jeden jediný drát 
ztratil kontakt nebo huř ztratil ho jen částečně a některý z vedoucích např. na táboře stráví půl dne hledáním problému 
na jednom zařízení, zatím co by měl pomáhat všem dětem v místnosti.

% 1minuta (prakticky přesně)

Slide 5

Z tohoto důvodu je BlackBox navržen tak aby k tento problém pokud možno nebyl. 
Vše je přehledně umístěno na dvou kruhových deská z nichž na hlavní desce je procesor, power management, většina
senzoriky... a vlastně vůbec většina elektroniky. Na druhé desce je pak světelný kruh a snímání tlakové plochy, 
o obojím vám něco řeknu za chvíli.
Obě desky jsou pak přehledně spojeny jediným plochým kabelem a smontováni dohromady v kompaktní celek 
v mechanicky odolném a v zamčeném stavu dokonce vodotěsném pouzdře. %todo slide s pouzdrem a slidi s ledkruhem

% 42 s

Slide 6.0

Dřívější stavebnice také vetšinou neměli možnost bezdrátové komunikace např. dálkového ovládáním.
To totiž uživatel častokráte potřebuje a museli to řešit externími modulů.

Z tohoto důvodu uživatel jistě ocení že je BlackBox založen na čipu ESP32, 
který má vedle výkoného dvaatřicetibitového procesor i integrovaný WiFi a Blutooth modul. 
Jeden BlackBox je tak schopen se například spojit s jiným BlackBoxem a sdílet libovolná data, 
nebo se třeba spojit s mobilem uživatele a nechat se jím dálkově ovládat. % fotku 

% 36 s
Slide 6.1

Další problém který uživatelé a primárně organizátoři větších akcí mají je nabíjení baterii. 
Představte si např. že máte za večer nabýt dvě baterie pro každé z padesát dětí. Určitě je vám jasné že pokud 
nemáte dostatek nabíječek stojíte před velmi zdlouhavým úkolem.
Proto jistě oceníte i fakt že BlackBox disponuje integrovanou nabíječku a tak jej stačí 
jen připojit kabelem stejně jako třeba telefon.

% 28 s
Slide 6.2

Další problém který často řeší vedoucí robotických kurzu a táborů je pájení. Málo který účastní 
je totiž sto si sám zapájet robota bez zásadní podpory od některého z vedoucích. Z tohoto důvodu jsem 
se snažil snížit na minimum nutnost pájení a určitě vám tak přijde vhod že naprostou většinu součástek 
osazuji strojně u firmy JLCPCB u které zároveň i vyrábím všechny desky plošných spojů. % Aby však děti nepřišli o pájení 
% úplně mohou si ho vyzkoušet na jednodušších zapojeních, například na blikačích, který když zničí nic se neděje.

% 28 s
% dohromady 1 m 27 s 
Slide 7

Primárně organizátoři her pak jistě ocení světelný kruhu který je schopen velmi dobře sloužit i při nočních hrách,
které jsou u táborníku velmi oblíbené.
Je složen z 60 inteligentních RGB ledek WS2812, které mají schopnost se řetězit za sebe a tak mi na celí kruh o 60 lledkách
stačí jeden jediný pinem procesoru. Kdybi ledky tuto schopnost neměli rozhodně bych si jich nemohl dovolit 60,
ale spíš jen několik jednotek. Číslo 60 jsem pak zvolil v návaznosti na šedesát 
minut v hodině aby se na něm dal dobře zobrazovat čas nebo třeba výstup z magnetického kompasu. 
Jedna z dřívějších variant pak měla ledek 12 v návaznosti na ciferníkové obyčejných hodiny,
to však bylo málo a tak jsem jejich počet zvedl na zmíněních 60.
Na rozdíl od dnes využívaných displejů aď už alfanumerických tak grafických je tento display dobře viditelný i po tmě na velké vzdálenosti, 
zárověn podporuje fantazii dětí při vymýšlení různých způsobů zobrazení a v neposlední řadě je poměrně levný a mechanicky odolný. % testování vzdálenost čitelnosti za šera za zamračena atd.

% 1 m 10 s

Slide 8

Na tomto blokovém schéma můžete vidět zapojení všech užitých senzoru a jiných logických komponent.
Můžete si např. povšimnout že ESP je programováno pomocí převodníku USB UART a konektoru USB-C,
nebo že čipy jako QMC5883 nebo M41T62 komunikují s ESP pomocí sběrnice I2C.

Pro maximální možnou interaktivitu má totiž systém senzory jako gyroskop, akcelerometr nebo barometr. 
Asi nejzajímavějším senzorem ja však tlaková plocha která je založena na čipu LDC1614, 

% 41 s

Slide 9

což je čip který 
pomocí čtyř cívek měří vzdálenost vodivého terčíku. tlaková plocha je pak vlastně 
jen vodivá deska která se při používání naklání podle toho kde jí uživatel stlačí. 

Na rozdíl od dnešních dotykových displejů,
které pravděpodobně většina z vás má na svých telefonech, je tlaková plocha schopna detekovat ne jen pozici doteku ale také jeho sílu,
a to velmi přesně. Ve správné aplikaci je totiž LDC1614 schopno měřit vzdálenost s přesností vyšší než na desetinu mikrometru.

Dokonce jsem díky této přesnosti neplánovaně schopen tuto plochu využívat i jako mikrováhu.

% 42 s  otázka jestli z tohoto slidu neudělat část slidu 8?

Slide 10

Zde můžete vidět blokové schéma rozvržení napájení.
Můžete si všimnout že BlackBox má tři různá napájecí napětí, za prvé vstupní které poskytují dva li-on články 18650, 
ze druhé napětí 5V které poskytuje step up měnič založení na čipu fp6276. V neposlední řadě je to pak napětí 3.3V,
které je produktem lineárního stabilizátoru LD39200 a slouží pro napájení většiny elektroniky.
Pětivoltová větev pak slouží pro napájení světelného kruh a motor západky.

Také si můžete všimnout že napětí zdroje není využíváno jen pro výroby 5 a 3.3V ale je na něj ještě před zapínáním 
napojeno RTC, které se nikdy nevypíná a za zapínáním IR příjmač který pro svou funkci vyžaduje o něco vyšší napětí.

% 1 m

Slide 11

Elektroniku jsem původě navrhoval v programu Eagle, ale pozdeji jsem přešel na program KiCad, primárně kvůli možnosti
využívat modul KiKit. Což je nástroj, který zásadně zjednodušuje přípravu podkladů pro reálnou výrobu desek. 

Mechanické části jsem pak navrhoval v programu fusion 360, ve kterém již nekolik let pracuji a jsem s ním velmi spokojený.
První verze systému jsem pak vyráběl kombinací technologie laserové řezačky a 3D tisku, a to jak technologií FDM tak SLA.
Po odzkoušení konstrukce jsem u dříve tisknutých dílů přešel z 3D tisku na odlévání z polyuretanu. 

Tuto část však v textu mé práce nenajdete, nejedná se totiž o elektronika a práce mi i tak přišla velmi dlouhá.

% 50 s

Slide 13

A co jsem vlastně vytvořil? 
BlackBox je zařízení vhodné pro výuku robotiky, především programování a dá se také velmi dobře použít i při návrhu rozsáhlých 
kolektivních her. Řeší řadu problémů které jsme měli v zájmových kroužcích i na robotických táborech a dodává i něco navrch.
Disponuje řadou senzorů, možností bezdrátové komunikace a to dokonce třemi způsoby, silným procesorem který pro vás asi nikdy nebude omezením
a periferiemi jako světelný kruh nebo zámek které pro vás bude obě velmi důležitým prvkem budete-li chtít s BlackBoxem něco hrát.

V realitě zatím kvůli světové pandemii byli nasazeni jen dřívější vývojové varianty a to jen v omezené míře, i tak však s nějakou verzí BlackBoxu 
pracovalo alespoň 110 lidí a spíš více bohužel přesné číslo neznám. BlackBox má totiž v plánu využívat DDM.Helceletova Brno na řadě svých poboček.
Některé verze BlackBoxu se již takto testovaly a to jak při využití jako stavebnice tak jako herní prvek. Určitě mohu říct že se BlackBox objeví 
na letošním robotickém táboře který pořádá pobočka DDM.Helceletova Brno, Robotárna a který je každoročně naplněn na svou kapacitu 50 účastníku. %zanedbávám ty co odpadnou den předem kvůli ledsčemu 

% 1 m 24 s

% všechno je to přibližné
% Příměšťák řekněme 10 lidí, Robotika pokročilí 2019/20 řekněme 10 lidí, trpaslíci 20, Jirkova akce 10, Kubova akce 20, robotábor 2020 50 (jen na jednom stanovišti ale i tak ;*) )

%*******************
% Vyvinul jsem úžasnou věc s množstvím aplikací. 


Slide 19

BlackBox by ale nikdy nefungoval bez vhodného programu, ten však není náplní mojí práce ale vytvořil jej v samostatně 
práci můj kolega Tomáš Rohlínek. Tuto práci by jste mohli najít v oboru informatiky pod názvem Software pro BlackBox.
Já mu tímto děkuji že BlackBoxu vdechl život.

Na konec bych také rád poděkoval svému školiteli Miroslavu Burdovi za vytrvalou pomoc a neskutečnou trpělivost s opravou mích gramatických chyb, 
také bych rád poděkoval Jihomoravskému kraji za finanční podporu, Robotárně za skvělé zázemí 
a vám děkuji za pozornost a rád zodpovím vaše dotazy, nazhledanou.

% 46 s

%********************
u slidu důvody práce 

-očekávám jako divák, že se u zde zastavíš a jednotlivé body okomentuješ 


slide blackbox 
u toho rendru to na první pohled vypadá jak vozítko + není dané měřítko - je větší než ty 
kromě vzhledu jsem se pořád nedozvěděl, co je a proč je blackbox 
(chce to např. něco jako vedle něj krabičku zápalek... ) 


slide ESP-32 - "my jsme potřebovali bezdrátovou komunikaci" 
NE 
Prezentuješ možným uživatelům: 
Např.: Jako organizátoři během hry v terénu potřebujete (nebo oceníte) 
možnost bezdrátové komunikace, protože ... 


zahození kabelů v 3:20 je velmi efektní -> klidně si do záběru dej i koš ... 
zlepšil bych formulaci "s tímhle můžeme udělat tohle" %jak? jakože nevím ale u podobného vtipu mě moc nenapadá lepší varianta. Jakože myslíš třeba "a tohle můžeme hodit do koše" nebo tak? 

5:20 velká vzdálenost je kolik? % se musí zjístit  
cca 6:35 - moc pěkné odtlačování dalšího schématu 
napřed bych zařadil blokové schéma a až potom schéma napájení 
u blokového schématu by stálo za to jednou větou/více větami okomentovat schopnosti/důvod každého senzoru 
8:40  

-----------------
BtW: kolik takový blackbox stojí? 1.600,- Kč -> jestli není levnější 
pro městkou hru koupit 20x levný mobil a naprogramovat do něj appku.    
možná odpověď: mobil nemůže sloužit jako dveře od trezorku ...  + odolnost tlakové plochy -> dále není tak jednoduché to programovat, učíme robotiku a ne programování apek pro android, dálě děcka nevidí elektroniku a nemůžou tak mít pocit že si to sami vyrobili, ještě sem taky neviděl mobil s barometrem (možná jsou ale rozhodně to není běžná záležitost), mobil taky rozhodně nebude tolik odolnej (minimálně ty levný) a když už se něco posere tak ho nemáme šanci spravit. Každopádně si ale myslím že tohle nemám čas zmiňovat v prezentaci.