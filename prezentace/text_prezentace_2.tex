zahájeni
{
    pozdrav
    jméno
    původ
    název práce
}-> Dobrý den, jmenuji se Tomáš Vavrinec, jsem ze Střední průmyslové školy a Vyšší odborná školy Brno, Sokolská
    a dnes vám budu prezentovat 
    svojí práci SOČ s názvem Elektronický trezor. % Platforma pro pedagogiku táborových her

úvod
{
    důvod práce
    cíl práce
    nástin konstrukce
}-> Toto téma jsem si zvolil z několika důvodů. Za prvé protože se již nějakou dobu zabývám 
    výukou robotiky v zájmových kroužcích a na robotických táborech. Z toho důvodu potřebuji 
    vhodné stavebnice, se kterými se účastníci mohou učit, a na táborech ideálně i hrát hry. 
    Jednou z takovýchto stavebnic byl i robot SchoolBot, o kterém byla moje loňská práce SOČ.
    Trezor však na rozdíl od SchoolBotu není robotické vozítko, kterých jsme již měli na Robotárně
    spoustu. 
    Tyto vozítka byli totiž vždy vázány na hladký povrch a jinde se s nimi jezdit nedalo.
    Vedle toho byli jakékoli teréní úpravy těchto vozítek velmi drahé a ani ty nebyli sto projet všude.
    Tyto problémy BlackBox z pochopitelných důvodu nemá, není to totiž vozítko ale zařízení které uživatel
    sám nosí v ruce.
    Dříve jsme také měli velmi podstatný problém s dálkovým ovládáním a obeceně bezdrátovým připojením.
    To jsme totiž častokráte potřebovali a museli jsme to řešit externími součástkami.
    BlackBox je částečně i proto založen na čipu ESP32 který má integrovaný WiFi a Blutooth modul.
    Díky možnosti bezdrátové komunikace je tak více BlackBoxů schopno se mezi sebou spojit a sdílet 
    libovolná data nebo se třeba spojit s mobilem uživatele a nechat se jím dálkově ovládat.
    


    Dalším důvodem vedle výuky byla možnost 
    takto získat jedinečné zařízení i pro nejrůznější kolektivní hry a v neposlední řadě 
    moje záliba v návrhu elektroniky. 
    Cílem práce se tedy stal jakýsi herní trezor. Zařízení které by bylo schopno uchovávat 
    menší předměty. Aby mělo rozmanitou senzoriku, díky které by se dalo používat při 
    nejrůznějších disciplínách. Zároveň aby bylo pokud možno malé a pohodlně se vlezlo 
    do jedné ruky. To jsem vyřešil úplným oddělením dveří od těla trezoru a umožnil 
    tak používat samostatně dveře, ve kterých je veškerá elektronika.

popis B.B.
{
    M3
        existuje
        pro mladší a je levnější
    E4
        disponuje ....
            ESP32 wrover
                2xXtensa LX6
                    32bitu
                    240MHz
                wrover protože PSRam
                WiFi
                Bluetooth
            LED kruh
                60xWS2812
            senzorika
                % IR
                enkodér  
                barometr
                RTC
                akcelerometr
                gyroskop
                kompas
                    BMX/MPU-QMC
                Tlaková deska
                    LDC1614                 %vynést do oblak xD
            power management
                2x1860       -> zdroj, IR přijímač a pro napájení RTC
                stabilizátor -> pro napájení všeho s výjimkou LED, motoru a RTC
                stepup       -> pro LED a motor
            programátor
                CP2102 
                možnost zákazu programování
                USB-C
        rozloženo
            leddeska
            hlavní deska + enkodér
        návrh elektra
            první verze v Eaglu
            pak KiCad protože KiKit
        návrh mechaniky
            Fusion 360
            tisk na SLA a následně odlévaní z polyuretanu
        Software
            spolupracující SOČ, software a knihovny pro firmware vyvinul Tomáš Rohlínek 
                    v jazyce C++
}-> Vedle Trezoru vybaveného elektronikou, jsem navrhl i trezor čistě mechanický, 
    pro mladší účastníky kroužku nebo menší stavební akce.
    
    Elektronický trezor je pak založen na čipu ESP32, od firmy Espresif, který disponuje 
    dvěma járdy procesoru Xtensa LX6, 
    což je dvaatřicetibitový procesor, který běží na taktu 240MHz. 
    Čip ESP32 je zde uvnitř modulu wrover, konkrétně tento modul jsem zvolil pro dodatečnou 
    PSRAM, která rozšiřuje RAM o 8MB
    což je 16ti násobek původní velikosti.
    Čip také disponuje Wifi a bluetooth. To jsou pro mě velice důležité funkce a jsou 
    jedním z hlavních důvodů volby tohoto čipu.

    ---

    Trezoru na přední straně vévodí velký světelný krh, složení z 60ti inteligentních ledek 
    WS2812. Tento kruh slouží vlastně jako display na kterém má trezor možnost zobrazovat 
    vše co potřebuje. Konkrétně toto číslo  a pokud by mu nedostačoval muže se pomocí WiFi nebo Bluetooth spojit 
    třeba s mobilem a zobrazovat data tam. 

    Zařízení má k dispozici rozmanité senzory, a to
        enkodér sloužící pro zjištění polohy motoru zápdky,
        Barometr který je schopen detekovat změnu nadmořské výšky
        s rozlišením na půl metru. % na slide napsat -> rozlišení 0.06 hpa
        Dále obsahuje RTC, Ril Time Clock v překladu hodiny reálného času, 
        které uchovávají přesný čas i ve chvíli kdy je trezor vypnutý.
        
        Dalšími senzory jsou akcelerometr, gyroskop a magnetický kompas, který na první verzi 
        trezoru zajišťoval čip BMX055, ale pro jeho obtížnou dostupnost jsem v další verzi 
        k tomuto čipu přidal čipy MPU6050 a QMC5883. Z nich 
        MPU zajišťuje akcelerometr s gyroskopem a QMC magnetický kompas.
        Díky těmto inerciálním senzorům lze trezor ovládat pomocí nejrůznějších pohybových 
        i statických gest.
        
        Asi nejpodstatnějším senzorem je čip LDC1614, případně se dá použít i jeho méně přesná verze LDC1314.
        Na tomto senzoru je založen hlavní ovládací prvek systému, kterým je tlaková plocha vyplňující střed 
        signalizačního kruhu. %Čip LDC1614 je totiž schopen velmi přesně měřit indukčnost, na desce má 
        % k dispozici čtyři cívky, realizovány jako reliéf vyleptaný přímo v mědi. Tlaková deska je pak 
        % vlastně jen kousek vodivého plechu, v mém případě hliníku, který při naklánění a stlačování mění 
        % vzdálenost od jednotlivých cívek. Cívky proto mění svoji indukčnost, 
        % což LDC zaznamená a dá o to vědět ESP32. Čip LDC1614 je natolik přesný, že je ve vhodné aplikaci 
        % schopen detekovat pohyb terčíku s rozlišením na 10nm, já ovšem tuto možnost nevyužívám a přesnost 
        % měřícího 28bitového 
        % AD převodníku využívám radši pro vetší rozsah vzdáleností než pro maximální přesnost měření.

        Tlaková plocha je také schopna bez problému odolat i ráně pěstí. To je u herního prvku velice užitečné
        jelikož děti i někteří dospělí v zápalu hry často používají vetší sílu než je potřeba.

    Celý sytém je napájen dvěma paralelně zapojenými li-on články 18650, systém je ale sto bez omezení pracovat i s jedním článkem, 
    dva články jsou zde čistě kvůli kapacitě a využití volného místa.
    
    Přímo z baterií je ale napájen jen IR přijímač obvod RTC. Trezor je totiž schopen komunikovat i po IR. 
    Čip M41T62 neboli RTC (real time clock), který se nikdy nevypíná, protože uchovává aktuální čas. %možná líp navázat? 
    Pro všechny ostatní periferie je zde lineární stabilizátor respektive stepup konvertor, pro napájecí větve 3.3 a 5V.
    Stepup napájí primárně světelný kruh, ale je s něj napájen i motor zámku, aby se pokud možno choval stejně bez ohledu na stav baterií.

    Lineární stabilizátor pak napájí vše ostatní, co má trezor k dispozici. % na slaid hodit výpis

    Aby se trezor dal jednoduše programovat, má k dispozici převodník USB UART, který slouží jako programátor.
    Protože trezor by nemělo být snadné přeprogramovat ve chvíli, kdy uživatel nechce, je tento čip vybaven 
    i obvodem, který je schopen zakázat možnost využití programátoru. Zároveň tento obvod šetří energii, protože 
    převodníku poskytne napájení jen ve chvíli, kdy je připojen programovací kabel. 
    Samotné připojení je realizován pomocí USB-C konektoru.

    Všechny tyto systémy jsou pak rozložil na dvě kruhové desky.
    Na jedné desce je jen světelný kruh a LDC spolu se snímacími cívkami.
    Na druhé desce je vše ostatní, zvláštním případem je pak enkodér, 
    který se sice vyrábí jako součíst hlavní desky ale je realizován jako vylomitelná část hlavní desky. %doladit 

    Elektroniku jsem původě navrhoval v programu Eagle, ale pozdeji jsem přešel na program KiCad kvůli modulu KiKitu. 
    Kikit je nástroj, který zásadně zjednodušuje výrobu podkladů pro výrobu desek. 
    

    Mechanické části jsem pak navrhoval v programu fusion 360, ve kterém již dlouho pracuji a který umožňuje jednoduchý 
    export různých podkladů pro výrobu na CNC strojích.
    První verze systému jsem pak vyráběl kombinací technologie laserové řezačky a 3D tisku, jak technologií FDM tak SLA.
    Po odzkoušení konstrukce jsem přešel na variantu odlévání tisknutelných částí z polyuretanu. 
    Tuto část v textu mé práce nenajdete, protože
    to není elektronika a práce mi už tak přišla hodně dlouhá.

    Moje práce by nikdy nebyla funkční bez softwaru, který ale není součástí mojí práce. Je totiž obsahem samostatné práce SOČ
    s názvem SDK pro výuku programování na IoT stavebnici, % taky lepší název 
    kterou vypracoval můj kolega Tomáš Rohlínek a soutěží s ní letos v oboru Informatiky.
    Napsal mimo jiné knihovny v jazyce C++ pro snadnou manipulaci se všemi periferiemi trezoru.

využití a zhodnocení
{
    M3
        pro jednoduchost se stihly akce
        jel na něm kroužek
    E3
        rozjela se v kroužku, ale pak pandemie
    E1
        příměšťák
    E4
        pandemie
}-> Kvůli jednoduchosti byla mechanická verze trezoru hotová mnohem dřív než verze elektronická, tedy ještě před započetím světové pandemie.
    Z tohoto důvodu se stihla stát náplní jednoho kroužku, který jsem spoluvedl. Když byl mechanický trezor v kroužku postaven,
     pokračovali jsme se třetí variantou elektronického 
    trezoru. Bohužel kvůli pandemii byl kroužek loni v říjnu po pouhých dvou schůzkách přerušen 
    a na dálku jsme nebyli sto pokračovat v jeho stavbě.

    Úplně první verze trezoru, který jsem navrhl, byl s úspěchem použit jednom příměstském táboře ještě v roce 2019.

    Nynější verze trezoru za sebou ještě bohužel kvůli pandemii žádné akce nemá. Doufám ale, že bude možno brzy nějakou zorganizovat.

    Celkem trezor využilo .... dětí/účastníků/stavitelů 

    Pro období po skončení pandemie je v plánu využití na DDM Helceletova v rozsahu: .... účastníků kroužku a her ročně. 
    -> Papír od Jirky Váchy s důrazem na využití.  % (možná má v plánu městskou hru, jen co bude pandemická opatření odvolána )
    -> Chystáme celotáborou hru pro 50 účastníků. 

Děkuji Vám za pozornost a prosím Vaše dotazy. 