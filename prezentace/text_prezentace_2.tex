https://docs.google.com/presentation/d/1z_a-4hQFeqkxhdXX8CfWHh0_gHIVp0qsNtIAHn8cLWs/edit#slide=id.gc9a0d5e08f_2_101

1. Věta 2. slidu má být: Co je to blackbox? -> Představím na 1 minutu v kostce.  
ukážu ho v ruce, lze ho ukázat i jako součást nějaké skříňky? Ať je tam taky ten BOX  
a ne jenom jeho dveře, 
+ vysvětlit název -> reklama, nadsázka (např.: protože může pro účastníka kurzu nebo 
hry skrývat překvapivé možnosti (později rozvedu -> a navážu v předposledním slidu, 
např. hráč musí přijít na ovládání náklonem, aby se dostal dovnitř ... ).


3. slide - proč blackbox (výuka i hry)

4. slide - jak se blackbox vyvíjel ??? - možná ani ne ? 

5. slide (+ možná i pátý) - přehled elektroniky 

další slidy - jednotlivé části elektroniky podrobně % hehe to není podrobně ani v textu


předposlední slide/slidy - použití: počty lidí, .. varianty a nebo náměty her ... => KILLER-APP

poslední - závěr + Rohlínkovo programování  

úvod 30s závěr 30s 
co je to 1min 
proč 2 min 
tlakovka 1 min 
elektronika 2-3 min 
použití 1-2 min 

% Řekněte jim, co jim chcete říct. Řekněte jim to. Řekněte jim, co jste jim řekli. 


%použití použití použití

Slide 1 (pozdrav)

Dobrý den, mé jméno je Tomáš Vavrinec, jsem studentem SPŠ a VOŠ Brno, Sokolská
a dnes bych vám rád prezentoval svojí práci SOČ se jménem BlackBox.

% 15 s

Slide 2 (BlackBox)

Toto je BlackBox, jeho koncept vychází z původní vize elektronického trezoru.
Z rané fáze jeho vývoje má schopnost v sobě zamykat menší předměty řekněme velikosti jabka. %(ale dá se jednoduše upravit i pro větší).
BlackBox mel původně být jen atraktivní stavebnicí pro výuku mechanické stavby a programování. 
% Obojí je totiž v robotice velmi zásadní částí návrhu robotů.
Postupem vývoje však BlackBox získával různé senzory. 
% Senzoriku potřebuji pro maximální možnou interaktivitu.
Přesto, že začal jako výuková stavebnice 
v kroužcích na DDM Helceletka, je také velmi dobře 
schopen sloužit jako herní prvek a právě k tomu využívá svou senzoriku. 
Díky té je totiž schopen například vydat odměnu, nebo
předmět potřebný pro další části hry na základě řekněme správného náklonu 
a zároveň předem stanoveného času.

Během vývoje se také oddělily dveře od těla a dá se s nimi tak pracovat samostatně.

A proč jméno BlackBox? No protože je bílej, ne. % a kulatej 

% 1minuta (prakticky přesně)

Slide 3 (proč)
Do vývoje a hlavně reálné stavby Blackboxu jsem se pustil z několika důvodů. 
Za prvé, protože se již nějakou dobu zabývám 
výukou robotiky v zájmových kroužcích a na robotických táborech. 

% K čemu je dobrý Blackbox 
    Slide 3.1

    Z toho důvodu potřebuji 
    vhodné stavebnice, se kterými se účastníci mohou učit, a ideálně i hrát hry.
    Stavebnice jsou pro výuku robotiky ideální, protože účastníkům umožňují práci jak 
    s hardwarem, tak se softwarem. 

    Slide 3.2

    Jednou z takovýchto stavebnic byl i robot SchoolBot, o kterém byla moje loňská práce SOČ.
    SchoolBot byl robotické vozítko, které bylo možné stavět a programovat.  
    Podobná vozítka jsou ale vždy vázána na hladký povrch a jinde se s nimi jezdit nedalo.
    % Jakékoli teréní úpravy těchto vozítek pak byly velmi drahé a ani tak nebyla vozítka sto projet všude.
    Robotické vozítko se taky nedá jen tak někde zahrabat nebo pověsit na strom a už vůbec ne třeba do deště,
    toto však u B.B není problém.

% 1minuta (prakticky přesně)

Slide 4 (změť kabelů)

Asi víte o tom, že existuje mnoho již navržených stavebnic založených například na známém Arduinu, 
ke kterým jde připojit celá řada modulů. Aby toho nebylo málo, tyto moduly jsou často velmi levné. 
Proč jsem potom vlastně BlackBox vůbec stavěl? Proč jsem jen nesestavil dohromady několik desek Arduina? 
Při sestavování různých desek
totiž ne vždy vím, jak konkrétně jsou navrženy a jednotlivé desky mi tak můžou mezi sebou kolidovat.
Primárně jde však o fakt, že takováto zařízení jsou vetšinou velmi nepřehledná a velmi obtížně se v nich 
hledají chyby. Na obrázku vlevo si můžete udělat představu o tom, jak nepřehledné může být zapojení 
pouhých dvou modulů, na obrázku vpravo je pak příklad zapojení vice těchto modulů. 
V podobné změti kabelu pak stačí, aby jeden jediný drát 
ztratil kontakt, nebo hůř, ztratil ho jen částečně a některý z vedoucích 
např. na táboře stráví půl dne hledáním problému 
na jednom zařízení, zatím co by měl pomáhat všem dětem v místnosti.

% 1minuta (prakticky přesně)

Slide 5 

Z těchto všech důvodů je BlackBox navržen tak, že  % aby tyto problémy pokud možno nebyly. 
veškerá elektronika je přehledně umístěna na dvou deskách spojených jediným plochým kabelem.
Na hlavní desce je procesor, power management, většina
senzoriky... a vlastně vůbec většina elektroniky. Na druhé desce je pak světelný kruh
 a snímání tlakové plochy.
 
% 42 s
Slide 6

Především organizátoři her pak jistě ocení světelný kruh, který je schopen velmi dobře sloužit 
i při nočních hrách, které jsou u táborníků velmi oblíbené.
Je složen z 60 inteligentních RGB ledek WS2812, které mají schopnost se řetězit za sebe 
a tak mi na celý kruh o 60 ledkách stačí jediný pin procesoru. 
Číslo 60 jsem pak zvolil v návaznosti na šedesát minut v hodině, 
aby se na tomto kruhu dal dobře zobrazovat čas nebo třeba výstup z magnetického kompasu. 
Na rozdíl od dnes využívaných displejů, ať už alfanumerických nebo grafických, 
je tento display dobře viditelný i po tmě na velké vzdálenosti, 
zároveň podporuje fantazii dětí při vymýšlení různých způsobů zobrazení 
a v neposlední řadě je poměrně levný a mechanicky odolný. 
% testování vzdálenost čitelnosti za šera za zamračena atd. Řekl bych ideálně i kapitolku do textu

% 1 m 10 s

Slide 7

Tlakovou plochu si můžete představit jako touchpad nebo dotykový displej, který pravděpodobně máte na svém telefonu.
Tlaková plocha však není jen tak ledajaký touchpad -  je totiž krom pozice doteku 
schopen měřit i sílu stisku, a to velmi přesně.
Díky přesnosti měření jsem dokonce neplánovaně schopen tuto plochu využívat i jako mikrováhu. 
% by se hodilo laboratorní měření do textu, grami jsem schopnej vážit bez problému už teď 
% a na miligrami to chce z Rohlajze vytáhnout lepšejší filtraci ale každopádně když koukáš 
% na křivku v Lorris tak si schopnej poznat i váhu 0806 kondenzátoru
Je založena na čipu LDC1614, od firmy texas instrument, který
pomocí čtyř cívek měří vzdálenost vodivého terčíku. Tlaková plocha je pak vlastně 
jen vodivá deska, která se při používání naklání podle toho, kde ji uživatel stlačí.
Navíc vydrží i ránu pěstí, aby odolala do hry zapáleným dětem.

Slide 8

Dřívější stavebnice také vetšinou neměly možnost bezdrátové komunikace např. dálkového ovládání.
Tu totiž uživatel častokrát potřebuje a musel to řešit externími moduly.

Proto uživatel jistě ocení, že je BlackBox založen na čipu ESP32, 
který má vedle dvou výkoných dvaatřicetibitových procesorů i integrovaný WiFi a Blutooth modul. 
Jeden BlackBox je tak schopen se například spojit s jiným BlackBoxem a sdílet libovolná data, 
nebo se třeba spojit s mobilem uživatele a nechat se jím dálkově ovládat. % fotku 

% 36 s
% Slide 9
% 
% Další problém, který uživatelé a primárně organizátoři větších akcí mají, je nabíjení baterií. 
% Představte si např. že máte za večer nabít dvě baterie pro každé z padesáti dětí. Určitě je vám jasné, že pokud 
% nemáte dostatek nabíječek, stojíte před neřešitelným problémem.
% Proto jistě oceníte i fakt že BlackBox disponuje integrovanou nabíječku a tak jej stačí 
% jen připojit kabelem stejně jako třeba telefon.

% 28 s
    % Slide 8.2
    % 
    % Další problém který často řeší vedoucí robotických kurzu a táborů je pájení. Málo který účastní 
    % je totiž sto si sám zapájet robota bez zásadní podpory od některého z vedoucích. Z tohoto důvodu jsem 
    % se snažil snížit na minimum nutnost pájení a určitě vám tak přijde vhod že naprostou většinu součástek 
    % osazuji strojně u firmy JLCPCB u které zároveň i vyrábím všechny desky plošných spojů. % Aby však děti nepřišli o pájení 
% úplně mohou si ho vyzkoušet na jednodušších zapojeních, například na blikačích, který když zničí nic se neděje.

% 28 s
% dohromady 1 m 27 s 

Slide 10

Na tomto blokovém schématu můžete vidět zapojení všech použitých senzorů a jiných logických komponent.
Můžete si např. povšimnout že ESP je programováno pomocí převodníku USB UART a konektoru USB-C,
nebo možnosti IR komunikace, která může také sloužit pro jednoznačnou identifikaci dveří. 
% *******************************************************************
%přidat ostatní součásti 

Pro maximální možnou interaktivitu má Blackbox řadu senzorů, 
konkrétně gyroskop, akcelerometr, barometr, magnetický kompas a RTC připojených pomocí sběrnice I2C. 
Asi nejzajímavějším senzorem ja však dříve zmíněná tlaková plocha tlaková plocha reprezentovaná čipem LDC1614. 

% 41 s


% 42 s  otázka jestli z tohoto slidu neudělat část slidu 9?

Slide 11

Zde můžete vidět blokové schéma rozvržení napájení.
Můžete si všimnout, že BlackBox má tři různá napájecí napětí, za prvé vstupní, 
které poskytují dva Li-on články 18650, 
ze druhé napětí 5V, které poskytuje step up měnič založený na čipu fp6276. 
V neposlední řadě je to pak napětí 3.3V,
které je produktem lineárního stabilizátoru LD39200 a slouží pro napájení většiny elektroniky.
Pětivoltová větev pak slouží pro napájení světelného kruhu a motoru západky.

Také si můžete všimnout, že napětí zdroje není využíváno jen pro výrobu 5 a 3.3V, 
ale je na něj ještě před zapínacím obvodem 
napojeno RTC, které se nikdy nevypíná a za zapínáním je připojen IR příjmač, 
který pro svou funkci vyžaduje o něco vyšší napětí.

% 1 m

Slide 12

%Elektroniku jsem původě navrhoval v programu Eagle, ale pozdeji jsem přešel na program KiCad, primárně kvůli možnosti
%využívat modul KiKit. Což je nástroj, který zásadně zjednodušuje přípravu podkladů pro reálnou výrobu desek. 

Mechanickou stránku věci jsem navrhoval v programu fusion 360, ve kterém již pracuji několik let.
Jak můžete vidět je zámek založen na mechanizmu bajonetu a z toho důvodu jsou desky s elektronikou kruhové.
Take si můžete všimnou že hlavní část jsou dveře které se tak dají použít s různými těly, dle potřeby.
První verze systému jsem pak vyráběl kombinací technologie laserové řezačky a 3D tisku, 
a to jak technologií FDM tak SLA.
Po odzkoušení konstrukce jsem však u dříve tisknutých dílů přešel z 3D tisku na odlévání z polyuretanu. 

BlackBox je následně v celku schopen odolávat jak mechanickým nárazům tak i vodě aby se dal bez rizika 
používat třeba i za deště nebo u potoka. 

%Tuto část však v textu mé práce nenajdete, 
%nejedná se totiž o elektroniku a práce mi i tak přišla velmi dlouhá.% ????????????????????????????

% **********************************************************
    %   různá (voděodolná těla? )
    %   a smontovány dohromady v kompaktní celek v mechanicky odolném a v zamčeném stavu dokonce vodotěsném pouzdře.
    %   proč kruhový tvar? 
% **********************************************************

% 50 s

Slide 13

BlackBox je zařízení vhodné pro výuku robotiky, 
především programování a dá se také velmi dobře použít i při návrhu rozsáhlých 
zážitkovích akcí. Řeší řadu problémů které jsme měli v zájmových kroužcích i na 
robotických táborech a dodává i řadu funkcí navrch.
Disponuje řadou senzorů, možností bezdrátové komunikace a to dokonce třemi způsoby, 
silným procesorem, který pro vás asi nikdy nebude omezením
a periferiemi jako světelný kruh nebo zámek, které pro vás bude obě velmi důležitým 
prvkem, budete-li chtít s BlackBoxem něco hrát.

V realitě zatím kvůli pandemii byly nasazeny jen dřívější vývojové varianty 
a to jen v omezené míře, i tak však s nějakou verzí BlackBoxu 
pracovalo alespoň 110 lidí.
BlackBox má totiž v plánu využívat DDM Helceletova Brno na řadě svých poboček.
Některé verze BlackBoxu se již takto testovaly a to při využití jako stavebnice 
i jako herní prvek. Určitě mohu říct že se BlackBox objeví 
na letošním robotickém táboře, který je každoročně naplněn na svou kapacitu 50 účastníků. 

% 1 m 24 s

Slide 14

Na závěr bych také rád poděkoval svému školiteli Miroslavu Burdovi za vytrvalou pomoc
také bych rád poděkoval Jihomoravskému kraji za finanční podporu, 
Robotárně za skvělé zázemí 
a vám děkuji za pozornost a rád zodpovím vaše dotazy, nashledanou.

slide 15 

% BlackBox by nikdy nefungoval bez vhodného programu, ten však není náplní mojí práce,
% ale vytvořil jej v samostatné práci můj kolega Tomáš Rohlínek. 
% Jeho práci můžete najít v oboru informatiky pod názvem Software pro BlackBox.
% Já mu tímto děkuji, že BlackBoxu vdechl život.

% 46 s

%komplet 11 m 5s zkrácená verze (-Slide 6.2 a sem tam nějaká veta)
%komplet 12 m 10 s celé
%********************

zahození kabelů v 3:20 je velmi efektní -> klidně si do záběru dej i koš ... 
zlepšil bych formulaci "s tímhle můžeme udělat tohle" %jak? jakože nevím ale u podobného vtipu mě moc nenapadá lepší varianta. Jakože myslíš třeba "a tohle můžeme hodit do koše" nebo tak? 

5:20 velká vzdálenost je kolik? % se musí zjístit  
cca 6:35 - moc pěkné odtlačování dalšího schématu 
napřed bych zařadil blokové schéma a až potom schéma napájení 
u blokového schématu by stálo za to jednou větou/více větami okomentovat schopnosti/důvod každého senzoru 
8:40  

-----------------
BtW: kolik takový blackbox stojí? 1.600,- Kč -> jestli není levnější 
pro městkou hru koupit 20x levný mobil a naprogramovat do něj appku.    
možná odpověď: mobil nemůže sloužit jako dveře od trezorku ...  + odolnost tlakové plochy -> dále není tak jednoduché to programovat, učíme robotiku a ne programování apek pro android, dálě děcka nevidí elektroniku a nemůžou tak mít pocit že si to sami vyrobili, ještě sem taky neviděl mobil s barometrem (možná jsou ale rozhodně to není běžná záležitost), mobil taky rozhodně nebude tolik odolnej (minimálně ty levný) a když už se něco posere tak ho nemáme šanci spravit. Každopádně si ale myslím že tohle nemám čas zmiňovat v prezentaci.