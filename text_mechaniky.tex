\documentclass{template/priloha}

\usepackage{subcaption} 
\usepackage{amsmath} 
\usepackage{enumitem} 
\usepackage{hyperref} % reference
\usepackage{gensymb} % balíček symbolů
\usepackage{booktabs}

\usepackage[toc,page]{appendix}
\usepackage{color} % balíček pro obarvování textů
\usepackage{xcolor}  % zapne možnost používání barev, mj. pro \definecolor
\definecolor{mygreen}{RGB}{0,150,0} % nastavení barev odkazů 
\definecolor{myblue}{RGB}{0,0,200} % nastavení barev odkazů 

\usepackage{listings} % balíček pro formátování zdrojových kódů 
\usepackage[author=,status=final]{fixme} % vkládání poznámek  
% dva módy (status): draft (poznámky se zobrazují v PDF) / final (poznámky se nezobrazují v PDF)
\usepackage{multirow}
\usepackage{hyperref} % pro vkládání odkazu

\usepackage{wrapfig}
\usepackage{epsfig}

\usepackage{blindtext}
\usepackage[T1]{fontenc}
\usepackage[utf8]{inputenc}
\usepackage[czech]{babel}
\usepackage{caption}
\usepackage{biblatex}
\usepackage{graphicx}
\usepackage{fancyhdr}
%\usepackage[backend=biber, style=iso-numeric,sorting=ynt]{biblatex}

\hypersetup{colorlinks=true, linkcolor=myblue, urlcolor=mygreen, citecolor=blue, anchorcolor = magenta,
	 linktocpage = true, frenchlinks } % nastavení barvy odkazů 


\lstset { %
    language=C++,
    backgroundcolor=\color{black!5}, % set backgroundcolor
    basicstyle=\footnotesize  % basic font setting
}

\newcommand{\obr}[1]{[obr. \ref{#1}/str. \pageref{#1}]}
\newcommand{\chapterm}[1]{
    
    \chapter{#1}

    \vspace{-10mm}
}

\addbibresource{kapitoly/text_mecha.bib} % soubor s bibliografií
\nocite{*}

\titlecz{Mechanický BlackBox} % český název práce
\titleen{Mechanical BlackBox} % anglický název práce
\originaltitlecz{BlackBox}
\author{Tomáš Vavrinec} % jméno a příjmení autora
\field{10} % obor (pouze číslo, zbytek vysází šablona - číslo oboru viz http://www.soc.cz/obory-soc/)
\school{Střední průmyslová škola a~Vyšší odborná škola Brno, Sokolská, příspěvková organizace} % celý název školy
\mentor{Mgr. Miroslav Burda} % jméno a příjmení školitele
\mentorstatement{Mgr. Miroslava Burdy} % jméno a příjmení ve druhém pádě 

% Změňte, pokud se liší
%\region{Jihomoravský} % kraj
\placefooter{Brno 2021} % místo a rok
% hinty k používání balíčků hyperref, url, hyperlink a hypertarget
% \usepackage{hyperref} % balíček pro hypertextové odkazy
% \url{www.odkaz.cz}
% \href{http://www.odkaz.cz}{Text který bude jako odkaz}
% \hyperlink{label}{proklikávací_text} - odkaz na text 
% \hypertarget{label}{cíl_odkazu} - cíl odkazu 
% konec preambule dokumentu

\fancyhf{}
\fancyhead[LO,LO]{
    % \begin{wrapfigure}{L}{0.4\textwidth}
    % {
    %     \vspace{-15mm}
    %     \includegraphics[width=0.15\textwidth]{kapitoly/obrazky/E4/B.B_Minibox.png}
    % }
    % \end{wrapfigure}
    \B{Mechanický BlackBox}
}

\fancyhead[LE,RO]{
    \B{příloha k~práci BlackBox}
}
\fancyfoot[CE,CO]{\leftmark}
\fancyfoot[LE,RO]{\thepage}

\renewcommand{\headrulewidth}{ 2pt } 
\renewcommand{\footrulewidth}{ 1pt }

\begin{document}

\maketitle

\pagestyle{empty}

\section*{Anotace}
\color{black}

Robotika se stává čím dál tím významnějším oborem, což s~sebou nese i~potřebu vzdělávání v~tomto oboru.
Pro naprosté začátečníky nebo lidi, kteří se nechtějí robotikou zabývat dlouhé stovky hodin, však nejsou vhodné 
relativně dražší elektronické stavebnice. Jedna z~nich je například elektronická varianta BlackBoxu, který má ohromné možnosti, 
ale to s~sebou pochopitelně nese i~potřebu jistých znalostí a~zkušeností.
Z~těchto důvodu jsem ze pustil i~do vývoje BlackBoxu v~čistě mechanické verzi.

\subsection*{Klíčová slova}

\color{black}

trezor, BlackBox, jednoduchá stavebnice, mechanická konstrukce % snadná robotika?

\newpage

\newpage

\tableofcontents % vysází obsah

\voffset = -40pt
\headsep = 20mm

\newpage

%%% Začátek práce
\setcounter{figure}{0}
\setcounter{table}{0}

\pagestyle{fancy}

\chapter{Úvod}
\thispagestyle{fancy}
Tato práce rozšiřuje informace o zařízeních BlackBox, což jsou primárně elektronická zařízení určená pro 
výuku robotiky, programování a jako platforma pro návrh a realizaci zážitkových akcí a táborových her.

Tato práce se věnuje méně sofistikovaným verzím BlackBoxu, které nemají žádnou elektroniku 
a jejich schopnosti jsou omezené jen na schopnost uzamčení menších předmětů, pomocí mechanicky určeného hesla.

Cílem práce je rozvést informace o mechanických verzích BlackBoxu, popsat jejich vývoj a možnosti jejich užití, 
včetně již existující reálné aplikace.

\chapter{Vývoj mechanického trezoru}
\thispagestyle{fancy}
\label{M-vyvoj}

\section*{první mechanická varianta}
\addcontentsline{toc}{section}{první mechanická varianta}
První, čistě mechanická varianta, vznikla začátkem srpna 2019, chvíli po výše obšírněji popsané elektronické variantě.
Měla stále poměrně klasický vzhled trezoru, tedy zamykatelná skříňka, která obsahovala dvě kola, která ovládala možnost pohybu jednoduché západky.
Na rozdíl od jeho elektronického předchůdce bylo vše zajímavé uvnitř dveří. Také byla určená jako základ pro případný upgrade na elektronickou
variantu. Na podobné vylepšení mělo stačit odstranění kódovacích kol a přidělání elektronické části. Toto sice fungovalo obstojně, zároveň 
i jako motivace, ale kvůli pozdější změně konceptu mechanizmu tento nápad padl.
Tato varianta však nebyla, kvůli přílišným nárokům na přesnost, vhodná pro stavbu s malými dětmi, pro které byla určena jakožto předstupeň 
k variantě elektronické (která vyžaduje i znalosti, nebo alespoň ochotu k učení, programování).

\begin{figure}[htbp]
    \centering
    \includegraphics[width=400]{kapitoly/obrazky/M1-mechanizmus.png}
    \caption{zelená značí kódová kola, červená západku, modrá pevnou část trezoru(otvor) a žluté díly tvoří distanci}
    \label{fig:M1}
\end{figure}
\newpage
\section*{druhá mechanická varianta}
\addcontentsline{toc}{section}{druhá mechanická varianta}

Druhá mechanická varianta je až na drobnosti stejná jako verze dnešní.
Ovládá se pěti koly, z nichž čtyři zajišťují heslo a páté otáčí s rotační západkou, které drží dveře na svém místě.
Tato varianta tedy přichází z možností dveře úplně oddělit od skříně trezoru. To by při využití jako trezor, který
má za úkol jen ochraňovat svůj obsah, sice nepřinášelo žádný velký užitek, ale při mém využití, spíše jako herní 
prvek než trezor, to může být užitečné.

\begin{figure}[htbp]
    \centering
    \includegraphics[width=150]{kapitoly/obrazky/M2-mechanizmus.png}
    \includegraphics[width=150]{kapitoly/obrazky/M2-mechanizmus_zamceno.png}
    \label{fig:M1}
\end{figure}

\begin{figure}[htbp]
    \centering
    \includegraphics[width=\textwidth]{kapitoly/obrazky/M2-render.PNG}
    %\caption{}
    \label{fig:M1.0}
\end{figure}


\newpage
\section{Dnešní mechanická varianta}

Dnešní mechanická varianta je téměř stejná jako druhá verze rozdíl je jen v uložení kol, které kolem hřídelů získalo distanční kroužky, které
zjednodušují lepení. 

\begin{figure}[htbp]
    \centering
    \includegraphics[width=70pt]{kapitoly/obrazky/M3/predni_render.png}
    \caption{render varianty M3}
    \label{fig:M3-render}
\end{figure}

\begin{figure}[htbp]
    \centering
    \includegraphics[width=220pt]{kapitoly/obrazky/M3/rez.png}
    \caption{Řez kódovacím kolem}
    \label{fig:M3-rez-kolem}
\end{figure}

\newpage

\chapter{Mechanický trezor} 
\thispagestyle{fancy}
\label{M3}

\label{M3}
Dnes využívaná varianta je právě M3 s mechanizmem zámku ukázaným na \obr{fig:M3}.
V přílohách práce jsou také k této variantě výrobní výkresy.

\begin{figure}[h]
	\centering
    \includegraphics[width=\textwidth]{kapitoly/obrazky/M3/SOC_render.png}
    \caption{Vzhled mechanizmu zamykání u~mechanické verze}
    \label{fig:M3}
\end{figure}

\section{Popis jednotlivých součástek a důvody konkrétního tvaru}

Trezor má tvar krychle a~délku hrany má 128~mm, násobek šestnácti jsem zvolil kvůli jednoduché návaznosti na dřívka, %todo přidáme pár vět o dřívkách 
dřevěná dřívka s obdélníkovým průřezem 3x16~mm nebo 2x16~mm.
Protože je trezor vyroben z překližky o síle 4~mm, jsou jeho vnitřní rozměry o 4~mm na každé straně menší (takže 122~mm).

\paragraph{Geometrie západky}
\begin{wrapfigure}{R}[0.2\textwidth]{0.7\textwidth}
    \includegraphics[width=0.7\textwidth]{kapitoly/obrazky/M3/geometrie_zapadky.png}
    \caption{náčrt západky} %todo čeho náčrt? 
    \label{fig:M3-geometrie-zapadky}
\end{wrapfigure} %todo zvážil bych obrázek centrovaný a obtékaný pouze nahoře a dole 

Protože se západka otáčí musí jí být zajištěn dostatek prostoru, zároveň však otvor pro dveře je lepší mít větší, protože se potom trezor dá použít pro větší objekty.
Z tohoto důvod jsou hrany západky definovány kružnicí o~průměru, délky vnitřní hrany trezoru. Západka má v~rozích sražení ze dvou důvodů. Za prvé aby byl otvor pro
dveře větší a~za~druhé aby namáhání působící v~západce působilo na větší délce.

\newpage

\paragraph{Distanční deska}

\begin{wrapfigure}{R}[0.2\textwidth]{0.7\textwidth}
    \includegraphics[width=0.7\textwidth]{kapitoly/obrazky/M3/distancka.png}
    \caption{náčrt Distanční desky}
    \label{fig:M3-distancka}
\end{wrapfigure}

Abi se západka dostala za desku přední stěny bedny trezoru je potřeba jí od přední stěny dveří posunout právě o tloušťku stěny. To zajišťuje jednoduchá čtvercová deska jen s~pěti otvory
pro průchod ovládacích kol.

\paragraph{Kámen} % jaksi mě nenapadl lepší název 
\begin{wrapfigure}{L}[0.2\textwidth]{0.7\textwidth}
    \includegraphics[width=0.7\textwidth]{kapitoly/obrazky/M3/kamen.png}
    \caption{náčrt kamene}
    \label{fig:M3-kamen}
\end{wrapfigure}

Kamen, který zajišťuje kód, má z~části tvar drážky, ve které jezdí, a~z~části kruh který se~muže otáčet v~kruhovém otvoru, na jedné staně drážky.
Uprostřed má~kruhoví otvor o~průměru 8mm pro kolík který kamenem otáčí.

\paragraph{Lepící distanční kroužek}

\begin{wrapfigure}{R}[0.2\textwidth]{0.7\textwidth}
    \includegraphics[width=0.7\textwidth]{kapitoly/obrazky/M3/lepici_distance.png}
    \caption{náčrt lepícího distančního kroužku}
    \label{fig:M3-lepici-distance}
\end{wrapfigure}

Tyto distanční kroužky jsou zde čistě z~technologického důvodu. Při lepení kolíku, totiž měli děti problém s~lepidlem, které jim zatékalo do~prostoru mezi kolíkem 
a~stěnou dveří čímž znemožňovalo otáčení kol. Proto jsem přidal tyto kroužky, do~kterých když zateče lepidlo tak~se nic neděje.

\newpage
\section{Odolnosti proti násilnému vniknutí}

\paragraph{Vytržení dveří}
Jedním ze způsobů~namáhání~mechanizmu~je~vytržení~dveří~z~trezoru.

\subsection{Západka}

\begin{figure}[htbp]
    \centering
    \includegraphics[width=\textwidth]{kapitoly/obrazky/M3/simulace/odolnost_proti_vytrzeni_4kN.png}
    \caption{simulace pokusu o vytržení dveří silou 4 000 N}
    \label{fig:M3-simulace-vytrzeni}
\end{figure}
Ke kompletní simulaci se můžete dostat \href{https://myhub.autodesk360.com/ue2d7aa41/g/shares/SH56a43QTfd62c1cd96843f1e03a0eb48053?viewState=NoIgbgDAdAjCA0IDeAdEAXAngBwKZoC40BlASwFsBXAGwEN1SB7AOzXjVoGdPd1C0ARjABsATlEQItALQBjcbmkAWCMIjSBuWgA5lAM22ilAVgAmMAOyy9%2BBGkYCAVrlnoAkqcIBmAL4gAukA}{zde}
po kliknutí na "Simulation" a "Simulation Model 2". V tabulce napravo se pak můžete přepínat mezi barevným zobrazení několika veličin.

\newpage

\subsection{Kolík}
Při pokusu o vytržení je celá síla přenášena kolíkem.

\(\sigma _M_A_X = 132 MPa\)    ( \href{https://is.mendelu.cz/eknihovna/opory/zobraz_cast.pl?fit_w=1;cast=9190}{dubové dřevo ve směru vláken při vlhkosti 12 \% }) % strana 22 tabulka 2 -> https://www.vutbr.cz/www_base/zav_prace_soubor_verejne.php?file_id=66237

D = 6

\(\sigma _M_A_X = F/S \Rightarrow F = \sigma _M_A_X * S = 132 * (\pi * D^2/4) = 3 732.21 N \) z toho a ze simulace vyplývá že kolík je při namáhání nejslabším členem, přesto že ne o moc.

\paragraph{Otevření bez odemčení}
Dalším způsobem namáhání může být snaha otočit západkou pez zadání správného hesla.

\subparagraph{Západka a kamen}

\begin{figure}[htbp]
    \centering
    \includegraphics[width=\textwidth]{kapitoly/obrazky/M3/simulace/odolnost_proti_nasilnemu_odemceni_10Nm.png}
    \caption{simulace pokusu o otevření bez předchozího odemčení při kroutícím momentu 10 000 Nmm, zobrazeno jen napětí nad 1 MPa}
    \label{fig:M3-simulace-vytrzeni}
\end{figure}

Ke kompletní simulaci se můžete dostat \href{https://myhub.autodesk360.com/ue2d7aa41/g/shares/SH56a43QTfd62c1cd96843f1e03a0eb48053?viewState=NoIgbgDAdAjCA0IDeAdEAXAngBwKZoC40BlASwFsBXAGwEN1SB7AOzXjVoGdPd1C0ARjABsATlEQItALQBjcbmkAWCMIjSBuWgA5lAM22ilAVgAmMAOyy9%2BBGkYCAVrlnoAkqcIBmAL4gAukA}{zde}
po kliknutí na "Simulation" a "Simulation Model 3". V tabulce napravo se pak můžete přepínat mezi barevným zobrazení několika veličin.

\subparagraph{Kolík}
Kroutící moment který je dřevěný kolík o průměru 6mm schopen přenést. 

\(\tau _M_A_X = 52.3 MPa\)    (\href{https://is.mendelu.cz/eknihovna/opory/zobraz_cast.pl?fit_w=1;cast=9190}{dubové dřevo ve směru vláken při vlhkosti 12 \% })

D = 6

\tau _M_A_X = \frac{M_K}{W_K} \Rightarrow M_K = \tau _MAX * W_K = \sigma _D * \frac{\pi * D^3}{16} 

\(M_K = 52.3 * \frac{\pi * 6^3}{16} = 2 218.16 N*mm \Rightarrow\) a ze simulace že kolík je při namáhání v krutu nejslabším místem \(\Rightarrow\) pro zvýšení odolnosti by bylo 
potřeba zvětšit kolík nebo změnit materiál.


\chapter{Závěr}
\thispagestyle{fancy}
Cílem této části mé práce bylo vyvinout čistě mechanický BlackBox pro mechanické stavby a~náplň různých 
kolektivních her. 
Cíle jsem dosáhl, zařízení je vyrobeno z překližky a dá se jednoduše sestavit za několik desítek minut. 

%todo   Seznam použitých odborných výrazů
%to-do      uvádí se významné odborné termíny s vysvětlením 
\newpage
\pagestyle{empty}
\pagestyle{plain}

\appendix
\phantomsection

\newcommand{\OdsazeniNadpisu}{10mm}
%\section{Vzhled druhé elektronické varianty} 
\begin{figure}

	
    %\vspace{\OdsazeniNadpisu}
    %\centering
    %\includegraphics[width=0.5\textwidth]{kapitoly/obrazky/E2/predni_render.png}

    %\appendix
    \chapter{Obrazová příloha}
    \section{Vzhled první mechanické varianty}
	\centering
	\includegraphics[width=\textwidth]{kapitoly/obrazky/M1/render.png}
	\caption{Render varianty M1}
	\label{fig:E3-renderi}
\end{figure}

\begin{figure}
    \section{Vzhled druhé elektronické varianty}
	\centering
	\includegraphics[width=\textwidth]{kapitoly/obrazky/E3/rendery.pdf}
	\caption{Render varianty M2}
	\label{fig:E3-renderi}
\end{figure}

\begin{figure}
	\section{Vzhled poslední mechanické varianty}
	\vspace{\OdsazeniNadpisu}
    \centering
    \includegraphics[width=\textwidth]{kapitoly/obrazky/M2/predni_render.PNG}
    \caption{Render varianty M3}
    \label{fig:M2-render}
\end{figure}


\phantomsection
\begin{figure}
    \chapter{Ostatní přílohy}
    \vspace{-30mm}
    \small
    \addtocounter{section}{1}
    \addcontentsline{toc}{section}{\protect\numberline{\thesection}Seznam obrázků} 
    \listoffigures
\end{figure}

% \phantomsection
% \normalsize
% \addtocounter{section}{1}
% \addcontentsline{toc}{section}{\protect\numberline{\thesection}Seznam tabulek}
% \listoftables

\phantomsection
\addtocounter{section}{1}
\addcontentsline{toc}{section}{\protect\numberline{\thesection}Literatura}
%\printbibheading
\printbibliography[title={Literatura}]

\end{document}