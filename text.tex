\documentclass{template/socthesis}

\usepackage{subcaption} 
\usepackage{amsmath} 
\usepackage{enumitem} 
\usepackage{hyperref} % reference
\usepackage{gensymb} % balíček symbolů
\usepackage{booktabs}

\usepackage[toc,page]{appendix}
\usepackage{color} % balíček pro obarvování textů
\usepackage{xcolor}  % zapne možnost používání barev, mj. pro \definecolor
\definecolor{mygreen}{RGB}{0,150,0} % nastavení barev odkazů 
\usepackage{listings} % balíček pro formátování zdrojových kódů 
\usepackage[author=,status=final]{fixme} % vkládání poznámek  
% dva módy (status): draft (poznámky se zobrazují v PDF) / final (poznámky se nezobrazují v PDF)
\usepackage{multirow}
\usepackage{hyperref} % pro vkládání odkazu

\lstset { %
    language=C++,
    backgroundcolor=\color{black!5}, % set backgroundcolor
    basicstyle=\footnotesize,% basic font setting
}

\addbibresource{text.bib} % soubor s bibliografií
\nocite{*}

\titlecz{Postav si svého druhého robota} % český název práce
\titleen{Build your second robot} % anglický název práce
\author{Tomáš Vavrinec} % jméno a příjmení autora
\field{7} % obor (pouze číslo, zbytek vysází šablona - číslo oboru viz http://www.soc.cz/obory-soc/)
\school{Střední průmyslová škola a~Vyšší odborná škola Brno, Sokolská, příspěvková organizace} % celý název školy
\mentor{Mgr. Miroslav Burda} % jméno a příjmení školitele
\mentorstatement{Mgr. Miroslava Burdy} % jméno a příjmení ve druhém pádě 

% Změňte, pokud se liší
%\region{Jihomoravský} % kraj
\placefooter{Brno 2020} % místo a rok

% hinty k používání balíčků hyperref, url, hyperlink a hypertarget
% \usepackage{hyperref} % balíček pro hypertextové odkazy
% \url{www.odkaz.cz}
% \href{http://www.odkaz.cz}{Text který bude jako odkaz}
% \hyperlink{label}{proklikávací_text} - odkaz na text 
% \hypertarget{label}{cíl_odkazu} - cíl odkazu 

\begin{document} % konec preambule dokumentu

\maketitle % vysází titulky

\makecopyrightstatement{V~Brně} % místo

% poděkování
\makethanks{Děkuji svému školiteli Mgr. Miroslavu Burdovi za obětavou pomoc, podnětné připomínky a~hlavně nekonečnou trpělivost, kterou mi během práce poskytoval.}

\pagestyle{empty}

\section*{Anotace}
% \color{mygreen}
% Anotace má za úkol stručně popsat cíle práce a velmi stručný úvod k tématu. 
% Většinou bývá použit první odstavec, nebo jiná část úvodu.
\color{black}

Robotika se stává čím dál tím významnějším oborem, což s sebou nese i potřebu vzdělávání v tomto oboru.
Při výuce robotiky jsou proto potřeba různé pomůcky na kterých se mohou žáci učit potřebné dovednosti. Jednou s takovýchto pomůcek 
by mohl být například SchoolBoard (viz práce Postav si svého prvního robota), ale pokročilejším studentům jiš tento hárdware nemusí
stačit. Proto jsem začal pracovat na novém systnému který má více možností.

\subsection*{Klíčová slova}

\color{black}

trezor, ESP32, ESP32 wrover, inteligentní ledky, WS2812, BMX055, LDC1614, LDC1314, open-source hardware

\newpage % pokud se anotace vleze na jednu stránku (což by měla), tento rádek zakomentuj

\vspace{20mm}

\section*{Annotation}
\color{black}

Robotics is becoming an increasingly important field, which brings with it the need for education in this field.
When teaching robotics, therefore, various aids are needed on which students can learn the necessary skills. Once with such aids
could be, for example, SchoolBoard (see the work Build Your First Robot), but for more advanced students this hardware may no 
longer need suffice. That's why I started working on a new system that has more options.

\subsection*{Keywords}
\color{black}
safe, ESP32, ESP32 wrover, smart leds, WS2812, BMX055, LDC1614, LDC1314, open-source hardware

\newpage
\pagestyle{plain}

\tableofcontents % vysází obsah

%%% Začátek práce
\setcounter{figure}{0}
\setcounter{table}{0}
\newpage

% zde můžeš s pomocí příkazu \input{cesta k souboru} vložit soubory; doporučuji každou větší kapitolu dát do samostatného souboru pro větší přehlednost


% Úvod práce

\chapter*{Úvod}
\addcontentsline{toc}{chapter}{Úvod}

Na konci července roku 2019 jsem dostal za úkol navrhnout výrobek pro děti na příměstský tábor pobočky DDM Helceletova Brno, Robotárny. %todo odkaz 
 Poža\-dav\-kem 
byla jednoduchá a levná konstrukce, kterou děti zvládnou sestavit za pár dní a ve zbytku času tábora si stihnou vyzkoušet základy programování 
s využitím tohoto výrobku. Z tohoto důvodu jsem začal vyvíjet elektronicky řízený trezor. Z původní vize trezoru se ale poměrně rychle vyvinulo
univerzální elektronické zařízení, kterému zůstala schopnost sloužit jako trezor. Také se přidala čistě mechanická varianta trezoru pro mladší účastníky táborů a volnočasových aktivit.

Trezor byl pro mě poněkud změnou oproti mé dřívějším práci, která se do té doby vždy točila kolem různých létajících nebo častěji jezdících robotů s velkým důrazem na orientaci v prostoru.
Trezor je oproti těmto vozítkům daleko statičtější, a protože se sám nepohybuje, má jeho vnímání prostoru jiné požadavky. Vozítka také vždy počítala s jistou univerzalitou senzoriky i~mechaniky,
zatímco trezor by měl být upravitelný jen po stránce softwaru. 

Další odlišností trezoru je menší konkurence, která je u různých robotických stavebnic poměrně veliká, jak si můžete přečíst 
v mé dřívější práci \href{https://github.com/TVavrinec/SOC-text/blob/master/SOČ.pdf}{Postav si svého prvního robota}.
%todo Jiné trezory (elektronizované, ve formě stavebnice pro děti/tábory) jsem zatím nikde nenašel. 


%todo popis, proč se oddělila mechanická a ele verze, původní záměr byl mít mecha verzi pro oba trezory stejnou  

 %todo vlastně vznikly trezory dva - odstaveček 


Na konci července roku 2019 jsem dostal za úkol navrhnout výrobek pro děti na příměstský tábor
pobočky DDM Helceletova Brno, Robotárny. 
Poža\-dav\-kem byla jednoduchá a levná konstrukce,
kterou děti zvládnou sestavit za pár dní a ve zbytku času tábora se jim ukážou základy programování
s~využitím tohoto výrobku. Proto, a také pro poněkud nižší věk účastníků, jsme se s vedoucím 
Robotárny, Jiřím Váchou, rozhodli jít cestou \uv{trezoru}. To byl rozdíl oproti našim běžným 
výrobkům, které většinou měly možnost pohybu, ale byly pro děti náročnější na výrobu
a pochopitelně i cena u nich šla nahoru.


% % todo tohle je spíš anotace, úvod musíme zásadně rozšířit
% třeba takhle? snažil jsem se tam procpat i tu minulo sočku.

% Úvod práce má za cíl uvést:
% \begin{itemize}
%     \item cíl práce
%     \item jak ho chcete dosáhnout
%     \item popis tématu práce, musí být výstižný, ale stručný a poutavý
% \end{itemize}

% Úvodu a závěru práce je třeba věnovat obzvláště velkou pozornost.
% Myslete na to, že úvod a někdy i závěr si porotce čte jako první, teprve potom, jestli ho práce zaujme se rozhodne, zda ji přečte celou.

\newpage

\chapter*{vývoj}
\addcontentsline{toc}{chapter}{vývoj}

Na konci července roku 2019 jsem dostal za úkol navrhnout výrobek pro děti na příměstský tábor
pobočky D.D.M.Helceletova Brno, Robotárny. Požadavkem byla jednoduchá a levná konstrukce,
kterou děti zvládnou sestavit za pár dní a ve zbytku času tábora, se jim ukážou základy programování
s využitím tohoto výrobku. Proto, a i pro poněkud nižší věk účastníků, jsme se s vedoucím 
Robotárny, Jirkou Váchou, rozhodli jít cestou "trezoru". To byl rozdíl oproti našim běžným 
výrobkům, které většinou měly možnost pohybu, ale byly pro děti náročnější na výrobu
a pochopitelně i cena u nich šla nahoru.

\chapter*{Úvod k vývoji}
\addcontentsline{toc}{chapter}{Úvod k vývoji}
%Úvod práce má za cíl uvést:
%\begin{itemize}
%    \item cíl práce
%    \item jak ho chcete dosáhnout
%    \item popis tématu práce, musí být výstižný, ale stručný a poutavý
%\end{itemize}

% Úvodu a závěru práce je třeba věnovat obzvláště velkou pozornost.
% Myslete na to, že úvod a někdy i závěr si porotce čte jako první, teprve potom, jestli ho práce zaujme se rozhodne, zda ji přečte celou.


Na konci července roku 2019 jsem dostal za úkol navrhnout výrobek pro děti na příměstský tábor
pobočky D.D.M.Helceletova Brno, Robotárny. Požadavkem byla jednoduchá a levná konstrukce,
kterou děti zvládnou sestavit za pár dní a ve zbytku času tábora, se jim ukážou základy programování
s využitím tohoto výrobku. Proto, a i pro poněkud nižší věk účastníků, jsme se s vedoucím 
Robotárny, Jirkou Váchou, rozhodli jít cestou "trezoru". To byl rozdíl oproti našim běžným 
výrobkům, které většinou měly možnost pohybu, ale byly pro děti náročnější na výrobu
a pochopitelně i cena u nich šla nahoru.

Dal jsem se tedy do kreslení trezoru, pochopitelně ne do nějaké nedobytné pevnosti, ale do malé
krabičky, na které se dají ukazovat principy elektronických zámků. Jelikož se mi na podobné
výrobky osvědčila jako materiál překližka, navrhoval jsem vše s úmyslem výroby z překližky 
za využití laseru. Konstrukce byla z velké části přizpůsobená dostupné elektronice, kterou 
jsem měl k dispozici, a která musela být stejně použita poněkud odlišně než jak byla zamýšlena.
Němel jsem totiž čas, a vlastně ani rozpočet, navrhovat a především vyrábět konkrétní elektroniku
pro výrobek, který se měl předložit dětem ani ne za týden. Použil jsem tedy starší univerzální 
desku ALKS (\href{https://github.com/RoboticsBrno/ArduinoLearningKitStarter}{Arduino Learnikg Kit Starter})
kterých jsem měl dostatečnou zásobu. Ovládací prvky, dvě tlačítka, dva potenciometry a tři
barevné ledky, tedy celý ALKS jsem umístil na horní stranu trezoru. ALKS má v původní variantě
tři tlačítka. Já jsem však jedno musel pomocí magnetu a jazýčkového magnetického konektoru použít
jako kontrolu, zda jsou dveře otevřeny či zavřeny. Jako zámek jsem pak použil obyčejné servo
SG90, které velice jednoduše zajelo svou páčkou do drážky ve dveřích, a tím jim zabránilo 
se otevřít. Celý systém pak napájela malá powerbanka, která se dala vyjmout a nabýt, 
a používala se i ve dvou dalších verzích. Tato konstrukce měla kvůli uspěchanému návrhu 
spoustu problémů. Většinou však šlo o problémy, které by nebylo těžké odstranit a nebylo
tedy třeba předělávat celý koncept návrhu. V těsném závěsu za touto elektronickou variantou,
jsem ale dostal požadavek i na čistě mechanickou verzi trezoru. To byl následně jeden z 
velkých důvodů velkých změn, a to i změny samotného konceptu zařízení.

\newpage
\section*{první mechanická varianta}
\addcontentsline{toc}{section}{první mechanická varianta}
První, čistě mechanická varianta, vznikla začátkem srpna 2019, chvíli po výše obšírněji popsané elektronické variantě.
Měla stále poměrně klasický vzhled trezoru, tedy zamykatelná skříňka, která obsahovala dvě kola, která ovládala možnost pohybu jednoduché západky.
Na rozdíl od jeho elektronického předchůdce bylo vše zajímavé uvnitř dveří. Také byla určená jako základ pro případný upgrade na elektronickou
variantu. Na podobné vylepšení mělo stačit odstranění kódovacích kol a přidělání elektronické části. Toto sice fungovalo obstojně, zároveň 
i jako motivace, ale kvůli pozdější změně konceptu mechanizmu tento nápad padl.
Tato varianta však nebyla, kvůli přílišným nárokům na přesnost, vhodná pro stavbu s malými dětmi, pro které byla určena jakožto předstupeň 
k variantě elektronické (která vyžaduje i znalosti, nebo alespoň ochotu k učení, programování).

\begin{figure}[htbp]
    \centering
    \includegraphics[width=400]{kapitoly/obrazky/M1-mechanizmus.png}
    \caption{zelená značí kódová kola, červená západku, modrá pevnou část trezoru(otvor) a žluté díly tvoří distanci}
    \label{fig:M1}
\end{figure}
\newpage
\section{Druhá elektronická varianta}


Druhá verze elektronické varianty testovala použitelnost signalizačního kruhu, o dvanácti ledkách, kolem uprostřed dveří
umístěného enkodéru. Jako základ trezoru jsem použil první mechanickou variantu, ze které jsem odstranil zamykací kola 
a doplnil ji o servo, řídící elektroniku a již zmíněný kruh ledek a enkodér.
Vzhledem k tomu, že se jednalo jen o hrubý prototyp, neměl specializovanou desku~a elektroniku  tedy tvořila jen změť kabelů 
a kousek univerzální desky, takže nemám elektronickou variantu tohoto zapojení. % [schéma jsem kreslil jen na tabuli a to asi rok 
% zpátky, takže když bych došel k závěru že je potřeba, tak se dá udělat, ale je to práce navíc a nepovažoval bych to za podstatné]

Trezor měl pro komunikaci s uživatelem tedy kruh o dvanácti ledkách a~jeden vstupní prvek, enkodér s tlačítkem.
Ovládání tedy bylo od tohoto odvozené a trezor se zmáčknutím zapnul a tlačítko pak dál sloužilo jako potvrzování výběru.
Člověk tak mohl pomocí enkodéru vybírat jedinou rozsvícenou ledku a stiskem potvrdit, vstupní kód tedy mohl vypadat 
například jako čas, a uživatel ho zadal na kruhu odvozeném od ručičkových hodin, proto právě dvanáct ledek.
Konkrétní ovládání je pochopitelně závislé na nahraném programu a mohlo by se tedy jednoduše změnit do libovolné podoby --
to co popisuji je jen konkrétní možnost, kterou jsem použil.

\begin{figure}[htbp]
    \centering
    \includegraphics[width=\textwidth]{kapitoly/obrazky/E2/predni_render.png}
    \caption{render varianty E2}
    \label{fig:E2-render}
\end{figure}

\newpage
\section*{druhá mechanická varianta}
\addcontentsline{toc}{section}{druhá mechanická varianta}

Druhá mechanická varianta je až na drobnosti stejná jako verze dnešní.
Ovládá se pěti koly, z nichž čtyři zajišťují heslo a páté otáčí s rotační západkou, které drží dveře na svém místě.
Tato varianta tedy přichází z možností dveře úplně oddělit od skříně trezoru. To by při využití jako trezor, který
má za úkol jen ochraňovat svůj obsah, sice nepřinášelo žádný velký užitek, ale při mém využití, spíše jako herní 
prvek než trezor, to může být užitečné.

\begin{figure}[htbp]
    \centering
    \includegraphics[width=150]{kapitoly/obrazky/M2-mechanizmus.png}
    \includegraphics[width=150]{kapitoly/obrazky/M2-mechanizmus_zamceno.png}
    \label{fig:M1}
\end{figure}

\begin{figure}[htbp]
    \centering
    \includegraphics[width=\textwidth]{kapitoly/obrazky/M2-render.PNG}
    %\caption{}
    \label{fig:M1.0}
\end{figure}


\newpage
\section*{Třetí elektronická varianta}
\addcontentsline{toc}{section}{Třetí elektronická varianta}

Třetí verze elektronické varianty do značné míry vycházela z předchozí, druhé verze, a dále na ní stavěla. Asi nejzjevnější změna bylo navýšení počtu 
ledek z dvanácti, jakožto hodiny, na šedesát jakožto minuty, což pochopitelně znamenalo i zvětšení kruhu. Na desku se ale přidaly i nové funkcionality,
a to gyroskop, pro možnost znalosti náklonu zařízení, akcelerometr, pro znalost směru a velikosti zrychlování, magnetický kompas, pro určení světových
stran, RTC (Real Time Clock, hodiny reálného času), pro znalost přesného času a také GPS pro možnost určení svojí polohy.
Také jsem použil, po vzoru mechanické varianty, rotační západku, což znamenalo, že na stejný trezor se daly použít jak mechanické tak 
elektronické dveře.

\begin{figure}[htbp]
    \centering
    \includegraphics[width=\textwidth]{kapitoly/obrazky/E3.0-render0.pdf}
    \label{fig:M1}
\end{figure}

Tato verze měla dvě podverze, které se lišily motorem.
\begin{figure}[htbp]
    \centering
    \includegraphics[width=300]{kapitoly/obrazky/motory/hodinovyStrojek.jpg }
    \includegraphics[width=300]{kapitoly/obrazky/motory/zluty_motor.jpg}
    \label{fig:M1}
\end{figure}

Přes velké množství funkcí jsem, kvůli několika věcem ale opět koncept přepracoval. Hlavním důvodem změn bylo náročné uložení rotační západky, 
které vyžadovalo ozubený věnec a několik dalších tisknutých dílů.

\newpage
\section{Dnešní mechanická varianta}

Dnešní mechanická varianta je téměř stejná jako druhá verze rozdíl je jen v uložení kol, které kolem hřídelů získalo distanční kroužky, které
zjednodušují lepení. 

\begin{figure}[htbp]
    \centering
    \includegraphics[width=70pt]{kapitoly/obrazky/M3/predni_render.png}
    \caption{render varianty M3}
    \label{fig:M3-render}
\end{figure}

\begin{figure}[htbp]
    \centering
    \includegraphics[width=220pt]{kapitoly/obrazky/M3/rez.png}
    \caption{Řez kódovacím kolem}
    \label{fig:M3-rez-kolem}
\end{figure}

\newpage
\section{Dnešní elektronická varianta}

Čtvrtá elektronická varianta byla co se elektroniky týče přímým pokračováním předchozí verze. 
Hlavní dvě věci, co se změnily, bylo ovládání a princip zamykání. 

\subsection*{Princip mechanizmu}

Zamykání je založeno na mechanizmu bajonetu a zamčení je zajištěno západ\-kou, která zabraňuje zpětnému otočení.
Západka je ovládána motorem, který otáčí magnetem a přitahuje nebo odpuzuje magnet na západce. Důvodem pro magnetické ovládání
byla možnost západku ovládat i přes pevnou stěnu, a~také pružné spojení, které takto vznikne, takže se trezor například dá zavřít, i~když
je už zamčen (když například dveře nejsou dovřeny).

\begin{figure}[htbp]
    \centering
    \includegraphics[width=170pt]{kapitoly/obrazky/E4/predni_render.png}
    \includegraphics[width=170pt]{kapitoly/obrazky/E4/zadni_render.png}
    \caption{rendery trezoru E4}
    \label{fig:E4-render}
\end{figure}

\subsection*{Shrnutí změn oproti minulé verzi}

Trezor získal možnost komunikace pomocí IR, pro možnost identifikace růz\-ných dveří, dále získal magnetický enkodér, pro možnost snazšího ovládání
motoru zámku. 
Další inovací byl programovací systém s USB-C, na místo USB-micro jako dřív. Tento programátor má možnost úplně si odpojit napáje\-ní, a to v rámci šetření 
energie, když ho trezor nevyužívá, a zároveň možnost zákazu přeprogramování.
Podstatnou změnou také bylo rozdělení elektroniky do dvou různých desek, protože na jedné by nebyl dostatek místa. Jedna deska tak obsahuje ledkový 
kruh a čip LDC1614 nebo LDC1314 se čtyřmi cívkami, které měří vzdálenost tlakové desky. Na druhé desce pak bylo vše ostatní, tedy procesor, akcelerometr,
gyroskop, magnetický kompas, RTC (Real Time Clock, hodiny reálného času), barometr, IR vysílač a přijímač, magnetický enkodér, programátor, řešení 
napájení, řízení motoru a nabíječka.

\subsection*{Ovládání}
Předchozí varianty měly jako hlavní ovládací prvek enkodér s tlačítkem, ten jsem v nynější variantě odstranil, aby přední stěna neměla tak velký 
výstupek. Proto jsem tento prvek nahradil indukční tlakovou deskou, která vyplnila vnitřek kruhu ledek. 
Zbytek ovládání víceméně přetrval, jen kvůli nedostatku času a pandemií způsobenému nedostatku součástek, trezor přišel o GPS. Na druhou stranu 
získal barometr s rozlišením schopným detekovat změnu výšky o půl metru.

\subsection*{Napájení}
Předchozí verzím sloužila jako napájení powerbanka. Ta však kladla poměrné velké omezení, dokázala poskytnout proud pouze jedné ampéry, a proto 
jsem jí nahradil vlastním zdrojem, dvěma bateriemi 18650. To však samozřejmě znamenalo nutnost vlastního řešení stabilizace napětí, díky čemuž 
trezor dostal stepup, který spíná napětí z 3,5~V až 4,2~V na 5~V, a původně stepdown, později lineární stabilizátor, který poskytoval 3,3 a 5~V. %todo zkotroluj prosím ta napětí
Trezor také dostal vlastní nabíječku, aby pro nabíjení baterií stačilo připojit kabel, stejně jako třeba u mobilu.

\newpage


\chapter*{Mechanická varianta}
\addcontentsline{toc}{chapter}{Mechanická varianta}



\newpage
\chapter*{Elektronicka varianta}
\addcontentsline{toc}{chapter}{Elektronicka varianta}

\section{Přehled}

Dnešní verze elektronického trezoru se zamyká pomocí mechanizmu bajonetu a~magneticky řízené zpětné západky. 

Elektronika je vybavena čipem ESP32 \parencite{ESP32}, \parencite{ESP32-WROVER-B},
který obsahuje dva procesory Xtensa LX6, WiFi a bluetooth. Dále je trezor vybaven čipem BMX055 \parencite{bmx055} nebo dvojicí čipů MPU6050 \parencite{mpu6050} 
a QMC5883 \parencite{qmc5883}, které poskytují 
gyroskop, akcelerometr a magnetický kompas. Dále je zde SPL06 \parencite{spl06}, barometr s rozlišením 0,06~hPa, což umožňuje rozeznat změnu nadmořské výšky 
o polovinu metru. Další systém trezoru je možnost IR komunikace, která je zde pro možnost jednoznačné identifikace dveří, ale pochopitelně může 
sloužit i pro jiný účel. Deska je také vybavena RTC a má vlastní programátor pro usnadnění programování. Vedle ESP32 je zde asi 
nejvýznamnějším čipem LDC1614 \parencite{LDC1614}, případně LDC1314, který umožňuje funkci tlakové plochy (viz kapitoly \ref{E4-mech_tlakovky}, \ref{E4-tlakovka}).

%todo doplnit popis a využití a možnosti tlakové desky 

\begin{table}[h]
    \centering
    \resizebox{\textwidth}{!}{%
    \begin{tabular}{@{}lll@{}} 
    \textbf{čip} & \textbf{popis} & \textbf{poznámky}
                                                                                    \\ \midrule
    \textbf{ESP32}              & dva procesory Xtensa LX6, WiFi a bluetooth    &                                                   \\
    \textbf{BMX055}             & gyroskop, akcelerometr, magnetický kompas     & možno nahradit dvojicí čipů MPU6050 a QMC5883     \\
    \textbf{SPL06}              & barometr                                      & rozlišení až 0,06hPa                              \\
    \textbf{IRM-H936 a IR led}  & IR komunikace                                 &                                                   \\
    \textbf{LDC1614}            & snímání tlakové desky                         & počítá se s možnou záměnou za LDC1314             \\
    \textbf{CP2102}             & programátor                                   & s hardwarově zajištěným odpojováním napájení      \\ \bottomrule
    \end{tabular}%
    }
    \caption{Shrnutí elektronického vybavení}
    \label{tab:shrnuti}
\end{table}

\newpage
\section{Mechanika tlakové desky}

Indukčně snímaná tlaková deska funguje díky čtyřem cívkám na desce plošných spojů, které mění svojí indukčnost podle vzdálenosti snímané desky, terčíku.
Z tohoto důvodu se terčík při používání naklání, čímž zároveň mění svojí vzdálenost od jednotlivých cívek. Z toho také plyne nutnost uložit terčík
částečně volně. Terčík je proto od snímací desky oddělen pružnou vložkou, která je zároveň předepnuta pomocí nažehlovací fólie, která kryje přední 
stranu dveří a spojuje terčík s čelní krycí deskou. Díky nažehlovací fólii je také přední část dveří voděodolná.

\begin{figure}[htbp]
    \centering
    \includegraphics[width=\textwidth]{kapitoly/obrazky/E4/machanika_tlakove_desky/rez_po_ose.pdf}
    \caption{Řez varianty E4}
    \label{fig:E4-rez}
\end{figure}

Tlaková deska zárově počítá s možností působení síly o velikosti až 500~N, což samozřejmě zároveň znamená, že tělo dveří tomuto zatížení musí odolat.
Vzhledem k tomu, že nemám možnost vyrobit tělo z kovu a jsem odkázán na 3D tisk a laserovou řezačku, a zároveň chci mít dveře co možná nejmenší,
musel jsem napočítat kritické části napřesno. Z tohoto důvodu jsem v programu Fusion 360, ve kterém jsem trezor vyvíjel,
dělal simulaci, kterou zde přikládám. %todo obrázky se simulací do přílohy, a zde odkaz na č. obr a stránku, podobně i další velké obrázky (nad cca třetinu strany) 

Jako materiál těla jsem v první fázi zvolil standardní fotopolymer pro tiskárny typu SLA, s pevností v tahu 46 až 67 MPa.
V budoucnu bych ale chtěl tělo odlévat z nějakého houževnatého polyuretanu, aby se zlevnila výroba a zároveň stoupla odolnost.

\begin{figure}[htbp]
    \centering
    \includegraphics[width=370pt]{kapitoly/obrazky/E4/machanika_tlakove_desky/simulace/F100N,primo,uprostred,pohled_zepredu.png}
    \caption{Pevnostní simulace těla}
    \includegraphics[width=370pt]{kapitoly/obrazky/E4/machanika_tlakove_desky/simulace/F100N,primo,uprostred,pohled_zezadu.png}
    \caption{Pevnostní simulace těla pohled zezadu}
    Tato simulace testuje působení síly přímo na tělo, což není působení, které by v provozu nastávalo. Takovéto namáhání je ale o dost náročnější
    než to, které by reálně nastalo.
    \label{fig:E4-simulace_tela} %todo těla čeho? trezor je hranatý? 
\end{figure}

\begin{figure}[htbp]
    \centering
    \includegraphics[width=\textwidth]{kapitoly/obrazky/E4/machanika_tlakove_desky/simulace/zjednodusena_sestava_pri_F100N_nezobrazeno_napeti_pod_1,5MPa.png}
    \caption{Simulace sestavy}
    Jak je vidět, tak i sílu 100~N dokaže sendvič z terčíku, pružné podložky a~snímací desky rozložit na dostatečnou plochu, aby napětí v těle nestouplo 
    nad cca 3~MPa. Na obrázku je zobrazené jen napětí nad 1,5~MPa.
    \label{fig:E4-simulace_tlakovky}
\end{figure}

\clearpage
\newpage
\subsection*{dnešní elektronická varianta}
\addcontentsline{toc}{subsection}{dnešní elektronická varianta}

%    Západka
%        a) vývoj západky
%            1) původní verze (pružnost materiálu -> po čase ztrácí pružnost)
%            2) verze s pružinou (pružina je nepotřebná a jsou s ní zbytečné problémy)
%            3) b,c,d,e
%        b) obrázek
%        c) uhel čela (tak aby měla dosedací plocha co nejmenší vůli a zároveň opravdu dosedala v celé ploše [a ne v úseče nebo dokonce jednom bodě])
%            1) obrázek
%        d) pohyby
%            1) pohyb tam
%            2) pohyb spět (stačí magnet pružina je nepotřebná)
%            3) motor
%            4) enkoder
%            5) možná video?
%        e) matika
%            1) odolnost proti zpětnému otočení
%            2) obrázek ze simulace
%            3) možná video?
%            4) další návrhy?


\newpage

\begin{figure}[htbp]
    \section*{Ukosy}
    \addcontentsline{toc}{section}{Ukosy}    
    \centering
    \includegraphics[width=400]{kapitoly/obrazky/E4/ukozy/ukladaci_ukosy.pdf}
    \caption{ukládací úkosy}
    Aby bylo jednoduší při zavírání dveře správně natočit, mají zarážky na vnitřní straně velké úkosy, které tak zvětšují na vnitřní straně 
    vůli a při zasouvání navedou dveře do správné pozice.
    \label{fig:E4-ukosy}
\end{figure}

\begin{figure}[htbp]
    \centering
    \includegraphics[width=400]{kapitoly/obrazky/E4/ukozy/simetrie_zarazek.png}
    \caption{symetrie zarážky}
    Zarážky na obvodu otvoru mají obě kontaktní plochy stejné. Sice by mohlo být výhodné přizpůsobit tvar strany, kolem které, se pohybuje západka, 
    pohybu západky. Západka by tak mohla mýt vedení v průběhu celého pohybu. Pro symetrii jsem se však rozhodl kvůli možnosti díl s otvorem otočit.
    To je výhodné při stavbě s dětmi, kvůli zmenšení počtu chyb kterých se děti můžou při stavbě dopustit, a stráta vedení není tak zásadní.
    \label{fig:E4-simetrie_zarazky}
\end{figure}

\clearpage
\newpage
\section*{Elektronika tlakové desky}
\addcontentsline{toc}{section}{Elektronika tlakové desky}

Tlaková plocha se díky pružné podložce a nažehlovací folii muže ve všech směrech naklánět a díky tomu se při používání mění vzdálenost od čtyř snímacích
cívek. Tlaková plocha je primárně terčík, který slouží jako jádro cívky, která zvětšuje svou indukčnost když se terčík přibližuje a naopak.

\begin{figure}[htbp]
    \centering
    \includegraphics[width=200]{kapitoly/obrazky/E4/elektronika_tlakove_desky/civka_tercik_LDC.png}
    \caption{schematické zobrazení cívky a terčíku}
    \label{fig:E4-sch_civka_tercik}
\end{figure}

Pro snímání indukčnosti používám čip \href{https://www.ti.com/lit/ds/symlink/ldc1612.pdf?ts=1612018658531&ref_url=https%253A%252F%252Fwww.google.com%252F}{LDC1614}
nebo \href{https://www.ti.com/lit/ds/symlink/ldc1312.pdf?ts=1612017390818&ref_url=https%253A%252F%252Fwww.google.com%252F}{LDC1314} 
které se liší prakticky jen rozlišením. LDC1314 disponuje dvanáctibitovím AD převodníkem a LDC1614 dvacetiosmibitovím AD převodníkem, 
a je tak schopen detekovat pohyb terčíku s rozlišením až na 10 nm.

\begin{figure}[htbp]
    \centering
    \includegraphics[width=\textwidth]{kapitoly/obrazky/E4/elektronika_tlakove_desky/moje_zapojeni.png}
    \caption{zapojení čipu LDC1314 na desce trezoru}
    \label{fig:E4-LDC}
\end{figure}

Čip LDC komunikuje po sběrnici I2C která umožňuje komunikaci jednoho mastera, čip který řídí komunikaci, s až 128 slavi, čipy které přijímají příkazy
od mastra a maximálně mu odpovídají. LDC také umožňuje volbu ze dvou I2C adres, aby se dali použít dva tyto čipy na jednom I2C, nebo aby se dala změnit 
případná kolize s jiným čipem který by měl stejnou adresu.

\newpage

Cívky použité na trezoru jsou vyrobeny jako reliéf v vrstvě mědi přímo na DPS. Jejich vzhled jsem navrhoval v simulátoru od Texas Instruments, 
vytvořeného konkrétně pro LDC čipy, a s pomocí popisů dřívějších aplikací které firma Texas Instruments zveřejňuje.

% obrázek ze simulátoru

Výsledná cívka je vytvořena na dvouvrstvé desce a na každé vrstvě má patnáct závitů s drahou o síle 0.152mm se stejně velkou mezerou.

\begin{figure}[htbp]
    \centering
    \includegraphics[width=\textwidth]{kapitoly/obrazky/E4/elektronika_tlakove_desky/civka.png}
    \caption{vzhled reliéfu cívky}
    \label{fig:E4-relief_civka}
\end{figure}

\newpage

Celí trezor obsahuje dvě samostatné elektronické desky přičemž na jedné je osazen jen, kruh z ledek WS2812, a právě snímání tlakové desky které zabírá 
většinu teto desky.

\begin{figure}[htbp]
    \centering
    \includegraphics[width=\textwidth]{kapitoly/obrazky/E4/elektronika_tlakove_desky/leddeska-KiCad.png}
    \caption{vzhled desky s kruhem WS2812 a snímáním tlakové desky}
    \label{fig:E4-LedDeska}
\end{figure}

\newpage


% Zaver prace
\input{CHAPTERS/ZAVER.tex}
\newpage

\appendix
\addcontentsline{toc}{chapter}{Přílohy}

% Prilohy
% \input{CHAPTERS/PRILOHY.tex}

\printbibliography[title=Literatura]
\addcontentsline{toc}{chapter}{Literatura}

\listoffigures
\addcontentsline{toc}{section}{Seznam obrázků}

\listoftables
\addcontentsline{toc}{section}{Seznam tabulek}

\end{document}