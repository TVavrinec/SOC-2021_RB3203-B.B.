\documentclass{template/socthesis}

\usepackage{subcaption} 
\usepackage{amsmath} 
\usepackage{enumitem} 
\usepackage{hyperref} % reference
\usepackage{gensymb} % balíček symbolů
\usepackage{booktabs}

\usepackage[toc,page]{appendix}
\usepackage{color} % balíček pro obarvování textů
\usepackage{xcolor}  % zapne možnost používání barev, mj. pro \definecolor
\definecolor{mygreen}{RGB}{0,150,0} % nastavení barev odkazů 
\usepackage{listings} % balíček pro formátování zdrojových kódů 
\usepackage[author=,status=final]{fixme} % vkládání poznámek  
% dva módy (status): draft (poznámky se zobrazují v PDF) / final (poznámky se nezobrazují v PDF)
\usepackage{multirow}
\usepackage{hyperref} % pro vkládání odkazu

\usepackage{wrapfig}
\usepackage{epsfig}

\lstset { %
    language=C++,
    backgroundcolor=\color{black!5}, % set backgroundcolor
    basicstyle=\footnotesize  % basic font setting
}

\addbibresource{text.bib} % soubor s bibliografií
\nocite{*}

\titlecz{Postav si svého druhého robota} % český název práce
\titleen{Build your second robot} % anglický název práce
\author{Tomáš Vavrinec} % jméno a příjmení autora
\field{10} % obor (pouze číslo, zbytek vysází šablona - číslo oboru viz http://www.soc.cz/obory-soc/)
\school{Střední průmyslová škola a~Vyšší odborná škola Brno, Sokolská, příspěvková organizace} % celý název školy
\mentor{Mgr. Miroslav Burda} % jméno a příjmení školitele
\mentorstatement{Mgr. Miroslava Burdy} % jméno a příjmení ve druhém pádě 

% Změňte, pokud se liší
%\region{Jihomoravský} % kraj
\placefooter{Brno 2021} % místo a rok
% hinty k používání balíčků hyperref, url, hyperlink a hypertarget
% \usepackage{hyperref} % balíček pro hypertextové odkazy
% \url{www.odkaz.cz}
% \href{http://www.odkaz.cz}{Text který bude jako odkaz}
% \hyperlink{label}{proklikávací_text} - odkaz na text 
% \hypertarget{label}{cíl_odkazu} - cíl odkazu 
% konec preambule dokumentu

\begin{document}

\maketitle % vysází titulky

\makecopyrightstatement{V Brně} % místo

% poděkování
\makethanks{Děkuji svému školiteli Mgr. Miroslavu Burdovi za obětavou pomoc, podnětné připomínky a~hlavně nekonečnou trpělivost, kterou mi během práce poskytoval.}

\pagestyle{empty}

\section*{Anotace}
% \color{mygreen}
% Anotace má za úkol stručně popsat cíle práce a velmi stručný úvod k tématu. 
% Většinou bývá použit první odstavec, nebo jiná část úvodu.
\color{black}

Robotika se stává čím dál tím významnějším oborem, což s sebou nese i~potřebu vzdělávání v tomto oboru.
Při výuce robotiky jsou proto potřeba různé pomůcky, na kterých se mohou žáci učit potřebné dovednosti. Jednou z takovýchto pomůcek 
by mohl být například SchoolBoard (viz práce \href{https://github.com/TVavrinec/SOC-text/blob/master/SOČ.pdf}{Postav si svého prvního robota}), 
ale pokročilejším studentům již tento hardware nemusí stačit. Proto jsem začal pracovat na novém systému, který má více možností.

\subsection*{Klíčová slova}

\color{black}

trezor, ESP32, ESP32 Wrover, inteligentní LED, WS2812, BMX055, LDC1614, LDC1314, open-source hardware

\newpage % pokud se anotace vleze na jednu stránku (což by měla), tento rádek zakomentuj

\vspace{20mm}

\section*{Annotation}
\color{black}

Robotics is becoming an increasingly important field, which brings with it the need for education in this field.
When teaching robotics, therefore, various aids are needed on which students can learn the necessary skills. Once with such aids
could be, for example, SchoolBoard (see the work Build Your First Robot), but for more advanced students this hardware may no 
longer need suffice. That's why I started working on a new system that has more options.

\subsection*{Keywords}
\color{black}
safe, ESP32, ESP32 Wrover, smart LED, WS2812, BMX055, LDC1614, \\ LDC1314, open-source hardware

\newpage
\pagestyle{plain}

\tableofcontents % vysází obsah

%%% Začátek práce
\setcounter{figure}{0}
\setcounter{table}{0}
\newpage

% zde můžeš s pomocí příkazu \input{cesta k souboru} vložit soubory; doporučuji každou větší kapitolu dát do samostatného souboru pro větší přehlednost

% Úvod práce

\chapter*{Úvod}
\addcontentsline{toc}{chapter}{Úvod}

Na konci července roku 2019 jsem dostal za úkol navrhnout výrobek pro děti na příměstský tábor pobočky DDM Helceletova Brno, Robotárny. %todo odkaz 
 Poža\-dav\-kem 
byla jednoduchá a levná konstrukce, kterou děti zvládnou sestavit za pár dní a ve zbytku času tábora si stihnou vyzkoušet základy programování 
s využitím tohoto výrobku. Z tohoto důvodu jsem začal vyvíjet elektronicky řízený trezor. Z původní vize trezoru se ale poměrně rychle vyvinulo
univerzální elektronické zařízení, kterému zůstala schopnost sloužit jako trezor. Také se přidala čistě mechanická varianta trezoru pro mladší účastníky táborů a volnočasových aktivit.

Trezor byl pro mě poněkud změnou oproti mé dřívějším práci, která se do té doby vždy točila kolem různých létajících nebo častěji jezdících robotů s velkým důrazem na orientaci v prostoru.
Trezor je oproti těmto vozítkům daleko statičtější, a protože se sám nepohybuje, má jeho vnímání prostoru jiné požadavky. Vozítka také vždy počítala s jistou univerzalitou senzoriky i~mechaniky,
zatímco trezor by měl být upravitelný jen po stránce softwaru. 

Další odlišností trezoru je menší konkurence, která je u různých robotických stavebnic poměrně veliká, jak si můžete přečíst 
v mé dřívější práci \href{https://github.com/TVavrinec/SOC-text/blob/master/SOČ.pdf}{Postav si svého prvního robota}.
%todo Jiné trezory (elektronizované, ve formě stavebnice pro děti/tábory) jsem zatím nikde nenašel. 


%todo popis, proč se oddělila mechanická a ele verze, původní záměr byl mít mecha verzi pro oba trezory stejnou  

 %todo vlastně vznikly trezory dva - odstaveček 


Na konci července roku 2019 jsem dostal za úkol navrhnout výrobek pro děti na příměstský tábor
pobočky DDM Helceletova Brno, Robotárny. 
Poža\-dav\-kem byla jednoduchá a levná konstrukce,
kterou děti zvládnou sestavit za pár dní a ve zbytku času tábora se jim ukážou základy programování
s~využitím tohoto výrobku. Proto, a také pro poněkud nižší věk účastníků, jsme se s vedoucím 
Robotárny, Jiřím Váchou, rozhodli jít cestou \uv{trezoru}. To byl rozdíl oproti našim běžným 
výrobkům, které většinou měly možnost pohybu, ale byly pro děti náročnější na výrobu
a pochopitelně i cena u nich šla nahoru.


% % todo tohle je spíš anotace, úvod musíme zásadně rozšířit
% třeba takhle? snažil jsem se tam procpat i tu minulo sočku.

% Úvod práce má za cíl uvést:
% \begin{itemize}
%     \item cíl práce
%     \item jak ho chcete dosáhnout
%     \item popis tématu práce, musí být výstižný, ale stručný a poutavý
% \end{itemize}

% Úvodu a závěru práce je třeba věnovat obzvláště velkou pozornost.
% Myslete na to, že úvod a někdy i závěr si porotce čte jako první, teprve potom, jestli ho práce zaujme se rozhodne, zda ji přečte celou.

\newpage

\chapter{Vývoj elektronického trezoru}
\label{E-vyvoj}

\section{První verze}
\label{E1-vyvoj}

Dal jsem se tedy do kreslení trezoru. Pochopitelně ne nějaké nedobytné pevnosti, ale malé
krabičky\footnote{128x128mm} na které se dají ukazovat základy programování. 

Jelikož se mi na podobné výrobky osvědčila jako materiál překližka, navrhoval jsem trezor s úmyslem výroby z překližky za využití laseru. 

%Konstrukce byla z velké části přizpůsobená dostupné elektronice, kterou jsem měl k dispozici. 
%a která musela být stejně použita poněkud odlišně, než jak byla zamýšlena. 
%Neměl jsem totiž čas ani rozpočet, navrhovat a především vyrábět konkrétní elektroniku
%pro výrobek, který se měl předložit dětem na letním táboře ani ne za týden. %todo ?? souvisí s termínem tábora? 
Jako základ pro elektroniku  jsem použil univerzální 
desku \href{https://github.com/RoboticsBrno/ArduinoLearningKitStarter}{ALKS}\footnote{Arduino Learning Kit Starter} \parencite{ALKS}, 
kterých jsem měl dostatečnou zásobu. Ovládací prvky\footnote{dvě tlačítka, dva potenciometry a tři barevné LED} jsem umístil na horní stranu trezoru. 
ALKS má v původní variantě tři tlačítka. Já jsem však jedno musel pomocí magnetu a jazýčkového magnetického konektoru použít
jako kontrolu, zda jsou dveře otevřeny či zavřeny. Z toho důvodu měl trezor pro ovládání jenom dvě tlačítka. 
Jako zámek jsem pak použil obyčejné servo SG90, které jednoduše zajelo svou páčkou do drážky ve dveřích, a tím jim zabránilo 
se otevřít. Celý systém pak napájela malá powerbanka, která se dala vyjmout a nabít  
a používala se i ve dvou dalších verzích.

% Tato konstrukce měla kvůli uspěchanému návrhu 
%spoustu problémů. Většinou však šlo o problémy, které by nebylo těžké odstranit a nebylo
%tedy třeba předělávat celý koncept návrhu. 

V těsném závěsu za touto elektronickou variantou jsem ale dostal poža\-da\-vek i~na~čistě mechanickou verzi trezoru. 
To byl následně jeden z důvodů velkých změn, a to i změny samotného konceptu zařízení.
\section{Druhá elektronická varianta}


Druhá verze elektronické varianty testovala použitelnost signalizačního kruhu, o dvanácti ledkách, kolem uprostřed dveří
umístěného enkodéru. Jako základ trezoru jsem použil první mechanickou variantu, ze které jsem odstranil zamykací kola 
a doplnil ji o servo, řídící elektroniku a již zmíněný kruh ledek a enkodér.
Vzhledem k tomu, že se jednalo jen o hrubý prototyp, neměl specializovanou desku~a elektroniku  tedy tvořila jen změť kabelů 
a kousek univerzální desky, takže nemám elektronickou variantu tohoto zapojení. % [schéma jsem kreslil jen na tabuli a to asi rok 
% zpátky, takže když bych došel k závěru že je potřeba, tak se dá udělat, ale je to práce navíc a nepovažoval bych to za podstatné]

Trezor měl pro komunikaci s uživatelem tedy kruh o dvanácti ledkách a~jeden vstupní prvek, enkodér s tlačítkem.
Ovládání tedy bylo od tohoto odvozené a trezor se zmáčknutím zapnul a tlačítko pak dál sloužilo jako potvrzování výběru.
Člověk tak mohl pomocí enkodéru vybírat jedinou rozsvícenou ledku a stiskem potvrdit, vstupní kód tedy mohl vypadat 
například jako čas, a uživatel ho zadal na kruhu odvozeném od ručičkových hodin, proto právě dvanáct ledek.
Konkrétní ovládání je pochopitelně závislé na nahraném programu a mohlo by se tedy jednoduše změnit do libovolné podoby --
to co popisuji je jen konkrétní možnost, kterou jsem použil.

\begin{figure}[htbp]
    \centering
    \includegraphics[width=\textwidth]{kapitoly/obrazky/E2/predni_render.png}
    \caption{render varianty E2}
    \label{fig:E2-render}
\end{figure}

\newpage
\section*{Třetí elektronická varianta}
\addcontentsline{toc}{section}{Třetí elektronická varianta}

Třetí verze elektronické varianty do značné míry vycházela z předchozí, druhé verze, a dále na ní stavěla. Asi nejzjevnější změna bylo navýšení počtu 
ledek z dvanácti, jakožto hodiny, na šedesát jakožto minuty, což pochopitelně znamenalo i zvětšení kruhu. Na desku se ale přidaly i nové funkcionality,
a to gyroskop, pro možnost znalosti náklonu zařízení, akcelerometr, pro znalost směru a velikosti zrychlování, magnetický kompas, pro určení světových
stran, RTC (Real Time Clock, hodiny reálného času), pro znalost přesného času a také GPS pro možnost určení svojí polohy.
Také jsem použil, po vzoru mechanické varianty, rotační západku, což znamenalo, že na stejný trezor se daly použít jak mechanické tak 
elektronické dveře.

\begin{figure}[htbp]
    \centering
    \includegraphics[width=\textwidth]{kapitoly/obrazky/E3.0-render0.pdf}
    \label{fig:M1}
\end{figure}

Tato verze měla dvě podverze, které se lišily motorem.
\begin{figure}[htbp]
    \centering
    \includegraphics[width=300]{kapitoly/obrazky/motory/hodinovyStrojek.jpg }
    \includegraphics[width=300]{kapitoly/obrazky/motory/zluty_motor.jpg}
    \label{fig:M1}
\end{figure}

Přes velké množství funkcí jsem, kvůli několika věcem ale opět koncept přepracoval. Hlavním důvodem změn bylo náročné uložení rotační západky, 
které vyžadovalo ozubený věnec a několik dalších tisknutých dílů.

\newpage
\section{Dnešní elektronická varianta}

Čtvrtá elektronická varianta byla co se elektroniky týče přímým pokračováním předchozí verze. 
Hlavní dvě věci, co se změnily, bylo ovládání a princip zamykání. 

\subsection*{Princip mechanizmu}

Zamykání je založeno na mechanizmu bajonetu a zamčení je zajištěno západ\-kou, která zabraňuje zpětnému otočení.
Západka je ovládána motorem, který otáčí magnetem a přitahuje nebo odpuzuje magnet na západce. Důvodem pro magnetické ovládání
byla možnost západku ovládat i přes pevnou stěnu, a~také pružné spojení, které takto vznikne, takže se trezor například dá zavřít, i~když
je už zamčen (když například dveře nejsou dovřeny).

\begin{figure}[htbp]
    \centering
    \includegraphics[width=170pt]{kapitoly/obrazky/E4/predni_render.png}
    \includegraphics[width=170pt]{kapitoly/obrazky/E4/zadni_render.png}
    \caption{rendery trezoru E4}
    \label{fig:E4-render}
\end{figure}

\subsection*{Shrnutí změn oproti minulé verzi}

Trezor získal možnost komunikace pomocí IR, pro možnost identifikace růz\-ných dveří, dále získal magnetický enkodér, pro možnost snazšího ovládání
motoru zámku. 
Další inovací byl programovací systém s USB-C, na místo USB-micro jako dřív. Tento programátor má možnost úplně si odpojit napáje\-ní, a to v rámci šetření 
energie, když ho trezor nevyužívá, a zároveň možnost zákazu přeprogramování.
Podstatnou změnou také bylo rozdělení elektroniky do dvou různých desek, protože na jedné by nebyl dostatek místa. Jedna deska tak obsahuje ledkový 
kruh a čip LDC1614 nebo LDC1314 se čtyřmi cívkami, které měří vzdálenost tlakové desky. Na druhé desce pak bylo vše ostatní, tedy procesor, akcelerometr,
gyroskop, magnetický kompas, RTC (Real Time Clock, hodiny reálného času), barometr, IR vysílač a přijímač, magnetický enkodér, programátor, řešení 
napájení, řízení motoru a nabíječka.

\subsection*{Ovládání}
Předchozí varianty měly jako hlavní ovládací prvek enkodér s tlačítkem, ten jsem v nynější variantě odstranil, aby přední stěna neměla tak velký 
výstupek. Proto jsem tento prvek nahradil indukční tlakovou deskou, která vyplnila vnitřek kruhu ledek. 
Zbytek ovládání víceméně přetrval, jen kvůli nedostatku času a pandemií způsobenému nedostatku součástek, trezor přišel o GPS. Na druhou stranu 
získal barometr s rozlišením schopným detekovat změnu výšky o půl metru.

\subsection*{Napájení}
Předchozí verzím sloužila jako napájení powerbanka. Ta však kladla poměrné velké omezení, dokázala poskytnout proud pouze jedné ampéry, a proto 
jsem jí nahradil vlastním zdrojem, dvěma bateriemi 18650. To však samozřejmě znamenalo nutnost vlastního řešení stabilizace napětí, díky čemuž 
trezor dostal stepup, který spíná napětí z 3,5~V až 4,2~V na 5~V, a původně stepdown, později lineární stabilizátor, který poskytoval 3,3 a 5~V. %todo zkotroluj prosím ta napětí
Trezor také dostal vlastní nabíječku, aby pro nabíjení baterií stačilo připojit kabel, stejně jako třeba u mobilu.

\newpage %todo sjednotit s aktuální verzí? 


\chapter{Vývoj mechanického trezoru}
\label{M-vyvoj}

\section*{první mechanická varianta}
\addcontentsline{toc}{section}{první mechanická varianta}
První, čistě mechanická varianta, vznikla začátkem srpna 2019, chvíli po výše obšírněji popsané elektronické variantě.
Měla stále poměrně klasický vzhled trezoru, tedy zamykatelná skříňka, která obsahovala dvě kola, která ovládala možnost pohybu jednoduché západky.
Na rozdíl od jeho elektronického předchůdce bylo vše zajímavé uvnitř dveří. Také byla určená jako základ pro případný upgrade na elektronickou
variantu. Na podobné vylepšení mělo stačit odstranění kódovacích kol a přidělání elektronické části. Toto sice fungovalo obstojně, zároveň 
i jako motivace, ale kvůli pozdější změně konceptu mechanizmu tento nápad padl.
Tato varianta však nebyla, kvůli přílišným nárokům na přesnost, vhodná pro stavbu s malými dětmi, pro které byla určena jakožto předstupeň 
k variantě elektronické (která vyžaduje i znalosti, nebo alespoň ochotu k učení, programování).

\begin{figure}[htbp]
    \centering
    \includegraphics[width=400]{kapitoly/obrazky/M1-mechanizmus.png}
    \caption{zelená značí kódová kola, červená západku, modrá pevnou část trezoru(otvor) a žluté díly tvoří distanci}
    \label{fig:M1}
\end{figure}
\newpage
\section*{druhá mechanická varianta}
\addcontentsline{toc}{section}{druhá mechanická varianta}

Druhá mechanická varianta je až na drobnosti stejná jako verze dnešní.
Ovládá se pěti koly, z nichž čtyři zajišťují heslo a páté otáčí s rotační západkou, které drží dveře na svém místě.
Tato varianta tedy přichází z možností dveře úplně oddělit od skříně trezoru. To by při využití jako trezor, který
má za úkol jen ochraňovat svůj obsah, sice nepřinášelo žádný velký užitek, ale při mém využití, spíše jako herní 
prvek než trezor, to může být užitečné.

\begin{figure}[htbp]
    \centering
    \includegraphics[width=150]{kapitoly/obrazky/M2-mechanizmus.png}
    \includegraphics[width=150]{kapitoly/obrazky/M2-mechanizmus_zamceno.png}
    \label{fig:M1}
\end{figure}

\begin{figure}[htbp]
    \centering
    \includegraphics[width=\textwidth]{kapitoly/obrazky/M2-render.PNG}
    %\caption{}
    \label{fig:M1.0}
\end{figure}


\newpage
\section{Dnešní mechanická varianta}

Dnešní mechanická varianta je téměř stejná jako druhá verze rozdíl je jen v uložení kol, které kolem hřídelů získalo distanční kroužky, které
zjednodušují lepení. 

\begin{figure}[htbp]
    \centering
    \includegraphics[width=70pt]{kapitoly/obrazky/M3/predni_render.png}
    \caption{render varianty M3}
    \label{fig:M3-render}
\end{figure}

\begin{figure}[htbp]
    \centering
    \includegraphics[width=220pt]{kapitoly/obrazky/M3/rez.png}
    \caption{Řez kódovacím kolem}
    \label{fig:M3-rez-kolem}
\end{figure}

\newpage


\chapter{Mechanický trezor} 
\label{M3}

\section{Úvodní shrnutí}

Dnešní mechanická varianta se zamyká pomocí rotační západky a~čtyř kódovacích kol, která blokují západku v zamčeném stavu. \newline
Princip mechanizmu viz obrázek \ref{fig:M2-mechanizmus}.

Vedle elektronické varianty jsem navrhl variantu mechanickou, abych měl jednodušší a~levnější trezor pro mladší účastníky táborů a~jiných akcí. 

\newpage
\section{Popis jednotlivých součástek a důvody konkrétního tvaru}

Trezor má tvar krychle a~délku hrany má 128~mm, násobek šestnácti jsem zvolil kvůli jednoduché návaznosti na dřívka, %todo přidáme pár vět o dřívkách 
dřevěná dřívka s obdélníkovým průřezem 3x16~mm nebo 2x16~mm.
Protože je trezor vyroben z překližky o síle 4~mm, jsou jeho vnitřní rozměry o 4~mm na každé straně menší (takže 122~mm).

\paragraph{Geometrie západky}
\begin{wrapfigure}{R}[0.2\textwidth]{0.7\textwidth}
    \includegraphics[width=0.7\textwidth]{kapitoly/obrazky/M3/geometrie_zapadky.png}
    \caption{náčrt západky} %todo čeho náčrt? 
    \label{fig:M3-geometrie-zapadky}
\end{wrapfigure} %todo zvážil bych obrázek centrovaný a obtékaný pouze nahoře a dole 

Protože se západka otáčí musí jí být zajištěn dostatek prostoru, zároveň však otvor pro dveře je lepší mít větší, protože se potom trezor dá použít pro větší objekty.
Z tohoto důvod jsou hrany západky definovány kružnicí o~průměru, délky vnitřní hrany trezoru. Západka má v~rozích sražení ze dvou důvodů. Za prvé aby byl otvor pro
dveře větší a~za~druhé aby namáhání působící v~západce působilo na větší délce.

\newpage

\paragraph{Distanční deska}

\begin{wrapfigure}{R}[0.2\textwidth]{0.7\textwidth}
    \includegraphics[width=0.7\textwidth]{kapitoly/obrazky/M3/distancka.png}
    \caption{náčrt Distanční desky}
    \label{fig:M3-distancka}
\end{wrapfigure}

Abi se západka dostala za desku přední stěny bedny trezoru je potřeba jí od přední stěny dveří posunout právě o tloušťku stěny. To zajišťuje jednoduchá čtvercová deska jen s~pěti otvory
pro průchod ovládacích kol.

\paragraph{Kámen} % jaksi mě nenapadl lepší název 
\begin{wrapfigure}{L}[0.2\textwidth]{0.7\textwidth}
    \includegraphics[width=0.7\textwidth]{kapitoly/obrazky/M3/kamen.png}
    \caption{náčrt kamene}
    \label{fig:M3-kamen}
\end{wrapfigure}

Kamen, který zajišťuje kód, má z~části tvar drážky, ve které jezdí, a~z~části kruh který se~muže otáčet v~kruhovém otvoru, na jedné staně drážky.
Uprostřed má~kruhoví otvor o~průměru 8mm pro kolík který kamenem otáčí.

\paragraph{Lepící distanční kroužek}

\begin{wrapfigure}{R}[0.2\textwidth]{0.7\textwidth}
    \includegraphics[width=0.7\textwidth]{kapitoly/obrazky/M3/lepici_distance.png}
    \caption{náčrt lepícího distančního kroužku}
    \label{fig:M3-lepici-distance}
\end{wrapfigure}

Tyto distanční kroužky jsou zde čistě z~technologického důvodu. Při lepení kolíku, totiž měli děti problém s~lepidlem, které jim zatékalo do~prostoru mezi kolíkem 
a~stěnou dveří čímž znemožňovalo otáčení kol. Proto jsem přidal tyto kroužky, do~kterých když zateče lepidlo tak~se nic neděje.

\newpage
\section{Odolnosti proti násilnému vniknutí}

\paragraph{Vytržení dveří}
Jedním ze způsobů~namáhání~mechanizmu~je~vytržení~dveří~z~trezoru.

\subsection{Západka}

\begin{figure}[htbp]
    \centering
    \includegraphics[width=\textwidth]{kapitoly/obrazky/M3/simulace/odolnost_proti_vytrzeni_4kN.png}
    \caption{simulace pokusu o vytržení dveří silou 4 000 N}
    \label{fig:M3-simulace-vytrzeni}
\end{figure}
Ke kompletní simulaci se můžete dostat \href{https://myhub.autodesk360.com/ue2d7aa41/g/shares/SH56a43QTfd62c1cd96843f1e03a0eb48053?viewState=NoIgbgDAdAjCA0IDeAdEAXAngBwKZoC40BlASwFsBXAGwEN1SB7AOzXjVoGdPd1C0ARjABsATlEQItALQBjcbmkAWCMIjSBuWgA5lAM22ilAVgAmMAOyy9%2BBGkYCAVrlnoAkqcIBmAL4gAukA}{zde}
po kliknutí na "Simulation" a "Simulation Model 2". V tabulce napravo se pak můžete přepínat mezi barevným zobrazení několika veličin.

\newpage

\subsection{Kolík}
Při pokusu o vytržení je celá síla přenášena kolíkem.

\(\sigma _M_A_X = 132 MPa\)    ( \href{https://is.mendelu.cz/eknihovna/opory/zobraz_cast.pl?fit_w=1;cast=9190}{dubové dřevo ve směru vláken při vlhkosti 12 \% }) % strana 22 tabulka 2 -> https://www.vutbr.cz/www_base/zav_prace_soubor_verejne.php?file_id=66237

D = 6

\(\sigma _M_A_X = F/S \Rightarrow F = \sigma _M_A_X * S = 132 * (\pi * D^2/4) = 3 732.21 N \) z toho a ze simulace vyplývá že kolík je při namáhání nejslabším členem, přesto že ne o moc.

\paragraph{Otevření bez odemčení}
Dalším způsobem namáhání může být snaha otočit západkou pez zadání správného hesla.

\subparagraph{Západka a kamen}

\begin{figure}[htbp]
    \centering
    \includegraphics[width=\textwidth]{kapitoly/obrazky/M3/simulace/odolnost_proti_nasilnemu_odemceni_10Nm.png}
    \caption{simulace pokusu o otevření bez předchozího odemčení při kroutícím momentu 10 000 Nmm, zobrazeno jen napětí nad 1 MPa}
    \label{fig:M3-simulace-vytrzeni}
\end{figure}

Ke kompletní simulaci se můžete dostat \href{https://myhub.autodesk360.com/ue2d7aa41/g/shares/SH56a43QTfd62c1cd96843f1e03a0eb48053?viewState=NoIgbgDAdAjCA0IDeAdEAXAngBwKZoC40BlASwFsBXAGwEN1SB7AOzXjVoGdPd1C0ARjABsATlEQItALQBjcbmkAWCMIjSBuWgA5lAM22ilAVgAmMAOyy9%2BBGkYCAVrlnoAkqcIBmAL4gAukA}{zde}
po kliknutí na "Simulation" a "Simulation Model 3". V tabulce napravo se pak můžete přepínat mezi barevným zobrazení několika veličin.

\subparagraph{Kolík}
Kroutící moment který je dřevěný kolík o průměru 6mm schopen přenést. 

\(\tau _M_A_X = 52.3 MPa\)    (\href{https://is.mendelu.cz/eknihovna/opory/zobraz_cast.pl?fit_w=1;cast=9190}{dubové dřevo ve směru vláken při vlhkosti 12 \% })

D = 6

\tau _M_A_X = \frac{M_K}{W_K} \Rightarrow M_K = \tau _MAX * W_K = \sigma _D * \frac{\pi * D^3}{16} 

\(M_K = 52.3 * \frac{\pi * 6^3}{16} = 2 218.16 N*mm \Rightarrow\) a ze simulace že kolík je při namáhání v krutu nejslabším místem \(\Rightarrow\) pro zvýšení odolnosti by bylo 
potřeba zvětšit kolík nebo změnit materiál.
\section{Použití trezoru}
Mechanické verze trezoru M1 a M2 měli na akcích a v kroužcích dětmi možnost trezor jen stavět. První akce založená na trezoru, která neobsahovala jen stavbu, využívala už variantu M3.
Protože na akci byli menší děti místo klasické číselné stupnice byl trezor vybaven obrázkovým kódem jak je vidět na obrázku \ref{fig:M3-trpaslici}.

\begin{figure}[htbp]
    \centering
    \includegraphics[width=\textwidth]{kapitoly/obrazky/M3/trpaslici.png}
    \caption{Render varianty M3}
    \label{fig:M3-trpaslici}
\end{figure}


\newpage
\chapter{Elektronický trezor}
\label{E4}

\section{Přehled}

Dnešní verze elektronického trezoru se zamyká pomocí mechanizmu bajonetu a~magneticky řízené zpětné západky. 

Elektronika je vybavena čipem ESP32 \parencite{ESP32}, \parencite{ESP32-WROVER-B},
který obsahuje dva procesory Xtensa LX6, WiFi a bluetooth. Dále je trezor vybaven čipem BMX055 \parencite{bmx055} nebo dvojicí čipů MPU6050 \parencite{mpu6050} 
a QMC5883 \parencite{qmc5883}, které poskytují 
gyroskop, akcelerometr a magnetický kompas. Dále je zde SPL06 \parencite{spl06}, barometr s rozlišením 0,06~hPa, což umožňuje rozeznat změnu nadmořské výšky 
o polovinu metru. Další systém trezoru je možnost IR komunikace, která je zde pro možnost jednoznačné identifikace dveří, ale pochopitelně může 
sloužit i pro jiný účel. Deska je také vybavena RTC a má vlastní programátor pro usnadnění programování. Vedle ESP32 je zde asi 
nejvýznamnějším čipem LDC1614 \parencite{LDC1614}, případně LDC1314, který umožňuje funkci tlakové plochy (viz kapitoly \ref{E4-mech_tlakovky}, \ref{E4-tlakovka}).

%todo doplnit popis a využití a možnosti tlakové desky 

\begin{table}[h]
    \centering
    \resizebox{\textwidth}{!}{%
    \begin{tabular}{@{}lll@{}} 
    \textbf{čip} & \textbf{popis} & \textbf{poznámky}
                                                                                    \\ \midrule
    \textbf{ESP32}              & dva procesory Xtensa LX6, WiFi a bluetooth    &                                                   \\
    \textbf{BMX055}             & gyroskop, akcelerometr, magnetický kompas     & možno nahradit dvojicí čipů MPU6050 a QMC5883     \\
    \textbf{SPL06}              & barometr                                      & rozlišení až 0,06hPa                              \\
    \textbf{IRM-H936 a IR led}  & IR komunikace                                 &                                                   \\
    \textbf{LDC1614}            & snímání tlakové desky                         & počítá se s možnou záměnou za LDC1314             \\
    \textbf{CP2102}             & programátor                                   & s hardwarově zajištěným odpojováním napájení      \\ \bottomrule
    \end{tabular}%
    }
    \caption{Shrnutí elektronického vybavení}
    \label{tab:shrnuti}
\end{table}

\newpage
\section{Mechanika tlakové desky}

Indukčně snímaná tlaková deska funguje díky čtyřem cívkám na desce plošných spojů, které mění svojí indukčnost podle vzdálenosti snímané desky, terčíku.
Z tohoto důvodu se terčík při používání naklání, čímž zároveň mění svojí vzdálenost od jednotlivých cívek. Z toho také plyne nutnost uložit terčík
částečně volně. Terčík je proto od snímací desky oddělen pružnou vložkou, která je zároveň předepnuta pomocí nažehlovací fólie, která kryje přední 
stranu dveří a spojuje terčík s čelní krycí deskou. Díky nažehlovací fólii je také přední část dveří voděodolná.

\begin{figure}[htbp]
    \centering
    \includegraphics[width=\textwidth]{kapitoly/obrazky/E4/machanika_tlakove_desky/rez_po_ose.pdf}
    \caption{Řez varianty E4}
    \label{fig:E4-rez}
\end{figure}

Tlaková deska zárově počítá s možností působení síly o velikosti až 500~N, což samozřejmě zároveň znamená, že tělo dveří tomuto zatížení musí odolat.
Vzhledem k tomu, že nemám možnost vyrobit tělo z kovu a jsem odkázán na 3D tisk a laserovou řezačku, a zároveň chci mít dveře co možná nejmenší,
musel jsem napočítat kritické části napřesno. Z tohoto důvodu jsem v programu Fusion 360, ve kterém jsem trezor vyvíjel,
dělal simulaci, kterou zde přikládám. %todo obrázky se simulací do přílohy, a zde odkaz na č. obr a stránku, podobně i další velké obrázky (nad cca třetinu strany) 

Jako materiál těla jsem v první fázi zvolil standardní fotopolymer pro tiskárny typu SLA, s pevností v tahu 46 až 67 MPa.
V budoucnu bych ale chtěl tělo odlévat z nějakého houževnatého polyuretanu, aby se zlevnila výroba a zároveň stoupla odolnost.

\begin{figure}[htbp]
    \centering
    \includegraphics[width=370pt]{kapitoly/obrazky/E4/machanika_tlakove_desky/simulace/F100N,primo,uprostred,pohled_zepredu.png}
    \caption{Pevnostní simulace těla}
    \includegraphics[width=370pt]{kapitoly/obrazky/E4/machanika_tlakove_desky/simulace/F100N,primo,uprostred,pohled_zezadu.png}
    \caption{Pevnostní simulace těla pohled zezadu}
    Tato simulace testuje působení síly přímo na tělo, což není působení, které by v provozu nastávalo. Takovéto namáhání je ale o dost náročnější
    než to, které by reálně nastalo.
    \label{fig:E4-simulace_tela} %todo těla čeho? trezor je hranatý? 
\end{figure}

\begin{figure}[htbp]
    \centering
    \includegraphics[width=\textwidth]{kapitoly/obrazky/E4/machanika_tlakove_desky/simulace/zjednodusena_sestava_pri_F100N_nezobrazeno_napeti_pod_1,5MPa.png}
    \caption{Simulace sestavy}
    Jak je vidět, tak i sílu 100~N dokaže sendvič z terčíku, pružné podložky a~snímací desky rozložit na dostatečnou plochu, aby napětí v těle nestouplo 
    nad cca 3~MPa. Na obrázku je zobrazené jen napětí nad 1,5~MPa.
    \label{fig:E4-simulace_tlakovky}
\end{figure}

\clearpage
\newpage
\subsection*{dnešní elektronická varianta}
\addcontentsline{toc}{subsection}{dnešní elektronická varianta}

%    Západka
%        a) vývoj západky
%            1) původní verze (pružnost materiálu -> po čase ztrácí pružnost)
%            2) verze s pružinou (pružina je nepotřebná a jsou s ní zbytečné problémy)
%            3) b,c,d,e
%        b) obrázek
%        c) uhel čela (tak aby měla dosedací plocha co nejmenší vůli a zároveň opravdu dosedala v celé ploše [a ne v úseče nebo dokonce jednom bodě])
%            1) obrázek
%        d) pohyby
%            1) pohyb tam
%            2) pohyb spět (stačí magnet pružina je nepotřebná)
%            3) motor
%            4) enkoder
%            5) možná video?
%        e) matika
%            1) odolnost proti zpětnému otočení
%            2) obrázek ze simulace
%            3) možná video?
%            4) další návrhy?


\newpage

\begin{figure}[htbp]
    \section*{Ukosy}
    \addcontentsline{toc}{section}{Ukosy}    
    \centering
    \includegraphics[width=400]{kapitoly/obrazky/E4/ukozy/ukladaci_ukosy.pdf}
    \caption{ukládací úkosy}
    Aby bylo jednoduší při zavírání dveře správně natočit, mají zarážky na vnitřní straně velké úkosy, které tak zvětšují na vnitřní straně 
    vůli a při zasouvání navedou dveře do správné pozice.
    \label{fig:E4-ukosy}
\end{figure}

\begin{figure}[htbp]
    \centering
    \includegraphics[width=400]{kapitoly/obrazky/E4/ukozy/simetrie_zarazek.png}
    \caption{symetrie zarážky}
    Zarážky na obvodu otvoru mají obě kontaktní plochy stejné. Sice by mohlo být výhodné přizpůsobit tvar strany, kolem které, se pohybuje západka, 
    pohybu západky. Západka by tak mohla mýt vedení v průběhu celého pohybu. Pro symetrii jsem se však rozhodl kvůli možnosti díl s otvorem otočit.
    To je výhodné při stavbě s dětmi, kvůli zmenšení počtu chyb kterých se děti můžou při stavbě dopustit, a stráta vedení není tak zásadní.
    \label{fig:E4-simetrie_zarazky}
\end{figure}

\clearpage
\newpage
\section*{Elektronika tlakové desky}
\addcontentsline{toc}{section}{Elektronika tlakové desky}

Tlaková plocha se díky pružné podložce a nažehlovací folii muže ve všech směrech naklánět a díky tomu se při používání mění vzdálenost od čtyř snímacích
cívek. Tlaková plocha je primárně terčík, který slouží jako jádro cívky, která zvětšuje svou indukčnost když se terčík přibližuje a naopak.

\begin{figure}[htbp]
    \centering
    \includegraphics[width=200]{kapitoly/obrazky/E4/elektronika_tlakove_desky/civka_tercik_LDC.png}
    \caption{schematické zobrazení cívky a terčíku}
    \label{fig:E4-sch_civka_tercik}
\end{figure}

Pro snímání indukčnosti používám čip \href{https://www.ti.com/lit/ds/symlink/ldc1612.pdf?ts=1612018658531&ref_url=https%253A%252F%252Fwww.google.com%252F}{LDC1614}
nebo \href{https://www.ti.com/lit/ds/symlink/ldc1312.pdf?ts=1612017390818&ref_url=https%253A%252F%252Fwww.google.com%252F}{LDC1314} 
které se liší prakticky jen rozlišením. LDC1314 disponuje dvanáctibitovím AD převodníkem a LDC1614 dvacetiosmibitovím AD převodníkem, 
a je tak schopen detekovat pohyb terčíku s rozlišením až na 10 nm.

\begin{figure}[htbp]
    \centering
    \includegraphics[width=\textwidth]{kapitoly/obrazky/E4/elektronika_tlakove_desky/moje_zapojeni.png}
    \caption{zapojení čipu LDC1314 na desce trezoru}
    \label{fig:E4-LDC}
\end{figure}

Čip LDC komunikuje po sběrnici I2C která umožňuje komunikaci jednoho mastera, čip který řídí komunikaci, s až 128 slavi, čipy které přijímají příkazy
od mastra a maximálně mu odpovídají. LDC také umožňuje volbu ze dvou I2C adres, aby se dali použít dva tyto čipy na jednom I2C, nebo aby se dala změnit 
případná kolize s jiným čipem který by měl stejnou adresu.

\newpage

Cívky použité na trezoru jsou vyrobeny jako reliéf v vrstvě mědi přímo na DPS. Jejich vzhled jsem navrhoval v simulátoru od Texas Instruments, 
vytvořeného konkrétně pro LDC čipy, a s pomocí popisů dřívějších aplikací které firma Texas Instruments zveřejňuje.

% obrázek ze simulátoru

Výsledná cívka je vytvořena na dvouvrstvé desce a na každé vrstvě má patnáct závitů s drahou o síle 0.152mm se stejně velkou mezerou.

\begin{figure}[htbp]
    \centering
    \includegraphics[width=\textwidth]{kapitoly/obrazky/E4/elektronika_tlakove_desky/civka.png}
    \caption{vzhled reliéfu cívky}
    \label{fig:E4-relief_civka}
\end{figure}

\newpage

Celí trezor obsahuje dvě samostatné elektronické desky přičemž na jedné je osazen jen, kruh z ledek WS2812, a právě snímání tlakové desky které zabírá 
většinu teto desky.

\begin{figure}[htbp]
    \centering
    \includegraphics[width=\textwidth]{kapitoly/obrazky/E4/elektronika_tlakove_desky/leddeska-KiCad.png}
    \caption{vzhled desky s kruhem WS2812 a snímáním tlakové desky}
    \label{fig:E4-LedDeska}
\end{figure}

\newpage
\section{LED kruh}
\label{WS2812}

Trezoru vévodí světelný kruh. Slouží jako displej, na kterém trezor může zobrazovat vše, co potřebuje. Kruh obsahuje šedesát jednotlivých ledek 
\href{https://cdn-shop.adafruit.com/datasheets/WS2812B.pdf}{WS2812} \parencite{WS2812}, konkrétně WS2812 mini. Variantu mini jsem zvolil, aby kruh mohl mít menší
průměr, který takto vychází na 80~mm. WS2812 mají totiž rozměr pouzdra 3,5x3,5~mm, zatím co ostatní varianty mají rozměry 5x5~mm,\footnote{V průběhu vývoje se na trhu 
objevily i WS2812 v pouzdře 2020, které mají rozměry 2x2~mm, ty však nebyly v nabídce JLCPCB a navíc byla deska již prakticky hotová.} což by znamenalo průměr kruhu alespoň 120~mm.

WS2812 nejsou jen LED, ale mají v sobě logiku, díky které je možné jich řetězit velké množství s pomocí jediného pinu, takže na řízení celého kruhu stačí jen jeden pin na ESP32.
Ukázka DPS je na obrázku \obr{fig:E4-LedDeska} a~zapojení desky je na obrázku \obr{fig:E4-sch_WS2812}.
\section*{Napájení}
\addcontentsline{toc}{section}{Napájení}

Jako napájení celého trezoru slouží dvě li-on baterie 18650. Napětí článků však nevyhovuje potřebám trezoru a~tak je na trezoru lineární 
stabilizátor NCP708, který zajišťuje napětí 3,3V pro většinu systému. Kromě NPC708 je zde také step-up FP6276 který zajišťuje napájení 5V 
sloužící primárně ledkám WS2812 a~v~druhá řdě motoru zámku. 

\begin{figure}[htbp]
    \centering
    \includegraphics[width=\textwidth]{kapitoly/obrazky/E4/napajeni/zdroj.pdf}
    \caption{zapojení zdroje}
    \label{fig:zdroj}
\end{figure}

\newpage

\paragraph*{Zapínání}
\addcontentsline{toc}{paragraph}{Zapínání}

\begin{wrapfigure}{R}{0.5\textwidth}
    \centering
    \includegraphics[width=0.55\textwidth]{kapitoly/obrazky/E4/napajeni/ochrana_proti_prepolovani_a_zapinani.png}
    \caption{\label{fig:frog1}Ochrana proti přepólovaní a zapínání}
\end{wrapfigure}
Aby se trezor mohl vypnout a~tak šetřit energii je vybaven obvodem který to zařizuje.

Při připojení článků se napětí dostane nejprve na polyfuse, které slouží jako ochrana proti nadproudu, například v případě kdy uživatel připojí dva 
různě nabyté články nebo jeden z~nich přepóluje.
Pokud se napětí dostane skrz polyfuse, dostane se na tranzistor Q11, skrz který projde jen pokud jsou články správně otočeny.
Když se napětí dostane přes ochranu proti přepólování dostane se na S tranzistoru Q5, skrz R6 na D Q1 a~pak skrz R7 na obě strany C61.
Pokud v~takovéto situaci dojde ke~stisku SW3, projde zem skrz C61 na G Q5. V~tu~chvíli se Q5 otevře na dostatečně dlouhou dobu, 
aby naběhla třívoltová větev a~skrz pull-up (na obrázku je jen poznámka ne reálná součástka) se zvedla napětí na G Q1 na téměř 3,3V. 
Q1 se tak otevře a~už~trvale připojí GND na G Q5, trezor se tak zapnul. Pokud v takové chvíli procesor stáhne dráhu SHUTDOWN 3V3-5V 
na GND, nebo dojde ke stisku SW5, opět se uzavře Q1 a~skrz R6 projde na G Q5 napětí které Q5 uzavře a~tak elektroniku opět vypne.

\newpage

\paragraph*{Stabilizátor}
\addcontentsline{toc}{paragraph}{Stabilizátor}

Stabilizátor \href{https://datasheet.lcsc.com/szlcsc/ON-Semicon-ON-NCP708MU330TAG_C183178.pdf}{NCP708} má pin EN který slouží k~jeho vypínání, 
pokud je na nem logická 0 je~stabilizátor vypnut a~pokud 1 je zapnut. Vzhledem k~tomu že v~mém zapojení toto vypínaní nepotřebuji je pin EN připojen 
přes R10 přímo na napájecí napětí a~tak je stabilizátor trvale zapnut. Druhý pin kterým se~NCP708 liší od~jiných stabilizátorů je~pin~SNS, 
který je vrchní stranou děliče, který určuje výstupní napětí. Vzhledem k~tomu že je dělič nastaven právě tak jak potřebuji nemusím ho~nijak 
upravovat a~tak~je~SNS napojen přímo na~výstup stabilizátoru. % \newline
Konkrétně NPC708 jsem vybral kvůli malému pádu napětí, který vyžaduje pro svůj provoz, typicky 0,25V při proudu 1A. Vzhledem k~tomu že~na~vstupu 
mám maximálně 4,2V tak maximální napěťoví pád který mám k~dispozici je~0,9V, protože na výstupu požaduji napětí 3.3V. Navíc musím počítat 
i~s~vybitou baterii, u~které počítám s~napětím 3.5V. Rád bych počítal s~napětím ještě nižším ale v~nabídce JLCPCB jsem nenašel stabilizátor 
s~nižším pádem napětí a~zároveň dostatečným proudem. 

\begin{figure}[htbp]
    \centering
    \includegraphics[width=400]{kapitoly/obrazky/E4/napajeni/stabilizator.png}
    \caption{Zapojení stabilizátoru}
    \label{fig:E4-stabilizator}
\end{figure}

\paragraph*{Step-up}
\addcontentsline{toc}{paragraph}{Step-up}

\subparagraph*{Step-up vysvětlení funkce}
Zapojení step-upu je~o~něco složitější nez stabilizátor, který stačí připojit a funguje. Spínané zdroje využívají ke své funkci cívku, na které
vzniká změna napětí v případě step-up je to jeho rust. Proud cívkou se nedá okamžitě zastavit a právě to se využívá. 
Když se step-up spustí, připne výstup cívky k~zemi. Ve chvíli kdy je proud cívkou dostatečný přepne se výstup cívky na výstup step-upu.
Protože proud cívkou se nedá zastavit a~cívkou už proud teče, zvedne se napětí za cívkou, které začne plnit kondenzátor na výstupu.
Když napětí stoupne nad horní hranici požadovaného napětí, výstup cívky se opět přepne na zem. Ve chvíli kdy napětí na kondenzátorech opět 
klesne, tím že dodává proud, připojí se cívka. Protože se~na~dobu pádu napětí na~výstupním kondenzátoru, připojila cívka na~zem, obnovil 
se~v~ní~proud a~cyklus se tak muže opakovat.


\subparagraph*{Step-up zapojení na desce trezoru}
Pro ovládání spínání step-upu jsem zvolil \href{https://datasheet.lcsc.com/szlcsc/Feeling-Tech-FP6276AXR-G1_C83308.pdf}{FP6276}.
Tento obvod jsem zvolil protože mi vyhovoval, jak po stráně napětí tak po~staně efektivity a~ceny a~zároveň byl v~nabídce firmy JLCPCB,
u~které jsem desky vyráběl a~osazoval. 
Obvod jsem z~většiny zapojil dle doporučení výrobce, mojí prací bylo vlastně jen správně určit hodnotu 
jednotlivých součástek. Na ovládání pinu EN, který FP6276 vypíná, jsem připojil pull-up k napájení a~pro možnost step-up vypnout tranzistor Q2. 
Pokud procesor stáhne dráhu SHUTDOWN 5V k zemi a~tak přivede na G Q2 zem, Q2 se zavře a~tím se na pin EN přivede skrz R18 napájecí napětí, 
které step-up spustí. Pokud se na G Q2 přivede naopak logická jedna, Q2 se~otevře a~na~EN~se~dostane zem, která naopak provoz step-upu zastaví.

\begin{figure}[htbp]
    \centering
    \includegraphics[width=400]{kapitoly/obrazky/E4/napajeni/step-up.png}
    \caption{zapojení step-upu}
    \label{fig:E4-step-up}
\end{figure}

\newpage

\paragraph*{měření napětí baterek}
\addcontentsline{toc}{paragraph}{měření napětí baterek}

\begin{wrapfigure}{R}{0.6\textwidth}
    \centering
    \includegraphics[width=0.6\textwidth]{kapitoly/obrazky/E4/napajeni/delic_baterimetru.png}
    \caption{\label{fig:frog1}}
\end{wrapfigure}

Aby trezor mohl zjísti že má vybité baterie musí mít možnost jim měřit napětí. ESP32 obsahuje AD převodník takže není problém měřit napětí baterii 
i poměrně přesně, kde však problem nastává je maximální napětí které je schopen měřit~a~to~1.1V. ESP32 sice má možnost připojit k~AD převodníku dělič,
aby se~na~pin dalo přivést napětí až 3.3V ale to~pořád není dostatečné a~také se~tím snižuje přesnost měření. Proto je na desce jednoduchý dělič napětí
složený ze~dvou odporů, jednoho s~hodnotou 1Mohm a~druhého 300kohmu takže při plně nabytých baterii bude na výstupu děliče 0.97V.

\newpage
\section*{Nabíjeni}
\addcontentsline{toc}{section}{Nabíjeni}

Aby se dveře trezoru nemuseli pokaždé rozebírat kvůli nabíjení, je deska vybavena lineární nabíječkou 
\href{https://datasheet.lcsc.com/szlcsc/Seaward-Elec-SE9017-HF_C115752.pdf}{SE9017}. 
Tuto nabíjecí obvod jsem zvolil z~nabídky JLCPCB kvůli volitelnému nabíjecímu proudu, který jsem pomocí R48 stanovil na 700 mA, a~také, kvůli malému 
pouzdru a~nízké ceně.
Pro signalizaci zda je~baterie dobita nebo zda se ještě dobijí jsou zde dvě LED~diodi LED4 a~LED5. Když se baterie dobíjí tak svítí LED4, která svítí 
červená, a~když je~baterie dobita svítí LED5, která svítí modře.

\begin{figure}[htbp]
    \centering
    \includegraphics[width=\textwidth]{kapitoly/obrazky/E4/nabijeni/nabijecka.png}
    \caption{zapojeni nabíječky}
    \label{fig:E4-step-up}
\end{figure}

\newpage
% nenapadá mě co víc říct?
\section{ESP32 a jeho programátor}

%\begin{wrapfigure}{L}[10mm]{0.4\textwidth}
%    \centering
%    \includegraphics[width=0.4\textwidth]{kapitoly/obrazky/E4/ESP32/BlockDiagram.png}
%    \caption{\centering \label{fig:E4-ESP32-BlockDiagram}Zapínání a~ochrana proti přepólovaní}
%\end{wrapfigure}

Mozkem celého trezoru je~čip~\href{https://www.espressif.com/sites/default/files/documentation/esp32-wrover-b_datasheet_en.pdf}{ESP32-wrover} \parencite{ESP32-WROVER-B}. 
Obsahuje dva dva\-a\-tři\-ce\-ti\-bi\-to\-vé procesory Xtensa LX6 taktované až~na~240Mhz. ESP32 \parencite{ESP32} má také na modulu wrover k dispozici 520 KiB SRAM 
a~4,8 nebo 16~Mb flash paměti. ESP32 má také k~dispozici řadu periferií, z~nichž asi nejvýznamnější je WiFi a~Bluetooth. Právě integrace 
WiFi a~Bluetoothu je také jeden z~primárních důvodů volby tohoto čipu. Dalším podstatným důvodem volby čipu ESP32 je jeho vysoký výpočetní výkon, 
alespoň na poměry mikrokontrolérů a~v~neposlední řadě také fakt, že s~tímto čipem už nějakou dobu pracuji a~tak s~ním již mám zkušenosti. 
Konkrétně wrover jsem pak zvolil kvůli dodatečné paměti PSRAM\footnote{Pseudo Static RAM} o velikosti 32 Mbit, 
\href{http://gamma.spb.ru/images/pdf/esp-psram32_datasheet_en.pdf}{ESP-PSRAM32} \parencite{ESP-PSRAM32}.

Kompletní zapojení je na obrázku \obr{fig:E4-sch_ESP32}

ESP32 také vyžaduje mít při startu definované úrovně na některých pinech, proto jsou zde čtyři pull-upy\footnote{To je rezistor připojen mezi dráhu a napájení.} 
a dva pull-downy,\footnote{To je rezistor připojen mezi dráhu a zem.} které definují výchozí stav pinů IO0, IO2, IO5, IO12, IO15 a EN \parencite{ESP32}.
\begin{table}[h]
    \centering
    \resizebox{\textwidth}{!}{%
    \begin{tabular}{l|l|l}
    \textbf{IO0}    & ovládá boot procesoru             & LOW při resetu ESP vstupuje do bootloaderu    \\
    \textbf{IO2}    & potvrzení pro spuštění bootu      & LOW potvrzuje                                 \\
    \textbf{IO12}   & určuje napětí komunikace s flash  & LOW znamená napětí 3,3~V a HIGH 1,8~V         \\
    \textbf{IO15}   & ovládá zprávy bootloaderu do UART & LOW zprávy vypíná a HIGH zapíná               \\
    \textbf{EN}     & reset pin                         & LOW ESP je drženo v resetu                    \\
    \end{tabular}%
    }
    \caption{Popis funkce pinů}
    \label{tab:COMPARATION}
\end{table}

\subsection*{Programátor}
Aby mohl uživatel trezor jednoduše naprogramovat, je~na~desce převodník USB-UART, \href{https://www.silabs.com/documents/public/data-sheets/cp2102n-datasheet.pdf}{CP2102} \parencite{cp2102}.
Protože však CP2102 není potřeba celou dobu provozu a~protože trezor nemá k~dispozici neomezený zdroj elektřiny, je~převodník zapnut jen ve~chvíli, 
kdy je~připojeno USB-C, které slouží jak pro nabíjení, tak pro programování. Vypínání převodníku je zajištěno tranzistorem Q3, který je zároveň společně 
s~DIP~switchem SW4 využit pro možnost zákazu programování, viz obrázek \obr{fig:E4-sch_ESP32}.

\section{Senzorika}
Mezi podstatné funkce trezoru patří jeho vnímání veličin jako čas, jeho náklon nebo okolní tlak.
Deska proto obsahuje tři nebo čtyři čipy (v závislosti na dostupnosti součástek), které trezoru poskytují gyroskop, akcelerometr, magnetický kompas,
barometr, RTC a také konektor pro připojení modulu GPS a GPRS. Díky těmto funkcím může trezor poskytnout možnost ovládání 
pomocí různých gest. 
Trezor třeba může sloužit, s~využitím magnetického kompasu a~LED kruhu, jako kompas, nebo se dá využít akcelerometr, 
aby se dal trezor odemknout jen v konkrétním náklonu. 

Všechny čipy zobrazené na obrázku \obr{fig:E4-sch_senzorika} komunikují s~ESP32 hlavně pomocí 
sběrnice I2C. Pro možnost zrychlení reakcí má však každý z~čipů také pin určený pro spuštění přerušení na procesoru, zapojení najdete na obrázku \obr{fig:E4-sch_senzorika}. 

To je užitečné, protože komunikaci na I2C řídí ESP32. Pokud se tedy ESP32 nerozhodne zeptat se jiného čipu na jím naměřené data, čip mu to po I2C nemá 
jak sdělit. Zároveň se však procesor nemůže bez ustání ptát na měření ostatních čipů, protože by pak nestíhal dělat nic jiného. Proto jsou čipy vybaveny 
pinem, který změní svou logickou hodnotu ve chvíli, kdy naměřené hodnoty splní nějaké podmínky. 
Například může být trezor naprogramován, aby se otevřel 
v~konkrétní čas. Tento čas se potom dá nastavit v RTC jako hodnota, při jejímž dosažení RTC přepne pin přerušení. ESP32 pak jen přečte logickou hodnotu 
pinu a vlastně ani nemusí komunikovat po I2C.

\paragraph{Akcelerometr, gyroskop a magnetický kompas}
\addcontentsline{toc}{paragraph}{Akcelerometr gyroskom a magnetický kompas}
Tyto funkce trezor má pro možnost sledování své pozice v prostoru. 
Díky akcelerometru má trezor k dispozici informaci o směru a velikosti svého zrychlení v prostoru.
Gyroskop poskytuje informaci o relativním natočení trezoru, což se může využít jako další podmínka pro otevření trezoru nebo pro různá ovládací gesta.
Magnetický kompas pak pochopitelně dodává informaci o natočení vůči zemskému magnetickému poli.

Na prvním prototypu verze E4 poskytoval akcelerometr, gyroskop i magnetický kompas čip \href{https://datasheet.lcsc.com/szlcsc/Bosch-Sensortec-BMX055_C94022.pdf}{BMX055} \parencite{bmx055},
protože však tento čip nebyl jednoduše dostupný, přidal jsem na další verzi i čip \href{https://datasheet.lcsc.com/szlcsc/TDK-InvenSense-MPU-6050_C24112.pdf}{MPU6050} \parencite{mpu6050},
který obsahuje akcelerometr a gyroskop a čip \href{https://datasheet.lcsc.com/szlcsc/QST-QMC5883L-TR_C192585.pdf}{QMC5883} \parencite{qmc5883}, který dodává magnetický kompas.
Na desce je tak místo pro všechny tři čipy, a pokud není k dispozici BMX055, jednoduše se osadí MPU6050 a QMC5883. 

\begin{figure}[htbp]
    \centering
    \includegraphics[width=\textwidth]{kapitoly/obrazky/E4/vnimani/BMX-MPU-QMC.png}
    \caption{Zapojení čipů BMX055, MPU6050 a QMC5883}
    \label{fig:E4-9axis}
\end{figure}

\subparagraph*{Akcelerometr}
U čipu BMX055 akcelerometr disponuje rozlišením 0,001 $m \cdot s^{-2}$ neboli 0,97mg při rozsahu měření $\pm2$ g. Jeho rozsah se ale dá nastavit, a to na $\pm2g$, $\pm4g$, $\pm8g$, $\pm16g$, 
podle rozsahu se také mění přesnost měření. Přesnost je totiž omezena velikostí dvanáctibitového registru do kterého se ukládají informace a při využívání většího 
rozsahu měření už tento registr není dostatečně velký aby uchovával stejnou přesnost.

MPU6050 má vedle BMX u akcelerometru stejné rozsahy měření disponuje však šestnáctibitovými registry a tak je sto dosáhnout i vetší přesnosti. Jeho maximální rozlišení je 
tak 60$\mu$g. Výměnou čipu BMX055 za čip MPU6050 jsem si tak polepšil i po straně přesnosti, přes to že tuto přesnost pravděpodobně nikdy BlackBox nevyužije. %todo jakože určitě ne ale přijde mi to divný zmiňovat i takhle mi to přijde jakési nekulantní ale možná je to zlozvyk z prezentace

\subparagraph*{Gyroskop}
U čipu BMX055 je gyroskop sto poskytovat informaci o uhlové rychlosti s rozlišení na 0.004 $°/s$ opět ale záleží na rozsahu měření měření kvůli stále stejné velikosti 
dvanáctibitového registru. Můžete si u něj vybrat z rozsahu $\pm125°/s$, $\pm250°/s$, $\pm500°/s$, $\pm1 000°/s$, $\pm2 000°/s$.

Po přechodu na čip MPU6050 jsem si opět po polepšil po straně přesnosti, i u gyroskopu totiž disponuje šestnáctibitovými registry a rozsahy měření jsou opět stejné jako u BMX055.

\subparagraph*{Magnetický kompas}
Čip BMX055 je sto měřit sílu magnetického pole s přesností až na  $0.3\mu T$ v rozsahu $\pm1200\mu T$ u os x a y, u osi z pak v rozsahu $\pm2500\mu T$.

Magnetometr který poskytuje čip QMC5883 pak měří v rozsahu $\pm8Gs$ což pro srovnání odpovídá $\pm800\mu T$ a s přesností až $2mGs$ opět pro srovnání to odpovídá $\pm0.2\mu T$.
Na rozdíl od BMX055 se u QMC5883 dá volit rozsah měření a tak si můžete vybrat z $\pm2Gs$ a nebo ze zmíněních $\pm8Gs$.

\newpage

\paragraph{Barometr, teplomer}
\addcontentsline{toc}{paragraph}{Barometr}
Barometr poskytuje informaci o okolním atmosferickém tlaku. 
Tato informace může sloužit pro rozeznávání nadmořské výšky. 
Také může BlackBox sloužit i jako jednoduchá meteorologická stanice, s možností měření tlaku i teploty.

Od doby, kdy jsem z nabídky JLCPCB vybíral čip \href{https://datasheet.lcsc.com/szlcsc/1907081118_Goertek-SPL06-007_C233787.pdf}{SPL06}, 
JLCPCB doplnilo do své nabídky několik dalších barometrů a dneska bych tedy dost možná zvolil jiný. Každopádně tehdy jsem volil 
mezi dvěma čipy, které byly v nabídce JLCPCB dostupné a SPL06 měl vyšší rozlišení a byl za téměř stejnou cenu.

SPL06 je sto měřit tlak v rozsahu 950 až 1050hPa s přesností na $0.06 hPa$ a je tak sto poznat změnu nadmořské víšky o 0.5m.

SPL06 má také vedle barometru i teploměr který je sto měřit teplotu od -40°C do 85°C s rozlišením na 0.01°C.

\begin{figure}[htbp]
    \centering
    \includegraphics[width=\textwidth]{kapitoly/obrazky/E4/vnimani/SPL06.png}
    \caption{Zapojení čipu SPL06}
    \label{fig:E4-SPL06}
\end{figure}

\newpage

\paragraph{RTC}
\addcontentsline{toc}{paragraph}{RTC}
Aby si trezor mohl zachovávat povědomí o aktuálním čase i ve chvíli, kdy je vypnut, má k dispozici čip 
\href{https://datasheet.lcsc.com/szlcsc/STMicroelectronics-M41T62Q6F_C113207.pdf}{M41T62} \parencite{m41t62}.

RTC je napájeno přímo z baterie, hned za ochranou proti přepólování, aby bylo sto uchovávat čas i ve vypnutém stavu.
Z toho důvodu jistě potěší nízká spotřeba 350 nA ve chvíli kdy jen uchovává čas a 35 µA ve chvíli kdy je aktivní I2C.
Maximální odchylka od skutečného času mak muže být 5s za měsíc provozu.

\begin{figure}[htbp]
    \centering
    \includegraphics[width=\textwidth]{kapitoly/obrazky/E4/vnimani/RTC.png}
    \caption{Zapojení čipu M41T62}
    \label{fig:E4-M41T62}
\end{figure}

\newpage

\paragraph{Konektor pro GPS/GPRS modul}
%\addcontentsline{toc}{paragraph}{Konektor pro A9G}
Ve chvíli, kdy jsem na trezor do\-pl\-ňo\-val čipy MPU6050 a QMC5883, jsem zároveň doplnil i tento konektor. Ve verzi, která je osazena MPU6050 a QMC5883 a nemá tedy osazen čip BMX055, 
je totiž více volných pinů. BMX055 totiž využívá pět pinů přerušení, zatím co MPU6050 a QMC5883 mají každý po jednom. Proto při nevyužití BMX055 zbudou tři volné piny.
Protože čip A9G\footnote{Čip využívám jako GPS a GPRS modul.} komunikuje po sběrnici UART, na rozdíl od ostatních čipu na desce. Pro UART však potřebuji
dva piny a ty kolidují s~piny přerušení čipu BMX055. Proto se konektor dá použít, jen pokud není osazen BMX055.

\begin{figure}[htbp]
    \centering
    \includegraphics[width=\textwidth]{kapitoly/obrazky/E4/vnimani/conn-A9G.png}
    \caption{Zapojení konektoru pro A9G}
    \label{fig:E4-A9G}
\end{figure}


\section{Ovládání západky a IR komunikace}

Zapojení je dostupná na obrázku \obr{fig:E4-sch_IR-Motor-enkoder}

\paragraph{IR komunikace}
\addcontentsline{toc}{paragraph}{IR komunikace}
IR slouží primárně pro identifikaci dveří, při vkládání většího množství dveří do stejného trezoru.
Trezor totiž počítá s možností vkládání více dveří do jednoho trezoru, což je jedna ze schopností, kterou více použije trezor jako hračka, než trezor jako bezpečnostní schránka.
Tento trezor s více dveřmi by zároveň mohl sloužit jako jakýsi displej a na to potřebuje vědět, které dveře jsou kde, na což slouží právě IR komunikace. %todo ???? <- ???

Jako IR přijímač jsem z nabídky JLCPCB \parencite{jlcpcb} zvolil \href{https://datasheet.lcsc.com/szlcsc/1912111437_Everlight-Elec-IRM-H936-TR2_C264266.pdf}{IRM-H936} \parencite{irm-h936}. 
V nabídce JLCPCB byly v době nárhu desky jen dva IR-přijímače, právě IRM-H936 a \href{https://datasheet.lcsc.com/szlcsc/2010221806_Everlight-Elec-IRM-H638T-TR2-DX_C390031.pdf}{IRM-H638},
z nichž IRM-H936 má skoro poloviční výšku a širší úhel záběru, a to byl důvod jeho volby.

Druhou částí IR komunikace je vysílač, který je zajištěn jednoduše IR ledkou \parencite{ir19-21c/tr8}.

\begin{figure}[htbp]
    \centering
    \includegraphics[width=\textwidth]{kapitoly/obrazky/E4/ir_motor_enkoder/IR.png}
    \caption{Zapojení IR vysílače a přijímače}
    \label{fig:E4-ir}
\end{figure}

\newpage

\paragraph{Ovládání motoru}
\addcontentsline{toc}{paragraph}{Ovládání motoru}
Protože motor je napájen z 5~V větve a protože ho připínám k napájení, a ne k zemi, nemůžu ho ovládat přímo z procesoru, proto je Q7 napojen na Q10, který je teprve řízen z~ESP. 
Kvůli napěťovým špičkám, které při běhu vznikají na komutátoru motoru, je zde i zpětná Schottkyho dioda, D3.

Motor bych sice mohl napájet z napětí 3,3V a nemusel bych tím pádem přidávat transistor navíc, ale zároveň bych tím zpomalil rychlost motoru %todo nový odstaveček
\footnote{otáčky motoru ale přes to mohu snížit dle potřeby}.
Další možností by bylo napájet motor přímo z napětí baterek a mohl bych tak motor spustit i bez zapnutí 5V zdroje. To by však znamenalo nutnost 
sofistikovanější řízení motoru protože by se motor točil různou rychlostí v závislosti na nabití baterii.

\begin{figure}[htbp]
    \centering
    \includegraphics[width=\textwidth]{kapitoly/obrazky/E4/ir_motor_enkoder/ovladani_motoru.png}
    \caption{Zapojení řízení motoru}
    \label{fig:E4-motor}
\end{figure}

\newpage

\paragraph{Enkodér}
\addcontentsline{toc}{paragraph}{Enkoder}

\begin{wrapfigure}[13]{R}{0.65\textwidth}
    \centering
    \includegraphics[width= 0.6\textwidth]{kapitoly/obrazky/E4/ir_motor_enkoder/pcb-enc.png}
    \caption{\label{fig:E4-enkoder_pcb}Vzhled enkorédu na desce}
\end{wrapfigure}
Aby bylo možno motor polohovat do správné polohy, je nutné mít zpětnou vazbu o~jeho poloze. Vzhledem k~tomu, že motor otáčí magnetem, samo se nabízí využít magnetický enkodér. 
Proto jsou na desce dvě digitální Hallovy sondy \href{https://datasheet.lcsc.com/szlcsc/Honeywell-SS360ST_C111924.pdf}{SS360NT} \parencite{ss360nt}, které se překlopí podle toho, 
v jakém pólu se nachází \footnote{jestli v severním nebo jižním pólu magnetu}. 

Na desce s~LED kruhem by sondy musely být na opačné straně než ledky, takže by se musely pájet ručně, protože JLCPCB osazuje jen z~jedné strany. Hlavní deska je ale zase, 
kvůli velikosti baterií, moc daleko od magnetu. 

Abych tedy nemusel dělat třetí desku jen kvůli enkodéru, zvolil jsem možnost vylomitelného enkodéru. Na hlavní desce jsem tedy nakreslil 
enkodér s konektorem a objel jsem ho frézou, aby se dal při montáži trezoru z desky vylomit a posunout do ideální polohy.
Vzhled enkodéru je vidět v horní části obrázku \obr{fig:E4-MainBoard}

\begin{figure}[htbp]
    \centering
    \includegraphics[width=\textwidth]{kapitoly/obrazky/E4/ir_motor_enkoder/enc.png}
    \caption{Zapojení enkodéru}
    \label{fig:E4-enkoder}
\end{figure}

\section{Tělo trezoru}
Dveře trezoru (o kterých je většina této práce) jsou schopny se zamknout do čehokoli se správným tvarem vstupního otvoru. 
Proto je možné velice jednoduše navrhnout libovolnou skříň, které by dveře sloužily jako zamykatelné víko. 

Aby trezor mohl sloužit (alespoň teoreticky) jako opravdový trezor a~ne jen jako hračka, navrhl jsem i bezpečnostní schránku, která by se zazdila do zdi. 
Vzhledem k~velikosti dveří by totiž ani nedobytná schránka nebyla bezpečná, jelikož by se dala jednoduše přenést celá.

\begin{figure}[htbp]
    \centering
    \includegraphics[width=240pt]{kapitoly/obrazky/E4/bedna/bedna.png}
    \includegraphics[width=240pt]{kapitoly/obrazky/E4/bedna/jen-bedna.png}
    \caption{Render bezpečnostního těla trezoru}
    \label{fig:E4-bedna}
\end{figure}

Tato bezpečnostní schránka by měla mít odolnou přední stěnu, ostatní stěny by pak měly zajišťovat jen pevné uchycení ve stěně.
Uchycení je tedy zajištěno tvarem schránky. Pro vyjmutí trezoru by tak bylo potřeba vybourat část zdi.
\enlargethispage{5mm}

Pro co možná největší bezpečnost by tedy bylo nutné přední stěnu trezoru alespoň zarovnat (lépe zahloubit) s povrchem stěny.
Zadní stěnu by pak bylo nutné umístit do prostoru zdi tak, aby nebylo snadné se probourat skrz tuto stěnu z druhé strany zdi.

% Zaver prace
\input{CHAPTERS/ZAVER.tex}

\appendix
\chapter{Obrazová příloha}

\newcommand{\OdsazeniNadpisu}{10mm}
%\section{Vzhled druhé elektronické varianty} 
\begin{figure}
    %\vspace{\OdsazeniNadpisu}
    %\centering
    %\includegraphics[width=0.5\textwidth]{kapitoly/obrazky/E2/predni_render.png}
    \chapter{Obrazová příloha}
	\section{Vzhled druhé elektronické varianty}
	\vspace{\OdsazeniNadpisu}
    \centering
    \includegraphics[width=0.7\textwidth]{kapitoly/obrazky/E2/predni_render.png}

    \caption{Render varianty E2}
    \label{fig:E2-render}
\end{figure}

\begin{figure}
    \section{Vzhled třetí elektronické varianty}
	\centering
	\includegraphics[width=\textwidth]{kapitoly/obrazky/E3/rendery.pdf}
	\caption{Rendery varianty E3}
	\label{fig:E3-renderi}
\end{figure}

\begin{figure}
	\section{Vzhled druhé mechanické varianty}
	\vspace{\OdsazeniNadpisu}
    \centering
    \includegraphics[width=\textwidth]{kapitoly/obrazky/M2/predni_render.PNG}
    \caption{Render varianty M2}
    \label{fig:M2-render}
\end{figure}



\begin{figure}
\section{Simulace pevnosti}
    %\vspace{\OdsazeniNadpisu}
    \centering
    \includegraphics[width=370pt]{kapitoly/obrazky/E4/machanika_tlakove_desky/simulace/F100N,primo,uprostred,pohled_zepredu.png}
    \includegraphics[width=370pt]{kapitoly/obrazky/E4/machanika_tlakove_desky/simulace/F100N,primo,uprostred,pohled_zezadu.png}
    \caption{Pevnostní simulace těla nahoře je pohled zepředu a dole pohled zezadu \centering}
    Tato simulace testuje působení síly přímo na tělo, což není působení, které by v provozu nastávalo. Takovéto namáhání je ale o dost náročnější
    než to, které by reálně nastalo.
    \label{fig:E4-simulace_tela} %todo těla čeho? trezor je hranatý? 
\end{figure}

\begin{figure}
    \centering
    \includegraphics[width=\textwidth]{kapitoly/obrazky/E4/machanika_tlakove_desky/simulace/zjednodusena_sestava_pri_F100N_nezobrazeno_napeti_pod_1,5MPa.png}
    \caption{Simulace sestavy}
    Jak je vidět, tak i sílu 100~N dokáže sendvič z terčíku, pružné podložky a~snímací desky rozložit na dostatečnou plochu, aby napětí v těle nestouplo 
    nad cca 3~MPa. Na obrázku je zobrazené jen napětí nad 1,5~MPa.
    \label{fig:E4-simulace_tlakovky}
\end{figure}

\begin{figure}[h]
    \centering
    \includegraphics[width=\textwidth]{kapitoly/obrazky/E4/zapadka/simulace/napeti_D1-M5000.png}
    \caption{\centering Simulace napětí v západce při kroutícím momentu \\ 5000 N $\cdot$ mm, což na rameni 48~mm znamená sílu působící na kolík 104~N}

    \vspace{10mm}

    \includegraphics[width=\textwidth]{kapitoly/obrazky/E4/zapadka/simulace/Dislokace_D10-M5000.png}
    \caption{\centering Zobrazení deformace, pro lepší zobrazení je deformace zdesetinásobená}
    \label{fig:E4-simulace_zapadky}
\end{figure}

\begin{figure}
\section{Obrázky DPS}
    %\vspace{\OdsazeniNadpisu}
    \centering
    \includegraphics[width=\textwidth]{kapitoly/obrazky/E4/elektronika_tlakove_desky/civka.png}
    \caption{Vzhled reliéfu cívky}
    \label{fig:E4-relief_civka}
\end{figure}

\begin{figure}
    \centering
    \includegraphics[width=\textwidth]{kapitoly/obrazky/E4/E4-LEDBoard.png}
    \caption{Vzhled desky s kruhem WS2812 a snímáním tlakové desky}
    \label{fig:E4-LedDeska}
\end{figure}

\begin{figure}
    \centering
    \includegraphics[width=\textwidth]{kapitoly/obrazky/E4/E4-MainBoard.png}
    \caption{Vzhled hlavní desky}
    \label{fig:E4-MainBoard}
\end{figure}



\begin{figure}
	\section{Schémata}
    \centering
    \includegraphics[width=0.93\textheight, angle=90]{kapitoly/ctvrta_elektronicka_varianta/E4_zapojeni/B.B.png}
    \caption{Propojení jednotlivých systémů -- schéma}
    \label{fig:E4-sch_B.B}
\end{figure}
\begin{figure}
    \centering
    \includegraphics[width=0.93\textheight, angle=90]{kapitoly/ctvrta_elektronicka_varianta/E4_zapojeni/zdroj.png}
    \caption{Zapojení zdroje -- schéma}
    \label{fig:E4-sch_zdroj}
\end{figure}
\begin{figure}
    \centering
    \includegraphics[width=0.93\textheight, angle=90]{kapitoly/ctvrta_elektronicka_varianta/E4_zapojeni/nabijecka.png}
    \caption{Zapojení nabíjecího obvodů -- schéma}
    \label{fig:E4-sch_nabijecka}
\end{figure}
\begin{figure}
    \centering
    \includegraphics[width=0.93\textheight, angle=90]{kapitoly/ctvrta_elektronicka_varianta/E4_zapojeni/ESP32.png}
    \caption{Zapojení ESP32 a programátoru -- schéma}
    \label{fig:E4-sch_ESP32}
\end{figure}
\begin{figure}
    \centering
    \includegraphics[width=0.93\textheight, angle=90]{kapitoly/ctvrta_elektronicka_varianta/E4_zapojeni/senzorika.png}
    \caption{Zapojení senzorů BMX055, MPU6050, QMC5883, M41T62, SPL06 a a konektoru pro A9G -- schéma \centering}
    \label{fig:E4-sch_senzorika}
\end{figure}
\begin{figure}
    \centering
    \includegraphics[width=0.93\textheight, angle=90]{kapitoly/ctvrta_elektronicka_varianta/E4_zapojeni/IR_motor_enkoder.png}
    \caption{Zapojení IR komunikace, motoru a enkodéru -- schéma}
    \label{fig:E4-sch_IR-Motor-enkoder}
\end{figure}
\begin{figure}
    \centering
    \includegraphics[width=0.93\textheight, angle=90]{kapitoly/ctvrta_elektronicka_varianta/E4_zapojeni/next_board.png}
    \caption{Zapojení LDC1614 -- schéma}
    \label{fig:E4-sch_next-board}
\end{figure}
\begin{figure}
    \centering
    \includegraphics[width=0.93\textheight, angle=90]{kapitoly/ctvrta_elektronicka_varianta/E4_zapojeni/WS2812.png}
    \caption{Zapojení LED WS2812 na desce trezoru -- schéma}
    \label{fig:E4-sch_WS2812}
\end{figure}

\chapter{Ostatní přílohy}

%\setcounter{section}{0}

\addtocounter{section}{1}
\addcontentsline{toc}{section}{\protect\numberline{\thesection}Seznam obrázků} 
\listoffigures

\addtocounter{section}{1}
\addcontentsline{toc}{section}{\protect\numberline{\thesection}Literatura}
\printbibliography[title=Literatura]

\section{Technická dokumentace použitých součástek}
https://datasheet.lcsc.com/szlcsc/Shenzhen-Ruilongyuan-Elec-SMD1206P200TF_C20988.pdf            % polyfuse
https://datasheet.lcsc.com/szlcsc/Vishay-Intertech-SI3447CDV-T1-E3_C145421.pdf                  % silové MOSFET
https://datasheet.lcsc.com/szlcsc/Changjiang-Electronics-Tech-CJ-CJ3134K_C110100.pdf            % primární logický MOSFET
https://datasheet.lcsc.com/szlcsc/Silergy-Corp-SY8009BABC_C79314.pdf                            % stepdown (jen na prvním prototipu E4)
https://datasheet.lcsc.com/szlcsc/Sunlord-MWSA0603-2R2MT_C112126.pdf                            % cívka pro spínané zdroje
https://datasheet.lcsc.com/szlcsc/Feeling-Tech-FP6276AXR-G1_C83308.pdf                          % stepup
https://datasheet.lcsc.com/szlcsc/Seaward-Elec-SE9017-HF_C115752.pdf                            % nabíječka
https://datasheet.lcsc.com/szlcsc/Alpha-Omega-Semicon-AOS-AO3400A_C20917.pdf                    % logický MOSFET pro napájení programátoru
https://www.silabs.com/documents/public/data-sheets/cp2102n-datasheet.pdf                       % USP -> UART CP2102
https://datasheet.lcsc.com/szlcsc/Changjiang-Electronics-Tech-CJ-MMBT3904_C20526.pdf            % NPN transistor pro programátor
https://www.espressif.com/sites/default/files/documentation/esp32-wrover-b_datasheet_en.pdf     % ESP32 wrover (jakože osobně mám k ESP celou složku (datasheeth/esp32), rozhodně ne jedno PDF) %todo co s tím? přiložit k práci nebo všechno někde pohledat na netu aby byl odkaz?
https://datasheet.lcsc.com/szlcsc/Bosch-Sensortec-BMX055_C94022.pdf                             % BMX055 gyroskop, akcelerometr, magnetický kompas
https://datasheet.lcsc.com/szlcsc/STMicroelectronics-M41T62Q6F_C113207.pdf                      % M41T60 RTC
https://datasheet.lcsc.com/szlcsc/SK-SD103AWS_C109204.pdf                                       % dioda pro RTC
https://datasheet.lcsc.com/szlcsc/Nihon-Dempa-Kogyo-NX3215SA-32-768KHZ-STD-MUA-14_C156244.pdf   % krystal pro RTC
https://datasheet.lcsc.com/szlcsc/1907081118_Goertek-SPL06-007_C233787.pdf                      % SPL06 barometr
https://datasheet.lcsc.com/szlcsc/1912111437_Everlight-Elec-IRM-H936-TR2_C264266.pdf            % IRM-H936 IR přijímač
https://datasheet.lcsc.com/szlcsc/Everlight-Elec-IR19-21C-TR8-AQL_C142316.pdf                   % IR led
https://datasheet.lcsc.com/szlcsc/1903061001_MDD-Jiangsu-Yutai-Elec-SS54_C22452.pdf             % Schottkyho dioda k motoru
https://datasheet.lcsc.com/szlcsc/Magnesensor-Tech-MST-MH253ESO_C114369.pdf                     % Hallova sonda
https://datasheet.lcsc.com/szlcsc/2005251033_Worldsemi-WS2812B-Mini_C527089.pdf                 % WS2812 inteligentní ledky
http://www.ti.com/lit/ds/symlink/ldc1416.pdf                                                    % LDC1314/11614
https://datasheet.lcsc.com/szlcsc/TDK-InvenSense-MPU-6050_C24112.pdf                            % MPU6050 akcelerometr, gyroskop
https://datasheet.lcsc.com/szlcsc/QST-QMC5883L-TR_C192585.pdf                                   % QMC5883 magnetický kompas
https://datasheet.lcsc.com/szlcsc/1808280153_STMicroelectronics-LD39200PU33R_C222192.pdf        % 3V3 stabilizátor
https://datasheet.lcsc.com/szlcsc/Honeywell-SS360ST_C111924.pdf                                 % Hallova sondy

\section{Záznamy o využití LDC}
https://www.ti.com/lit/an/snoa930b/snoa930b.pdf?ts=1614377965173&ref_url=https\%253A\%252F\%252Fwww.ti.com\%252Fsitesearch\%252Fdocs\%252Funiversalsearch.tsp\%253FsearchTerm\%253DLDC 
                                                                                                % Application note k LDC
https://www.ti.com/lit/an/slya048a/slya048a.pdf?ts=1614368129537&ref_url=https\%253A\%252F\%252Fwww.ti.com\%252Fproduct\%252FFDC2214
                                                                                                % Application note k LDC
https://www.ti.com/lit/an/snoaa04/snoaa04.pdf?ts=1614378318596&ref_url=https\%253A\%252F\%252Fwww.ti.com\%252Fproduct\%252FLDC1614-Q1
                                                                                                % Application note k LDC

\section{Použité projekty}
https://github.com/RoboticsBrno/ArduinoLearningKitStarter/projects                              % ALKS git repositori
https://github.com/TVavrinec/SOC-text/actions/new                                               % minulá SOČka

% to jsou zdroje použity na E4 takže součástky která byli jen na starších verzích tu nejsou

\addtocounter{section}{1}
\addcontentsline{toc}{section}{\protect\numberline{\thesection}Seznam tabulek}
\listoftables

\end{document}

% // todo velké obrázky do příloh 
%todo ověřit číslování stránek práce 
%todo seznam literatury 
%todo závěr