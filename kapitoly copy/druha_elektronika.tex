[poznámka: to co je v hranatých závorkách je jen poznámka a ve finální verzi by nemělo být.]

Druhá verze elektronické varianty testovala použitelnost signalizačního kruhu, o dvanácti ledkách, kolem uprostřed dveří
umístěného enkodéru. Jako základ trezoru jsem použil první mechanickou variantu, ze které jsem odstranil zamykací kola 
a doplnil o servo, řídící elektroniku a již zmínění kruh ledek a enkodér.
Vzhledem k tomu, že se jednalo jen o hrubý prototyp, tak neměl specializovanou desku a elektroniku - tedy tvořila jen změť kabelů 
a kousek univerzální desky, takže nemám elektronickou variantu tohoto zapojení. [schéma jsem kreslil jen na tabuli a to asi rok 
zpátky, takže když bych došel k závěru že je potřeba, tak se dá udělat, ale je to práce navíc a nepovažoval bych to za podstatné]

Trezor měl pro komunikaci s uživatelem tedy kruh o dvanácti ledkách a jeden vstupní prvek, enkodér s tlačítkem.
Ovládání tedy bylo od tohoto odvozené a trezor se zmáčknutím zapnul a tlačítko pak dál sloužilo jako potvrzování výběru.
Člověk tak mohl pomocí enkodéru otáčet jedinou rozsvícenou ledku a stiskem potvrdit, vstupní kód tedy mohl vypadat 
například jako čas, a uživatel ho zadal na kruhu odvozeném od ručičkových hodin, proto právě dvanáct ledek.
Konkrétní ovládání je pochopitelně závislé na nahraném programu a mohlo by se tedy jednoduše změnit do libovolné podoby -
to co popisuji je jen konkrétní možnost, kterou jsem použil.

E2-render.png