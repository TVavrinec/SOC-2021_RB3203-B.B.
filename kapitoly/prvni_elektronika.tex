\section{První verze}
\label{E1-vyvoj}

Dal jsem se tedy do kreslení trezoru. Pochopitelně ne nějaké nedobytné pevnosti, ale malé
krabičky,\footnote{128x128mm} na které se dají ukazovat základy programování. 

Jelikož se mi na podobné výrobky osvědčila jako materiál překližka, navrhoval jsem trezor s úmyslem výroby z překližky za využití laseru. 

%Konstrukce byla z velké části přizpůsobená dostupné elektronice, kterou jsem měl k dispozici. 
%a která musela být stejně použita poněkud odlišně, než jak byla zamýšlena. 
%Neměl jsem totiž čas ani rozpočet, navrhovat a především vyrábět konkrétní elektroniku
%pro výrobek, který se měl předložit dětem na letním táboře ani ne za týden. %todo ?? souvisí s termínem tábora? 
Jako základ pro elektroniku  jsem použil univerzální 
desku \href{https://github.com/RoboticsBrno/ArduinoLearningKitStarter}{ALKS}\footnote{Arduino Learning Kit Starter} \parencite{ALKS}, 
kterých jsem měl dostatečnou zásobu. Ovládací prvky,\footnote{dvě tlačítka, dva potenciometry a tři barevné LED} jsem umístil na horní stranu trezoru. 
ALKS má v původní variantě tři tlačítka. Já jsem však jedno musel pomocí magnetu a jazýčkového magnetického konektoru použít
jako kontrolu, zda jsou dveře otevřeny či zavřeny. %todo takže zbyla jenom dvě pro ovládání trezoru 
Jako zámek jsem pak použil obyčejné servo
SG90, které jednoduše zajelo svou páčkou do drážky ve dveřích, a tím jim zabránilo 
se otevřít. Celý systém pak napájela malá powerbanka, která se dala vyjmout a nabít  
a používala se i ve dvou dalších verzích.

% Tato konstrukce měla kvůli uspěchanému návrhu 
%spoustu problémů. Většinou však šlo o problémy, které by nebylo těžké odstranit a nebylo
%tedy třeba předělávat celý koncept návrhu. 

V těsném závěsu za touto elektronickou variantou jsem ale dostal poža\-da\-vek i na čistě mechanickou verzi trezoru. 
To byl následně jeden z důvodů velkých změn, a to i změny samotného konceptu zařízení.