\section{Třetí elektronická varianta}


Třetí verze elektronické varianty do značné míry vycházela z předchozí, druhé verze, a dále na ní stavěla. Asi nejzjevnější změna bylo navýšení počtu 
ledek z dvanácti, jakožto hodiny, na šedesát jakožto minuty, což pochopitelně znamenalo i zvětšení kruhu. Na desku se ale přidaly i nové funkcionality,
a to gyroskop, pro možnost znalosti náklonu zařízení, akcelerometr, pro znalost směru a velikosti zrychlování, magnetický kompas, pro určení světových
stran, RTC (Real Time Clock, hodiny reálného času), pro znalost přesného času a také GPS pro možnost určení svojí polohy.
Také jsem použil, po vzoru mechanické varianty, rotační západku, což znamenalo, že na stejný trezor se daly použít jak mechanické tak 
elektronické dveře.

Tato verze měla dvě podverze, které se lišily motorem.
\begin{figure}[htbp]
    \centering
    \includegraphics[width=160pt]{kapitoly/obrazky/E3/motory/hodinovyStrojek.jpg}
    \includegraphics[width=160pt]{kapitoly/obrazky/E3/motory/zluty_motor.jpg}
    \caption{fotky obrázků}
    \label{fig:E3-motory}
\end{figure}

\begin{figure}[htbp]
    \centering
    \includegraphics[width=\textwidth]{kapitoly/obrazky/E3/rendery.pdf}
    \caption{render varianty E3}
    \label{fig:E3-renderi}
\end{figure}

Přes velké množství funkcí jsem, kvůli několika věcem ale opět koncept přepracoval. Hlavním důvodem změn bylo náročné uložení rotační západky, 
které vyžadovalo ozubený věnec a několik dalších tisknutých dílů.

\newpage