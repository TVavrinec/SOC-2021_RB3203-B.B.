\section{Třetí verze}


Třetí verze do značné míry vycházela z předchozí verze a dále na ní stavěla. Asi nejvýraznější změna bylo navýšení počtu 
ledek z dvanácti (hodiny) na šedesát (minuty), což pochopitelně znamenalo i zvětšení kruhu. Na desku se ale přidaly i nové funkcionality,
a~to gyroskop pro možnost znalosti náklonu zařízení, akcelerometr pro znalost směru a~velikosti zrychlování, magnetický kompas pro určení světových
stran, RTC (Real Time Clock, hodiny reálného času), pro znalost přesného času a také GPS pro možnost určení svojí polohy.
Také jsem použil, po vzoru mechanického trezoru, rotační západku, což znamenalo, že na stejný trezor se daly použít jak mechanické tak 
elektronické dveře.

Z důvodů použití převodů pro otáčení rotační západky byly nově také použité díly tištěné na 3D tiskárně. 

Tato verze měla dvě varianty, které se lišily motorem.
\begin{figure}[htbp]
    \centering
    \includegraphics[width=160pt]{kapitoly/obrazky/E3/motory/hodinovyStrojek.jpg}
    \includegraphics[width=160pt]{kapitoly/obrazky/E3/motory/zluty_motor.jpg}
    \caption{fotografie obou testovaných motorů} 
    \label{fig:E3-motory}
\end{figure}


