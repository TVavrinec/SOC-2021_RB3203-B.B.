\section*{první trezor}
\addcontentsline{toc}{section}{první trezor}

Dal jsem se tedy do kreslení trezoru, pochopitelně ne do nějaké nedobytné pevnosti, ale do malé
krabičky, na které se dají ukazovat principy elektronických zámků. Jelikož se mi na podobné
výrobky osvědčila jako materiál překližka, navrhoval jsem vše s úmyslem výroby z překližky 
za využití laseru. Konstrukce byla z velké části přizpůsobená dostupné elektronice, kterou 
jsem měl k dispozici, a která musela být stejně použita poněkud odlišně než jak byla zamýšlena.
Němel jsem totiž čas, a vlastně ani rozpočet, navrhovat a především vyrábět konkrétní elektroniku
pro výrobek, který se měl předložit dětem ani ne za týden. Použil jsem tedy starší univerzální 
desku ALKS (\href{https://github.com/RoboticsBrno/ArduinoLearningKitStarter}{Arduino Learnikg Kit Starter})
kterých jsem měl dostatečnou zásobu. Ovládací prvky, dvě tlačítka, dva potenciometry a tři
barevné ledky, tedy celý ALKS jsem umístil na horní stranu trezoru. ALKS má v původní variantě
tři tlačítka. Já jsem však jedno musel pomocí magnetu a jazýčkového magnetického konektoru použít
jako kontrolu, zda jsou dveře otevřeny či zavřeny. Jako zámek jsem pak použil obyčejné servo
SG90, které velice jednoduše zajelo svou páčkou do drážky ve dveřích, a tím jim zabránilo 
se otevřít. Celý systém pak napájela malá powerbanka, která se dala vyjmout a nabýt, 
a používala se i ve dvou dalších verzích. Tato konstrukce měla kvůli uspěchanému návrhu 
spoustu problémů. Většinou však šlo o problémy, které by nebylo těžké odstranit a nebylo
tedy třeba předělávat celý koncept návrhu. V těsném závěsu za touto elektronickou variantou,
jsem ale dostal požadavek i na čistě mechanickou verzi trezoru. To byl následně jeden z 
velkých důvodů velkých změn, a to i změny samotného konceptu zařízení.

\newpage