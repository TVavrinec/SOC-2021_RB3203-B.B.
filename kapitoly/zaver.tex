Cílem mé práce bylo vyvinout systém v podobě trezoru pro výuku programování, mechanické stavby a náplň různých kolektivních her. 

Plánovaných cílu jsem dosáhl, přesto že se v průběhu vývoje částečně částečně změnil koncept celého systému. Trezor měl totiž původně sloužit primárně pro výuku ale v průběhu vývoje 
se objevilo daleko víc požadavků na nasazení trezoru jako hotového zařízení např. v nejrůznějších táborových hrách. % Kvůli pandemii a částečně i kvůli dosud probíhajícímu vývoji se těchto her ale dosud uskutečnilo velmi málo.

Díky teto práci jsem se zdokonalil v návrhu tištěních spojů. Také jsem pro návrh DPS začal využívat program KiCad, zatím dříve jsem využíval Eagle, který není špatný ale KiCad 
mi vyhovuje o něco více. Díky výrobě desek jsem se naučil používat program \href{https://github.com/yaqwsx/KiKit}{KiKit} \parencite{KiKit}, 
který vytvořil Jan Mrázek a zásadně usnadňuje výrobu podkladu pro reálnou výrobu DPS.


%V závěru by mělo být:
%\begin{itemize}
%    \item Rekapitulace cíle práce
%    \item Dosáhnul jsem jej? Ano, nebo ne?
%    \item Zhodnocení průběhu práce             %totálně netuším co o tom napsat
%    \item Co mi práce dala?
%\end{itemize}
