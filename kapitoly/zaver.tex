\pagestyle{empty}
\pagestyle{plain}
{\setlength{\voffset}{-10mm} 

Cílem mé práce bylo vyvinout systém v podobě trezoru pro výuku programování, mechanické stavby a náplň různých kolektivních her. 

Plánovaných cílů jsem dosáhl, přesto, že se v průběhu vývoje částečně změnil koncept celého systému. Trezor měl původně sloužit primárně pro výuku, ale v průběhu vývoje 
se objevilo daleko víc požadavků a možností na nasazení trezoru jako hotového zařízení např. v nejrůznějších táborových nebo městských hrách. 

Navrhl a vyrobil jsem dvě verze trezoru -- mechanickou pro mladší uživatele a jednodušší použití a elektronickou s rozsáhlými možnostmi využití pro náročné zájemce. 
K oběma verzím jsem také připravil výkresovou dokumentaci. 

Hotový elektronický trezor obsahuje wifi, bluetooth, gyroskop, akcelerometr,
magnetický kompas, barometr, RTC a také konektor pro připojení modulu GPS a GPRS. 
Dále je trezor osazen všesměrovou tlakovou deskou a kruhem LED.

Díky tomuto vybavení trezor poskytuje možnost 
ovládání pomocí různých gest, například otáčení, naklápění nebo zvedání. 
Trezor také může určovat svoji polohu nebo se po určité době 
provozu s minimální spotřebou \uv{vzbudit} a začít vysílat a přijímat signály.

%Trezor tak třeba může sloužit, s využitím magnetickéhokompasu a LED kruhu, jako kompas, nebo se dá využít 
%akcelerometr, aby se dal trezor odemknout jen v konkrétním náklonu.

% Kvůli pandemii  se těchto her ale dosud uskutečnilo velmi málo.

Díky této práci jsem se zdokonalil v návrhu tištěných spojů. Také jsem pro návrh DPS začal využívat program KiCad, zatímco dříve jsem využíval Eagle, který není špatný, ale KiCad 
mi vyhovuje o něco více. Díky výrobě desek jsem se naučil používat program \href{https://github.com/yaqwsx/KiKit}{KiKit} \parencite{KiKit}, 
který vytvořil Jan Mrázek a který zásadně usnadňuje přípravu podkladů pro reálnou výrobu DPS.
}

%V závěru by mělo být:
%\begin{itemize}
%    \item Rekapitulace cíle práce
%    \item Dosáhnul jsem jej? Ano, nebo ne?
%    \item Zhodnocení průběhu práce             %totálně netuším co o tom napsat
%    \item Co mi práce dala?
%\end{itemize}
