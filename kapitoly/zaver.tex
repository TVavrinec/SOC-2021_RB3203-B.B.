Cílem mé práce bylo vyvinout systém v~podobě trezoru pro výuku programování, mechanické stavby 
a~náplň různých zážitkových akcí. 

Zhodnocení použití BlackBoxu na DDM Helceletova Brno je v příloze \ref{papir} na straně \pageref{papir}. 

Plánovaných cílů jsem dosáhl, přesto, že se v~průběhu vývoje částečně změnil koncept celého systému. 
BlackBox měl původně sloužit primárně pro výuku, ale v~průběhu vývoje 
se objevilo daleko víc požadavků a~možností na nasazení trezoru jako hotového zařízení, např. v~nejrůznějších 
táborových nebo městských hrách. 

Navrhl a~vyrobil jsem dvě verze trezoru -- mechanickou pro mladší uživatele
a~jednodušší použití a~elektronickou s~rozsáhlými možnostmi využití pro 
náročné zájemce -- viz \hyperref[shrnuti_vysledky]{Shrnutí dosažených výsledků}.

K~oběma verzím jsem také připravil výkresovou dokumentaci (v~přilo\-že\-ných souborech).

%Hotový elektronický trezor obsahuje wifi, bluetooth, gyroskop, akcelerometr,
%magnetický kompas, barometr, RTC a také konektor pro připojení modulu GPS a GPRS. 
%Dále je trezor osazen všesměrovou tlakovou deskou a~kruhem LED.

Díky této práci jsem se zdokonalil v~návrhu tištěných spojů. Také jsem pro návrh DPS začal využívat program KiCad, 
zatímco dříve jsem využíval Eagle.
Díky výrobě desek jsem se naučil používat program \href{https://github.com/yaqwsx/KiKit}{KiKit} \parencite{KiKit}, 
který vytvořil Jan Mrázek a~který zásadně usnadňuje přípravu podkladů pro reálnou výrobu DPS navržených v~KiCadu.

\subsection*{Shrnutí dosažených výsledků} \label{shrnuti_vysledky}

\begin{itemize}
    \item \B{Zámek}  \\ Zámek (viz kapitola \ref{zamek}) je plně funkční a~je realizován pomocí mechanizmu bajonetu, který je opatřen zpětnou západkou ovládanou motorem a~magnetickou spojkou. 
        Díky magnetické spojce může být zámek vodotěsný a~tzv. měkký (dá se zamknout, i~když je již zamčen). 
    \item \B{LED kruh} \\ LED kruh (viz kapitola \ref{WS2812}) je plně funkční a~je realizován pomocí 60 inteligentních RGB LED \href{https://cdn-shop.adafruit.com/datasheets/WS2812B.pdf}{WS2812} \parencite{WS2812}.
    \item \B{Tlaková plocha} \\ Tlaková plocha (viz kapitoly \ref{E4-tlakovka} a~\ref{E4-mech_tlakovky}) je plně funkční, 
        je sto detekovat jak polohu doteku, tak jeho sílu. 
        Plocha je realizována pomocí čipu \href{https://www.ti.com/lit/ds/symlink/ldc1612.pdf?ts=1612018658531&ref_url=https%253A%252F%252Fwww.google.com%252F}{LDC1614} \parencite{LDC1614}, 
        který měří deformaci vodivého terčíku. Měření tlaku je navíc natolik přesné, že se plocha dá využít i~jako váha schopná detekovat i~tělesa vážící 
        desetiny gramu. %ověřeno jsem schopnej poznat jestli na ploše leží 0806 kondenzátor kterej ani nejsem schopnej zvážit mikrováhou která mi bez problému zváží 0.1g 
    \item \B{Wifi a~Bluetooth} \\ WiFi a~Bluetooth (viz kapitola \ref{ESP32}) jsem vyřešil v~rámci volby řídícího 
        modulu \href{https://www.espressif.com/sites/default/files/documentation/esp32-wrover-b_datasheet_en.pdf}{ESP32-wrover} 
        \parencite{ESP32-WROVER-B}, 
            který má integrovaný Wifi a~Bluetooth modul.
    \item \B{Gyroskop} \\ Gyroskop (viz kapitola \ref{gyroskop}) jsem vyřešil pomocí čipu \href{https://datasheet.lcsc.com/szlcsc/TDK-InvenSense-MPU-6050_C24112.pdf}{MPU6050} (dříve \href{https://datasheet.lcsc.com/szlcsc/Bosch-Sensortec-BMX055_C94022.pdf}{BMX055} \parencite{bmx055}).
    \item \B{Akcelerometr} \\ Akcelerometr (viz kapitola \ref{akcelerometr}) jsem vyřešil pomocí čipu \href{https://datasheet.lcsc.com/szlcsc/TDK-InvenSense-MPU-6050_C24112.pdf}{MPU6050} (dříve \href{https://datasheet.lcsc.com/szlcsc/Bosch-Sensortec-BMX055_C94022.pdf}{BMX055} \parencite{bmx055}).
    \item \B{Magnetický kompas} \\ Magnetický kompas (viz kapitola \ref{magnetometr}) jsem vyřešil pomocí čipu \href{https://datasheet.lcsc.com/szlcsc/QST-QMC5883L-TR_C192585.pdf}{QMC5883} \parencite{qmc5883} (dříve \href{https://datasheet.lcsc.com/szlcsc/Bosch-Sensortec-BMX055_C94022.pdf}{BMX055} \parencite{bmx055}).
    \item \B{RTC} (hodiny reálného času) \\ RTC (viz kapitola \ref{RTC}) jsem vyřešil pomocí čipu 
        \href{https://datasheet.lcsc.com/szlcsc/STMicroelectronics-M41T62Q6F_C113207.pdf}{M41T62} \parencite{m41t62}.
    \item \B{Programátor} s~možností zákazu programování
        \\ Programátor (viz kapitola \ref{programator}) je plně funkční a~řešení jeho odpojení navíc šetří energii ve chvíli, kdy programátor není využíván.
    \item \B{Barometr} \\ Barometr (viz kapitola \ref{barometr}) je plně funkční a~realizován na čipu 
    \href{https://datasheet.lcsc.com/szlcsc/1907081118_Goertek-SPL06-007_C233787.pdf}{SPL06} s~přesností měření 
            0,06~hPa. Je tak schopen rozpoznat změnu nadmořské výšku o~0,5~m.
    \item \B{Nabíječka} \\ Nabíječka (viz kapitola \ref{nabijecka}) je plně funkční a~je realizována pomocí čipu \href{https://datasheet.lcsc.com/szlcsc/Seaward-Elec-SE9017-HF_C115752.pdf}{SE9017} \parencite{se9017}.
    \item \B{GPS} \\ GPS (viz kapitola \ref{A9}) bohužel z~důvodu nedostatku času není implementována přímo na desce a~je pro ní vyhrazen jen konektor. 
            Realizována by ale byla pomocí modulu A9, který 
            je i~v~plánu na nynějších verzích používat prostřednictvím zmíněného konektoru. 
            Na budoucích verzích pak bude modul osazen přímo na desce a~bude zároveň používán jako koprocesor.
    \item \B{GPRS} \\ GPRS (viz kapitola \ref{A9}) bylo v~plánu realizovat stejně jako GPS pomocí modulu A9.
    \item \B{IR komunikace} \\ IR komunikace (viz kapitola \ref{IR}) je plně funkční.
    %\item Kompletní zvuková karta %todo otázka jestli zmiňovat je to jediná věc co tam není ani přes konektor. Každopádně to bylo v plánu a na nějaké budoucí verzi to určitě ještě bude.
\end{itemize}

Díky tomuto vybavení trezor poskytuje možnost 
ovládání pomocí různých gest, například otáčení nebo naklápění. 
Trezor také může určovat svoji polohu nebo se po určité době 
provozu s~minimální spotřebou \uv{vzbudit} a~začít vysílat a~přijímat signály nebo reagovat na své senzory.

Výrobní podklady a~historie vývoje hardwaru BlackBoxu je na webu 
\href{https://github.com/RoboticsBrno/RB3203-BlackBox}{github.com/RoboticsBrno/RB3203-BlackBox}.
Softwarovou část zpracoval v~samostatné práci s~názvem \textit{Software pro BlackBox} můj kolega Tomáš Rohlínek 
a~je dostupná na adrese 
\href{https://github.com/RoboticsBrno/BlackBox_library}{github.com/RoboticsBrno/BlackBox\_library}.

BlackBox již svoje využití našel a~do budoucna se s~ním počítá nejen na Robotický tábor 2021, 
ale i~na spoustu dalších akcí, ať už pořádaných Robotárnou nebo kýmkoliv jiným  (jen co skončí pandemie). 

% Kvůli pandemii a částečně i kvůli dosud probíhajícímu vývoji se těchto her ale dosud uskutečnilo velmi málo.

%todo planované využití helceletka Pter Sedlař

%V závěru by mělo být:
%\begin{itemize}
%    \item Rekapitulace cíle práce
%    \item Dosáhnul jsem jej? Ano, nebo ne?
%    \item Zhodnocení průběhu práce             %totálně netuším co o tom napsat
%    \item Co mi práce dala?
%\end{itemize}
