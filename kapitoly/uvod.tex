\chapter*{Úvod}
\addcontentsline{toc}{chapter}{Úvod}

Na konci července roku 2019 jsem dostal za úkol navrhnout výrobek pro děti na příměstský tábor pobočky D.D.M.Helceletova Brno, Robotárny. %todo odkaz 
 Poža\-dav\-kem 
byla jednoduchá a levná konstrukce, kterou děti zvládnou sestavit za pár dní a ve zbytku času tábora si stihnou vyzkoušet základy programování 
s využitím tohoto výrobku. Z tohoto důvodu jsem začal vyvíjet elektronicky řízený trezor. Původní vize trezoru se ale rychle změnila na poměrně 
univerzální zařízení, kterému zůstala schopnost sloužit jako trezor. Také se přidala čistě mechanická varianta pro mladší účastníky táborů.

Trezor byl pro mě poněkud změnou oproti mé dřívějším práci, která se do té doby vždy točila kolem různých létajících nebo častěji jezdících robotu s velkým důrazem na orientaci v prostotu.
Trezor je oproti těmto vozítkům daleko statičtější a protože se sám nepohybuje má jeho vnímání prostoru jiné požadavky. Vozítka také vždy počítaly s jistou univerzalitou senzoriky i mechaniky
zatímco trezor by měl být upravitelný jen po stránce softwaru. Další odlišností je menší konkurence která je u různých robotických stavebnic poměrně veliká jak si můžete přečíst 
v má dřívější práci \href{https://github.com/TVavrinec/SOC-text/blob/master/SOČ.pdf}{Postav s svého prvního robota}.

% % todo tohle je spíš anotace, úvod musíme zásadně rozšířit
% třeba takhle? snažil jsem se tam procpat i tu minulo sočku.

% Úvod práce má za cíl uvést:
% \begin{itemize}
%     \item cíl práce
%     \item jak ho chcete dosáhnout
%     \item popis tématu práce, musí být výstižný, ale stručný a poutavý
% \end{itemize}

% Úvodu a závěru práce je třeba věnovat obzvláště velkou pozornost.
% Myslete na to, že úvod a někdy i závěr si porotce čte jako první, teprve potom, jestli ho práce zaujme se rozhodne, zda ji přečte celou.

\newpage