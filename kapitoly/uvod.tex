\chapter*{Úvod}
\addcontentsline{toc}{chapter}{Úvod}

Na konci července roku 2019 jsem dostal za úkol navrhnout výrobek pro děti na příměstský tábor pobočky DDM Helceletova Brno, Robotárny. %todo odkaz 
 Poža\-dav\-kem 
byla jednoduchá a levná konstrukce, kterou děti zvládnou sestavit za pár dní a ve zbytku času tábora si stihnou vyzkoušet základy programování 
s využitím tohoto výrobku. Z tohoto důvodu jsem začal vyvíjet elektronicky řízený trezor. Z původní vize trezoru se ale poměrně rychle vyvinulo
univerzální elektronické zařízení, kterému zůstala schopnost sloužit jako trezor. Také se přidala čistě mechanická varianta trezoru pro mladší účastníky táborů a volnočasových aktivit.

Trezor byl pro mě poněkud změnou oproti mé dřívějším práci, která se do té doby vždy točila kolem různých létajících nebo častěji jezdících robotů s velkým důrazem na orientaci v prostoru.
Trezor je oproti těmto vozítkům daleko statičtější, a protože se sám nepohybuje, má jeho vnímání prostoru jiné požadavky. Vozítka také vždy počítala s jistou univerzalitou senzoriky i~mechaniky,
zatímco trezor by měl být upravitelný jen po stránce softwaru. 

Další odlišností trezoru je menší konkurence, která je u různých robotických stavebnic poměrně veliká, jak si můžete přečíst 
v mé dřívější práci \href{https://github.com/TVavrinec/SOC-text/blob/master/SOČ.pdf}{Postav si svého prvního robota}.
%todo Jiné trezory (elektronizované, ve formě stavebnice pro děti/tábory) jsem zatím nikde nenašel. 


%todo popis, proč se oddělila mechanická a ele verze, původní záměr byl mít mecha verzi pro oba trezory stejnou  

 %todo vlastně vznikly trezory dva - odstaveček 


Na konci července roku 2019 jsem dostal za úkol navrhnout výrobek pro děti na příměstský tábor
pobočky DDM Helceletova Brno, Robotárny. 
Poža\-dav\-kem byla jednoduchá a levná konstrukce,
kterou děti zvládnou sestavit za pár dní a ve zbytku času tábora se jim ukážou základy programování
s~využitím tohoto výrobku. Proto, a také pro poněkud nižší věk účastníků, jsme se s vedoucím 
Robotárny, Jiřím Váchou, rozhodli jít cestou \uv{trezoru}. To byl rozdíl oproti našim běžným 
výrobkům, které většinou měly možnost pohybu, ale byly pro děti náročnější na výrobu
a pochopitelně i cena u nich šla nahoru.


% % todo tohle je spíš anotace, úvod musíme zásadně rozšířit
% třeba takhle? snažil jsem se tam procpat i tu minulo sočku.

% Úvod práce má za cíl uvést:
% \begin{itemize}
%     \item cíl práce
%     \item jak ho chcete dosáhnout
%     \item popis tématu práce, musí být výstižný, ale stručný a poutavý
% \end{itemize}

% Úvodu a závěru práce je třeba věnovat obzvláště velkou pozornost.
% Myslete na to, že úvod a někdy i závěr si porotce čte jako první, teprve potom, jestli ho práce zaujme se rozhodne, zda ji přečte celou.

\newpage