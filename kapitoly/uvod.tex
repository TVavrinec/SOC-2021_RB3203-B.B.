\small
Na konci července roku 2019 jsem dostal za úkol navrhnout výrobek pro děti na příměstský tábor pobočky DDM Helceletova Brno, \href{https://helceletka.cz/robotarna/}{Robotárny} \parencite{robotarna}.
Poža\-dav\-kem byla jednoduchá a levná konstrukce s elektronikou, kterou děti zvládnou sestavit za pár dní a ve zbytku času tábora si stihnou vyzkoušet základy programování 
s~využitím tohoto výrobku. Z tohoto důvodu jsem začal vyvíjet elektronicky řízený trezor. Postup vývoje trezoru je popsán v~kapitolách \ref{E-vyvoj} a~\ref{M-vyvoj}.

Z původní vize trezoru se ale rychle vyvinulo poměrně univerzální elektronické zařízení, kterému zůstala schopnost sloužit jako trezor.
Také využití trezoru se rozšířilo -- přibyly mu nové funkce a hlavním cílem už není pouze trezor  s~dětmi stavět a programovat, 
ale také ho využívat jako herní prvek při táborových hrách. 
Trezor se tedy dá s dětmi jak stavět a učit s~jeho pomocí programování, tak ho využívat jako hotové zařízení při hrách pořádaných Robotárnou a dalšími subjekty.
Popis možností současné verze trezoru je v~kapitole \ref{E4}.

Dále přibyl požadavek na vývoj čistě mechanické varianty trezoru pro volnočasové aktivity, jednorázové akce nebo mladší účastníky táborů.
Mechanický trezor je z pochopitelných důvodů výrazně levnější než elektronický. Tím pádem se dá počítat s výrobou tohoto trezoru i na menších a levnějších akcích, ze kterých 
si účastníci trezor odnesou, což by v případě elektronické varianty znamenalo výrazně vyšší cenu i časovou náročnost.
\normalsize
%Původně byla mechanická i elektronická verze vyvíjené tak, aby se mechanická verze dala jednoduše upravit na elektronickou, což popisuji v kapitole \ref{M1-vyvoj}.

%Z důvodů náročné konstrukce elektronické varianty trezoru došlo v její poslední verzi k oddělení obou variant (viz kapitola \ref{E4-vyvoj}).
    % Trezor byl pro mě poněkud změnou oproti mé dřívějším práci, která se do té doby vždy točila kolem různých létajících nebo častěji jezdících robotů s velkým důrazem na orientaci v prostoru.
    % Trezor je oproti těmto vozítkům daleko statičtější, a protože se sám nepohybuje, má jeho vnímání prostoru jiné požadavky. Vozítka také vždy počítala s jistou univerzalitou senzoriky i~mechaniky,
    % zatímco trezor by měl být upravitelný jen po stránce softwaru. 

    % Další odlišností trezoru je menší konkurence, která je u různých robotických stavebnic poměrně veliká, jak si můžete přečíst 
    % v mé dřívější práci \href{https://github.com/TVavrinec/SOC-text/blob/master/SOČ.pdf}{Postav si svého prvního robota}.
    % Tato hračka/trezor/výrobek/zařízení je svým způsobem unikátní, jiné trezory (elektronizované, ve formě stavebnice pro děti/tábory) jsem zatím nikde nenašel. 

%todo popis, proč se oddělila mechanická a ele verze, původní záměr byl mít mecha verzi pro oba trezory stejnou  

%todo kapitola použití 

%todo zmínka o softwaru 

    % Na konci července roku 2019 jsem dostal za úkol navrhnout výrobek pro děti na příměstský tábor
    % pobočky DDM Helceletova Brno, Robotárny. 
    % Poža\-dav\-kem byla jednoduchá a levná konstrukce,
    % kterou děti zvládnou sestavit za pár dní a ve zbytku času tábora se jim ukážou základy programování
    % s~využitím tohoto výrobku. Proto, a také pro poněkud nižší věk účastníků, jsme se s vedoucím 
    % Robotárny, Jiřím Váchou, rozhodli jít cestou \uv{trezoru}. To byl rozdíl oproti našim běžným 
    % výrobkům, které většinou měly možnost pohybu, ale byly pro děti náročnější na výrobu
    % a pochopitelně i cena u nich šla nahoru.

%VARIANTA: 

%-- napsat cíl práce jako hotový, bez historie, vývoje a spol ... ???  

% Úvod práce má za cíl uvést:
% \begin{itemize}
%     \item cíl práce
%     \item jak ho chcete dosáhnout
%     \item popis tématu práce, musí být výstižný, ale stručný a poutavý
% \end{itemize}


\newpage