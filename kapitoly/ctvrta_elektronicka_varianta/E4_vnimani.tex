\section{Senzorika}
Mezi podstatné funkce trezoru patří jeho vnímání veličin jako čas, jeho náklon nebo okolní tlak.
Deska proto obsahuje tři nebo čtyři, v závislosti na dostupnosti součástek, čipy které, které trezoru poskytují gyroskop, akcelerometr, magnetický kompas,
barometr, RTC (Real Time Clock, hodiny reálného času) a také konektor pro připojení modulu GPS a QPRS. Díky těmto funkcím může trezor poskytnout možnost ovládání pomocí různých gest. 
Trezor třeba může sloužit, s~využitím magnetického kompasu a~lek kruhu, jako kompas, nebo se dá využít akcelerometr 
aby se dal trezor odemknout jen v konkrétním náklonu. Všechny čipy zobrazené na obrázku [\ref{fig:E4-sch_vnimani}] komunikují s~ESP hlavně pomocí 
sběrnice I2C. Pro možnost zrychlení reakcí má však každý z~čipů také pin určení pro spuštění přerušení na procesoru. 

\begin{figure}[htbp]
    \centering
    \includegraphics[width=\textwidth]{kapitoly/obrazky/E4/vnimani/sch.png}
    \caption{zapojení čipu pro vnímání}
    \label{fig:E4-sch_vnimani}
\end{figure}

\newpage

To je užitečné z~toho důvodu že komunikaci na I2C řídí ESP. Pokud se tedy ESP nerozhodne zeptat se jiného čipu na jím naměřené data, čip mu to nemá 
jak zdělit. Zároveň se však procesor nemůže bez ustání ptát na měření ostatních čipu protože by pak nestíhal dělat nic jiného. Proto jsou čipy vybaveny 
pinem který změní svou logickou hodnotu ve chvíli kdy naměřené hodnoty splní nějaké podmínky. 
Například muže být trezor naprogramován aby se otevřel 
v~konkrétní čas. Tento čas se potom dá nastavit v RTC jako hodnota při jejímž dosažení RTC přepne pin přerušení, a esp pak jen přečte logickou hodnotu 
pinu a nemusí komunikovat po I2C. % tuto větu by nebylo k zahození nakrouhat

\paragraph{Akcelerometr gyroskom a magnetický kompas}
\addcontentsline{toc}{paragraph}{Akcelerometr gyroskom a magnetický kompas}
Tyto funkce trezor má pro možnost vnímání své pozice v prostoru. 
Díky akcelerometru má trezor k dispozici informaci o směru a velikosti svého zrychlení v prostoru.
Giroskop poskytuje informaci o relativním natočení trezoru, což se muže využít jako další podmínka pro otevření trezoru nebo pro různá ovládací gesta.
Magnetický kompas pak pochopitelně dodává informaci o natočení vůči zemskému magnetickému poli.

Na prvním prototypu verze E4 poskytoval akcelerometr, gyroskop i magnetický kompas čip \href{https://datasheet.lcsc.com/szlcsc/Bosch-Sensortec-BMX055_C94022.pdf}{BMX055}, 
protože však tento čip nebyl jednoduše dostupní přidal jsem na další verzi i čipy \href{https://datasheet.lcsc.com/szlcsc/TDK-InvenSense-MPU-6050_C24112.pdf}{MPU6050},
který obsahuje akcelerometr a gyroskop, a \href{https://datasheet.lcsc.com/szlcsc/QST-QMC5883L-TR_C192585.pdf}{QMC5883} který dodává magnetický kompas.
Na desce je tak místo pro všechny tři čipy a pokud není k dispozici BMX055, tak se jednoduše osadí MPU6050 a QMC5883. 

\begin{figure}[htbp]
    \centering
    \includegraphics[width=\textwidth]{kapitoly/obrazky/E4/vnimani/BMX-MPU.png}
    \caption{zapojení čipů BMX055, MPU6050 a QMC5883}
    \label{fig:E4-9axis}
\end{figure}

\newpage

\paragraph{Barometr}
\addcontentsline{toc}{paragraph}{Barometr}
Barometr poskytuje informaci o okolním atmosferickém tlaku. Tato informace může sloužit pro rozeznávání nadmořské výšky, a to s relativní přesností až na 0.5m. Také může trezor sloužit 
i jako jednoduchá meteorologická stanice.

Od doby kdy jsem z nabídky JLCPCB vybíral čip SPL06, JLCPCB doplnilo do své nabídky několik dalších barometru a dneska bych tedy asi zvolil jiný. Každopádně tehdy jsem volil 
mezi dvěma čipy které byli v nabídce JLCPCB dostupné a SPL06 měl vyšší rozlišení.

\begin{figure}[htbp]
    \centering
    \includegraphics[width=\textwidth]{kapitoly/obrazky/E4/vnimani/SPL06.png}
    \caption{zapojení čipu SPL06}
    \label{fig:E4-SPL06}
\end{figure}

\newpage

\paragraph{RTC}
\addcontentsline{toc}{paragraph}{RTC}
Aby si trezor mohl zachovávat povědomí o aktuálním čase i ve chvíli kdy je vypnut, má k dispozici čip \href{https://datasheet.lcsc.com/szlcsc/STMicroelectronics-M41T62Q6F_C113207.pdf}{M41T62}.

M41T62 je napájen přímo z baterie a je na ni napojen ještě před zapínacím obvodem, hned za ochranou proti přepólování.

\begin{figure}[htbp]
    \centering
    \includegraphics[width=\textwidth]{kapitoly/obrazky/E4/vnimani/RTC.png}
    \caption{zapojení čipu M41T62}
    \label{fig:E4-M41T62}
\end{figure}

\newpage

\paragraph{Konektor pro A9G}
\addcontentsline{toc}{paragraph}{Konektor pro A9G}
Ve chvíli kdy jsem na trezor doplňoval čipy MPU6050 a QMC5883, jsem zároveň doplnil i tento konektor. Ve verzi která je osazena MPU6050 a QMC5883, a nemá tedy osazen čip BMX055 je totiž více
volných pinu protože BMX055 využívá pět interrupt pinu takže zbudou tři volné piny. Protože čip A9G komunikuje po sběrnici UART, na rozdíl od ostatních čipu na desce. Pro UART však potřebuji
dva piny a ty kolidují s interrupt piny čipu BMX055 a proto se konektor dá použít jen pokud není osazen BMX055.

\begin{figure}[htbp]
    \centering
    \includegraphics[width=\textwidth]{kapitoly/obrazky/E4/vnimani/conn-A9G.png}
    \caption{zapojení konektor pro A9G}
    \label{fig:E4-A9G}
\end{figure}

\newpage