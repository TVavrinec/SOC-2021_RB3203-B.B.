\section{Elektronika tlakové desky}
\label{E4-tlakovka}
Tlaková plocha se díky pružné podložce a nažehlovací fólii může ve všech směrech naklánět. Díky tomu se při používání mění vzdálenost od čtyř snímacích
cívek. Tlaková plocha je primárně terčík, který slouží jako jádro cívky, která zvětšuje svou indukčnost, když se terčík přibližuje a naopak.

\begin{figure}[htbp]
    \centering
    \includegraphics[width=200pt]{kapitoly/obrazky/E4/elektronika_tlakove_desky/civka_tercik_LDC.png}
    \caption{Schematické zobrazení cívky a terčíku \parencite{LDC-cd1}}
    \label{fig:E4-sch_civka_tercik}
\end{figure}

Pro snímání indukčnosti používám čip \href{https://www.ti.com/lit/ds/symlink/ldc1612.pdf?ts=1612018658531&ref_url=https%253A%252F%252Fwww.google.com%252F}{LDC1614} \parencite{LDC1614}
nebo \href{https://www.ti.com/lit/ds/symlink/ldc1312.pdf?ts=1612017390818&ref_url=https%253A%252F%252Fwww.google.com%252F}{LDC1314}.
Tyto čipy se liší prakticky jen rozlišením. LDC1314 disponuje dvanáctibitovým AD převodníkem a LDC1614 dvacetiosmibitovým AD převodníkem 
a je tak schopen detekovat pohyb terčíku s rozlišením až na 10 nm.

\begin{figure}[htbp]
    \centering
    \includegraphics[width=\textwidth]{kapitoly/obrazky/E4/elektronika_tlakove_desky/moje_zapojeni.png}
    \caption{Zapojení čipu LDC1314 na desce trezoru}
    \label{fig:E4-LDC}
\end{figure}

Čip LDC komunikuje po sběrnici I2C, která umožňuje komunikaci jednoho mastera\footnote{Čip, který řídí komunikaci.} s až 128 slavy.\footnote{Čipy, které přijímají příkazy od mastera a pouze mu odpovídají.} 
LDC také umožňuje volbu ze dvou I2C adres, aby se dala adresa změnit v případě 
kolize s jiným čipem, který by měl stejnou adresu.\footnote{Např. aby se daly použít dva čipy LDC na jedné sběrnici I2C.}

\newpage

Cívky použité na trezoru jsou vyrobeny jako reliéf ve vrstvě mědi přímo na DPS. Jejich vzhled jsem navrhoval v simulátoru od firmy Texas Instruments, 
vytvořeném konkrétně pro LDC čipy, a s pomocí popisů reálných aplikací \parencite{LDC-cd0}, \parencite{LDC-cd1}, které firma Texas Instruments zveřejňuje.

% obrázek ze simulátoru

Výsledná cívka je vytvořena na dvouvrstvé desce a na každé vrstvě má patnáct závitů s dráhou o síle 0,152~mm se stejně velkou mezerou, 
vzhled je vidět na obrázku \obr{fig:E4-relief_civka}.

Celý trezor obsahuje dvě samostatné elektronické desky, přičemž na jedné je osazen jen kruh z ledek WS2812 a snímání tlakové desky, které zabírá 
většinu této desky, což je vidět na obrázku \obr{fig:E4-LedDeska}.

\newpage