\section{ESP32 a jeho programátor}

%\begin{wrapfigure}{L}[10mm]{0.4\textwidth}
%    \centering
%    \includegraphics[width=0.4\textwidth]{kapitoly/obrazky/E4/ESP32/BlockDiagram.png}
%    \caption{\centering \label{fig:E4-ESP32-BlockDiagram}Zapínání a~ochrana proti přepólovaní}
%\end{wrapfigure}

Mozkem celého trezoru je~čip~\href{https://www.espressif.com/sites/default/files/documentation/esp32-wrover-b_datasheet_en.pdf}{ESP32-wrover} \parencite{ESP32-WROVER-B}. 
Obsahuje dva dva\-a\-tři\-ce\-ti\-bi\-to\-vé procesory Xtensa LX6 taktované až~na~240Mhz. ESP32 \parencite{ESP32} má také na modulu wrover k dispozici 520 KiB SRAM 
a~4,8 nebo 16~Mb flash paměti. ESP32 má také k~dispozici řadu periferií, z~nichž asi nejvýznamnější je WiFi a~Bluetooth. Právě integrace 
WiFi a~Bluetoothu je také jeden z~primárních důvodů volby tohoto čipu. Dalším podstatným důvodem volby čipu ESP32 je jeho vysoký výpočetní výkon, 
alespoň na poměry mikrokontrolérů a~v~neposlední řadě také fakt, že s~tímto čipem už nějakou dobu pracuji a~tak s~ním již mám zkušenosti. 
Konkrétně wrover jsem pak zvolil kvůli dodatečné paměti PSRAM\footnote{Pseudo Static RAM} o velikosti 32 Mbit, 
\href{http://gamma.spb.ru/images/pdf/esp-psram32_datasheet_en.pdf}{ESP-PSRAM32} \parencite{ESP-PSRAM32}.

Kompletní zapojení je na obrázku \obr{fig:E4-sch_ESP32}

ESP32 také vyžaduje mít při startu definované úrovně na některých pinech, proto jsou zde čtyři pull-upy\footnote{To je rezistor připojen mezi dráhu a napájení.} 
a dva pull-downy,\footnote{To je rezistor připojen mezi dráhu a zem.} které definují výchozí stav pinů IO0, IO2, IO5, IO12, IO15 a EN \parencite{ESP32}.
\begin{table}[h]
    \centering
    \resizebox{\textwidth}{!}{%
    \begin{tabular}{l|l|l}
    \textbf{IO0}    & ovládá boot procesoru             & LOW při resetu ESP vstupuje do bootloaderu    \\
    \textbf{IO2}    & potvrzení pro spuštění bootu      & LOW potvrzuje                                 \\
    \textbf{IO12}   & určuje napětí komunikace s flash  & LOW znamená napětí 3,3~V a HIGH 1,8~V         \\
    \textbf{IO15}   & ovládá zprávy bootloaderu do UART & LOW zprávy vypíná a HIGH zapíná               \\
    \textbf{EN}     & reset pin                         & LOW ESP je drženo v resetu                    \\
    \end{tabular}%
    }
    \caption{Popis funkce pinů}
    \label{tab:COMPARATION}
\end{table}

\subsection*{Programátor}
Aby mohl uživatel trezor jednoduše naprogramovat, je~na~desce převodník USB-UART, \href{https://www.silabs.com/documents/public/data-sheets/cp2102n-datasheet.pdf}{CP2102} \parencite{cp2102}.
Protože však CP2102 není potřeba celou dobu provozu a~protože trezor nemá k~dispozici neomezený zdroj elektřiny, je~převodník zapnut jen ve~chvíli, 
kdy je~připojeno USB-C, které slouží jak pro nabíjení, tak pro programování. Vypínání převodníku je zajištěno tranzistorem Q3, který je zároveň společně 
s~DIP~switchem SW4 využit pro možnost zákazu programování, viz obrázek \obr{fig:E4-sch_ESP32}.
