\section{Ovládání západky a IR komunikace}

Zapojení je dostupné na obrázku \obr{fig:E4-sch_IR-Motor-enkoder}.

\paragraph{IR komunikace}
\addcontentsline{toc}{paragraph}{IR komunikace}
IR slouží primárně pro identifikaci dveří, při vkládání většího množství dveří do stejného trezoru.
Trezor totiž počítá s možností vkládání více dveří do jednoho trezoru, což je jedna ze schopností, kterou více použije trezor jako hračka, než trezor jako bezpečnostní schránka.
Tento trezor s více dveřmi by zároveň mohl sloužit jako jakýsi displej. Na to potřebuje vědět, které dveře jsou kde, na což slouží právě IR komunikace. %todo ???? <- ???

Jako IR přijímač jsem z nabídky JLCPCB \parencite{JLCPCB} zvolil \href{https://datasheet.lcsc.com/szlcsc/1912111437_Everlight-Elec-IRM-H936-TR2_C264266.pdf}{IRM-H936} \parencite{irm-h936}. 
V~nabídce JLCPCB byly v době nárhu desky jen dva IR při\-jí\-ma\-če, právě IRM-H936 a \href{https://datasheet.lcsc.com/szlcsc/2010221806_Everlight-Elec-IRM-H638T-TR2-DX_C390031.pdf}{IRM-H638},
z nichž IRM-H936 má skoro poloviční výšku a širší úhel záběru, a to byl důvod jeho volby.

Druhou částí IR komunikace je vysílač, který je zajištěn jednoduše IR LED \parencite{ir19-21c/tr8}.

\begin{figure}[htbp]
    \centering
    \includegraphics[width=\textwidth]{kapitoly/obrazky/E4/ir_motor_enkoder/IR.png}
    \caption{Zapojení IR vysílače a přijímače}
    \label{fig:E4-ir}
\end{figure}

\paragraph{Ovládání motoru}
\addcontentsline{toc}{paragraph}{Ovládání motoru}
Protože motor je napájen z 5~V větve a protože ho připínám k napájení a ne k zemi, nemůžu ho ovládat přímo z procesoru. Proto je Q7 napojen na Q10, který je teprve řízen z~ESP. 
Kvůli napěťovým špičkám, které při běhu vznikají na komutátoru motoru, je zde i zpětná Schottkyho dioda, D3.

Motor bych sice mohl napájet z napětí 3,3~V a nemusel bych tím pádem přidávat tranzistor navíc, ale zároveň bych tím zpomalil rychlost motoru.\footnote{Otáčky
 motoru ale přesto mohu snížit dle potřeby.}
Další možností by bylo napájet motor přímo z baterií a mohl bych tak motor spustit i bez zapnutí 5~V zdroje. To by však znamenalo nutnost 
sofistikovanějšího řízení motoru, protože by se motor točil různou rychlostí v~závislosti na nabití baterií.

\begin{figure}[htbp]
    \centering
    \includegraphics[width=\textwidth]{kapitoly/obrazky/E4/ir_motor_enkoder/ovladani_motoru.png}
    \caption{Zapojení řízení motoru}
    \label{fig:E4-motor}
\end{figure}

\paragraph{Enkodér}
\addcontentsline{toc}{paragraph}{Enkoder}

\begin{wrapfigure}{R}{0.4\textwidth}
    \centering
    \includegraphics[width= 0.4\textwidth]{kapitoly/obrazky/E4/ir_motor_enkoder/pcb-enc.png}
    \caption{\label{fig:E4-enkoder_pcb}Vzhled enkodéru na desce \centering}
\end{wrapfigure}
Aby bylo možno motor polohovat do správné polohy, je nutné mít zpětnou vazbu o~jeho poloze. Vzhledem k~tomu, že motor otáčí magnetem, samo se nabízí využít magnetický enkodér. 
Proto jsou na desce dvě digitální Hallovy sondy \href{https://datasheet.lcsc.com/szlcsc/Honeywell-SS360ST_C111924.pdf}{SS360NT} \parencite{ss360nt}, které se překlopí podle toho, 
u jakého magnetického pólu se nachází. 

Na desce s~LED kruhem by sondy musely být na opačné straně než LED, takže by se musely pájet ručně, protože JLCPCB osazuje jen z~je\-dné strany. Hlavní deska je ale zase, 
kvůli velikosti baterií, moc daleko od magnetu. 

Abych tedy nemusel dělat třetí desku jen kvůli enkodéru, zvolil jsem možnost vylomitelného enkodéru. Na hlavní desce jsem tedy nakreslil 
enkodér s~konektorem a objel jsem ho frézou, aby se dal při montáži trezoru z desky vylomit a posunout do ideální polohy.
Vzhled enkodéru je vidět v~horní části obrázku \obr{fig:E4-MainBoard}.

\begin{figure}[htbp]
    \centering
    \includegraphics[width=\textwidth]{kapitoly/obrazky/E4/ir_motor_enkoder/enc.png}
    \caption{Zapojení enkodéru}
    \label{fig:E4-enkoder}
\end{figure}
