
\begin{figure}[htbp]
    \section*{Ukosy}
    \addcontentsline{toc}{section}{Ukosy}    
    \centering
    \includegraphics[width=400]{kapitoly/obrazky/E4/ukozy/ukladaci_ukosy.pdf}
    \caption{ukládací úkosy}
    Aby bylo jednoduší při zavírání dveře správně natočit, mají zarážky na vnitřní straně velké úkosy, které tak zvětšují na vnitřní straně 
    vůli a při zasouvání navedou dveře do správné pozice.
    \label{fig:E4-ukosy}
\end{figure}

\begin{figure}[htbp]
    \centering
    \includegraphics[width=400]{kapitoly/obrazky/E4/ukozy/simetrie_zarazek.png}
    \caption{symetrie zarážky}
    Zarážky na obvodu otvoru mají obě kontaktní plochy stejné. Sice by mohlo být výhodné přizpůsobit tvar strany, kolem které, se pohybuje západka, 
    pohybu západky. Západka by tak mohla mýt vedení v průběhu celého pohybu. Pro symetrii jsem se však rozhodl kvůli možnosti díl s otvorem otočit.
    To je výhodné při stavbě s dětmi, kvůli zmenšení počtu chyb kterých se děti můžou při stavbě dopustit, a stráta vedení není tak zásadní.
    \label{fig:E4-simetrie_zarazky}
\end{figure}

\clearpage
\newpage