\section{Přehled}

Dnešní verze elektronického trezoru se zamyká pomocí mechanizmu bajonetu a magneticky řízené zpětné západky. 

Elektronika je vybavena čipem ESP32,
který obsahuje dva procesory Xtensa LX6, WiFi a bluetooth. Dále je trezor vybaven čipem BMX055 nebo dvojicí čipů MPU6050 a QMC5883 které poskytují 
gyroskop, akcelerometr a magnetický kompas. Dále je zde SPL06, barometr s přesností měření 0,06~hPa, což umožňuje rozeznat změnu nadmořské výšky 
o polovinu metru. Další systém trezoru je IR s přijímačem a vysílačem, který je zde pro možnost jednoznačné identifikace dveří, ale pochopitelně může 
sloužit i pro jiný účel. Deska je také vybavena hodinami reálného času a má vlastní programátor pro usnadnění programování. Vedle ESP32 je zde asi 
nejvýznamnějším čipem LDC1614, případně LDC1314, který umožňuje funkci tlakové plochy.

%todo doplnit popis a využití a možnosti tlakové desky 

\begin{table}[h]
    \centering
    \resizebox{\textwidth}{!}{%
    \begin{tabular}{@{}lll@{}}                                                                                 \\ \midrule
    \textbf{ESP32}              & dva procesory Xtensa LX6, WiFi a bluetooth    &                                                                           \\
    \textbf{BMX055}             & gyroskop, akcelerometr, magnetický kompas     & možno nahradit dvojicí čipů MPU6050 a QMC5883                             \\
    \textbf{SPL06}              & barometr                                      & rozlišení až 0,06hPa což umožňuje rozeznat změnu nadmořské výšky o 0,5m   \\
    \textbf{IRM-H936 a IR led}  & IR komunikace                                 &                                                                           \\
    \textbf{LDC1614}            & snímání tlakové desky                         & počítá se s možnou záměnou za LDC1314                                     \\
    \textbf{CP2102}             & programátor                                   & s hardwarově zajištěným odpojováním napájení, pokud není využíván          \\ \bottomrule
    \end{tabular}%
    }
    \caption{shrnutí elektronického vybavení}
    \label{tab:COMPARATION}
\end{table}

%    ? proč dřevo a plast a ne ocel nebo jiný kov (vysvětlit!)
    
%    úkosy
%        a) obrázek
%        b) k čemu (navádění do otvoru v rámu)
%    
%    SLA tisk
%        a) zběžné porovnání SLA s FDM (pro SLA nemusím model krájet na 5 kusů)
%        b) obrázek ze sliceru
%        c) smrštění => korekce modelu
%        d) možná video z tisku?
%
%    vypalované díly
%        a) potřebuju PES 3 (jinak se nedostanu k laseru)
%        b) proč laseru
%        c) co po vypálení (obrousit, sestavit)
%        d) možná video?
%
%    uložení tlakové desky
%        a) potřeby uložení
%        b) jak a proč
%        c) její pohyby
%        d) její snímání
%        e) možná video?
%    
%    možná video odemčení a zamčení (otázka jak bude stíhat Tomáš Rohlínek (programátor), nic neslibuju)
%
%    vzhledy beden
%        a) primární krabice
%        b) více dveřová krabice s IR identifikací
%        c) ta věc do zdi (bezpečnostní)

\newpage