\section{Přehled}

Dnešní verze elektronického trezoru se zamyká pomocí mechanizmu bajonetu a~magneticky řízené zpětné západky. 

Elektronika je vybavena čipem ESP32 \parencite{ESP32}, \parencite{ESP32-WROVER-B},
který obsahuje dva procesory Xtensa LX6, WiFi a bluetooth. Dále je trezor vybaven čipem BMX055 \parencite{bmx055} nebo dvojicí čipů MPU6050 \parencite{mpu6050} 
a QMC5883 \parencite{qmc5883}, které poskytují 
gyroskop, akcelerometr a magnetický kompas. Dále je zde SPL06 \parencite{spl06}, barometr s rozlišením 0,06~hPa, což umožňuje rozeznat změnu nadmořské výšky 
o polovinu metru. Další systém trezoru je možnost IR komunikace, která je zde pro možnost jednoznačné identifikace dveří, ale pochopitelně může 
sloužit i pro jiný účel. Deska je také vybavena RTC a má vlastní programátor pro usnadnění programování. Vedle ESP32 je zde asi 
nejvýznamnějším čipem LDC1614 \parencite{LDC1614}, případně LDC1314, který umožňuje funkci tlakové plochy (viz kapitoly \ref{E4-mech_tlakovky}, \ref{E4-tlakovka}).

%todo doplnit popis a využití a možnosti tlakové desky 

\begin{table}[h]
    \centering
    \resizebox{\textwidth}{!}{%
    \begin{tabular}{@{}lll@{}} 
    \textbf{čip} & \textbf{popis} & \textbf{poznámky}
                                                                                    \\ \midrule
    \textbf{ESP32}              & dva procesory Xtensa LX6, WiFi a bluetooth    &                                                   \\
    \textbf{BMX055}             & gyroskop, akcelerometr, magnetický kompas     & možno nahradit dvojicí čipů MPU6050 a QMC5883     \\
    \textbf{SPL06}              & barometr                                      & rozlišení až 0,06hPa                              \\
    \textbf{IRM-H936 a IR led}  & IR komunikace                                 &                                                   \\
    \textbf{LDC1614}            & snímání tlakové desky                         & počítá se s možnou záměnou za LDC1314             \\
    \textbf{CP2102}             & programátor                                   & s hardwarově zajištěným odpojováním napájení      \\ \bottomrule
    \end{tabular}%
    }
    \caption{Shrnutí elektronického vybavení}
    \label{tab:shrnuti}
\end{table}

\newpage