\section{LED kruh}
\label{WS2812}

Trezoru vévodí světelný kruh. Slouží jako displej, na kterém trezor může zobrazovat vše, co potřebuje. Kruh obsahuje šedesát jednotlivých LED 
\href{https://cdn-shop.adafruit.com/datasheets/WS2812B.pdf}{WS2812} \parencite{WS2812}, konkrétně variantu mini. Tuto variantu jsem zvolil,
 aby kruh mohl mít menší
průměr, který takto vychází na 80~mm. WS2812 mají totiž rozměr pouzdra 3,5x3,5~mm, zatím co ostatní varianty mají rozměry 
5x5~mm,\footnote{V~průběhu vývoje se na trhu 
objevily i~WS2812 v~pouzdře 2020, které mají rozměru 2x2~mm, ty však nebyly v~nabídce JLCPCB a~navíc byla deska již prakticky hotová.} 
což by znamenalo průměr kruhu alespoň 120~mm.

WS2812 nejsou jen LED, ale mají v~sobě logiku, díky které je možné jich řetězit velké množství za sebe.
Díky tomu na řízení celého kruhu stačí jediný pin na ESP32. %\newline
Ukázka PCB na obrázku \obr{fig:E4-LedDeska} a~zapojení na \obr{fig:E4-sch_WS2812}.

Číslo šedesát jsem zvolil primárně kvůli zobrazování času, každá minuta v~hodině nebo sekunda v~minutě má tak svoji LED.
Zároveň toto číslo dobře navazuje i~na zobrazování úhlů, což koresponduje s~magnetickým kompasem, kterým BlackBox disponuje.

V~momentální verzi softwaru se světelný kruh ovládá pomocí jedné z~periferií ESP32, RMT a~dá se používat ve dvou základních módech. 
V~módu pro interakci s~uživatelem označeném jako DarkMode, který omezuje jas na 4\%\footnote{Jas každého ze tří kanálů RGB 
se dá ovládat v~rozmezí jednoho bajtu a~v~DarkMode je plný jas jen 10 místo 255.} plného jasu, aby se na displej dalo koukat. 
Druhý mód zapíná plný jas LED, je určen pro dálkovou signalizaci nebo pro prosté svícení a~dokáže si říct až o~1,4 A.