\section{Napájení}
Jako napájení celého trezoru slouží dvě li-on baterie 18650. Napětí článků však nevyhovuje potřebám trezoru, a~tak je na trezoru lineární 
stabilizátor \href{https://datasheet.lcsc.com/szlcsc/1808280153_STMicroelectronics-LD39200PU33R_C222192.pdf}{NCP708} \parencite{LD39200}, 
který zajišťuje napětí 3,3~V pro většinu systému. Kromě LD39200 je zde také step-up \href{https://datasheet.lcsc.com/szlcsc/Feeling-Tech-FP6276AXR-G1_C83308.pdf}{FP6276} \parencite{fp6276a}, 
který zajišťuje napájení 5~V sloužící primárně pro LED WS2812 a~v~druhé řadě napájí motor zámku,
jejich zapojení je přiloženo na obrázku \obr{fig:E4-sch_zdroj}.

\paragraph*{Zapínání}
\addcontentsline{toc}{paragraph}{Zapínání}
Aby se trezor mohl vypnout a~tak šetřit energii, je vybaven obvodem, který to umožňuje \obr{fig:E4-zapinani}.

Při připojení článků se napětí dostane nejprve na PTC\footnote{polymerová PTC, vratná pojistka } \parencite{polyfuse},
které slouží jako ochrana proti nadproudu, například v případě, kdy uživatel připojí dva různě nabité články nebo jeden z~nich přepóluje.
Pokud se proud dostane skrz PTC, dostane se na tranzistor Q11 \parencite{power_MOSFET}, skrz který projde, jen pokud jsou články správně pólovány.
Když se napětí dostane přes ochranu proti přepólování, dostane se na 
vývod source
tranzistoru Q5 \parencite{power_MOSFET}, skrz R6 na vývod drain
Q1 a~pak skrz R7 na obě strany C61.
Pokud v~takovéto situaci dojde ke~stisku SW3, projde zem skrz C61 na vývod gate tranzistoru Q5. 
V~tu~chvíli se Q5 otevře na dostatečně dlouhou dobu, 
aby naběhla třívoltová větev a~skrz pull-up\footnote{Na obrázku je jen poznámka, reálná součástka je ve schématu společně s ESP32 na obrázku \obr{fig:E4-sch_ESP32}.}
se zvedlo napětí na gate tranzistoru Q1 na téměř 3,3~V. Q1 se tak otevře a~už~trvale připojí GND na gate tranzistoru Q5, trezor se tak zapnul. Pokud v takové chvíli procesor stáhne dráhu SHUTDOWN 3V3-5V 
na GND, nebo dojde ke stisku SW5, opět se uzavře Q1 a~skrz R6 projde na gate Q5 napětí, které Q5 uzavře a~tak elektroniku opět vypne.

\begin{figure}[h]
    \centering
    \includegraphics[width=\textwidth]{kapitoly/obrazky/E4/napajeni/ochrana_proti_prepolovani_a_zapinani.png}
    \caption{Ochrana proti přepólovaní a zapínání}
    \label{fig:E4-zapinani}
\end{figure}

\newpage

\paragraph*{Stabilizátor}
\addcontentsline{toc}{paragraph}{Stabilizátor}

Stabilizátor \href{https://datasheet.lcsc.com/szlcsc/1808280153_STMicroelectronics-LD39200PU33R_C222192.pdf}{LD39200} \parencite{stabl} má pin EN, který slouží k~je\-ho vypínání. 
Pokud je na něm logická 0, je~stabilizátor vypnut a~pokud 1, je zapnut. Vzhledem k~tomu, že v~mém zapojení toto vypínaní nepotřebuji, je pin EN připojen 
přes R10 přímo na napájecí napětí, a~tak je stabilizátor trvale zapnut.
Konkrétně LD39200 jsem vybral kvůli malému pádu napětí, který vyžaduje pro svůj provoz, typicky 120~mV při proudu 1~A. Vzhledem k~tomu, že~na~vstupu 
mám maximálně 4,2~V, tak maximální napěťový pád, který mám k~dispozici je~0,9~V, protože na výstupu požaduji napětí 3,3~V. Navíc musím být připraven
i~na~vybitou baterii, u~které počítám s~napětím 3,5~V. %Rád bych počítal s~napětím ještě nižším, ale v~nabídce JLCPCB jsem nenašel stabilizátor s~nižším pádem napětí a~zároveň dostatečným proudem.

\begin{figure}[htbp]
    \centering
    \includegraphics[width=400pt]{kapitoly/obrazky/E4/napajeni/stabilizator.png}
    \caption{Zapojení stabilizátoru}
    \label{fig:E4-stabilizator}
\end{figure}


%\paragraph*{Step-up vysvětlení funkce}
%Zapojení step-upu\footnote{Spínaný zdroj, který spíná vstupní napětí na napětí vyšší.} je~o~něco složitější než stabilizátor, který stačí připojit a funguje. 
%Spínané zdroje využívají ke své funkci cívku, na které vzniká změna napětí. Proud cívkou se nedá okamžitě zastavit a právě toho se využívá. 
%Když se step-up spustí, připne výstup cívky k~zemi. Ve chvíli, kdy napětí na výstupu klesne pod zadanou úroveň,\footnote{Předpokládá se, že je v tu chvíli proud cívkou dostatečný.} 
%přepne se výstup cívky na výstup step-upu.
%Protože proud cívkou se nedá zastavit a~cívkou už proud teče, zvedne se napětí za cívkou, které začne plnit kondenzátor na výstupu.
%Když napětí stoupne nad horní hranici požadovaného napětí, výstup cívky se opět přepne na zem. Ve chvíli, kdy napětí na kondenzátoru opět 
%klesne, tím že dodává proud, připojí se cívka. Protože se~na~dobu pádu napětí na~výstupním kondenzátoru, připojila cívka na~zem, obnovil 
%se~v~ní~proud a~cyklus se tak může opakovat.

\paragraph*{Step-up zapojení na desce trezoru}\footnote{Obvod je na obrázku \obr{fig:E4-step-up}.}Pro ovládání spínání step-upu jsem zvolil \href{https://datasheet.lcsc.com/szlcsc/Feeling-Tech-FP6276AXR-G1_C83308.pdf}{FP6276}.
Tento obvod jsem zvolil, protože mi vyhovoval jak po stránce napětí tak po~stránce efektivity a~ceny, a~zároveň byl v~nabídce firmy JLCPCB.\footnote{Firma u~které jsem desky vyráběl a~osazoval.}
Obvod jsem z~většiny zapojil dle doporučení výrobce, mojí prací bylo vlastně jen správně určit hodnoty 
jednotlivých součástek. Na ovládání pinu EN, který FP6276 vypíná, jsem připojil pull-up k napájení a~pro možnost step-up vypnout tranzistor Q2 \parencite{cj3134k}. 
Pokud tedy procesor stáhne dráhu SHUTDOWN 5V k zemi a~tak přivede na gate tranzistoru Q2 zem, 
Q2 se zavře. Tím se na pin EN přivede skrz R18 napájecí napětí, které step-up spustí. 
Pokud se na gate Q2 přivede naopak logická jedna, Q2 se~otevře a~na~EN~se~dostane zem, která naopak provoz step-upu zastaví.

\begin{figure}[htbp]
    \centering
    \includegraphics[width=400pt]{kapitoly/obrazky/E4/napajeni/step-up.png}
    \caption{Zapojení step-upu}
    \label{fig:E4-step-up}
\end{figure}

\paragraph*{Měření napětí baterií}
\addcontentsline{toc}{paragraph}{měření napětí baterek}

\begin{wrapfigure}{R}{0.4\textwidth}
    \centering
    \includegraphics[width=0.4\textwidth]{kapitoly/obrazky/E4/napajeni/delic_baterimetru.png}
    \caption{Měření napětí baterií\label{fig:baterimetr} \centering}
\end{wrapfigure}

Aby trezor mohl zjistit, že má vybité baterie, musí mít možnost jim měřit napětí. ESP32 obsahuje AD převodník, takže není problém měřit napětí baterie 
i poměrně přesně. Kde však problem nastává, je maximální napětí, které je schopen měřit, a~to 1,1~V. ESP32 má možnost připojit k~AD převodníku dělič,
aby se~na~pin dalo přivést napětí až 3,3~V. To ale pořád není dostatečné, a~také se~tím snižuje přesnost měření. Proto je na desce jednoduchý dělič napětí
složený ze~dvou odporů, jednoho s~hodnotou 1~M$\Omega$ a~druhého 300~k$\Omega$, takže při plně nabitých bateriích bude na výstupu děliče 0,97~V. %todo kolik je tam těch baterek. <- nevim jak to tam teď elegantně dopsat a ani bych to nepovažoval za nějak zvlášť podstatnou informaci v odstavci o měření napětí, je to hned v první větě pod nadpisem napájení
