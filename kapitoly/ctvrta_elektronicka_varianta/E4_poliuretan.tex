\newpage
\section{Výroba mechanických dílů}
\label{odlivani}
Prototypové verze trezoru jsem vyráběl pomocí laserové řezačky a~3D~tisku, a~to technologiemi \href{https://www.3dhubs.com/guides/3d-printing/#fdm}{FDM} 
a~\href{https://www.3dhubs.com/guides/3d-printing/#sla-dlp}{SLA}. 
Po odzkoušení prototypů jsem však u~3D tisknutých dílů přešel na odlévání polyuretanu do silikonové formy. 

\paragraph*{Výroba silikonové formy}
Před tím, než se dá začít odlévat hotový díl, musí se pochopitelně udělat forma. 
Jako materiál formy jsem zvolil silikon pro jeho pružnost. 
Ta mi umožňuje odlévat i~záporné převisy, u~kterých bych musel běžnou formu rozdělit na velké množství dílů.
Konkrétně jsem zvolil dva lukoprény \href{https://www.lucebni.cz/cs/lukopren-n/49-silikonovy-kaucuk-lukopren-n-8200.html}{N8200} 
a~\href{https://www.lucebni.cz/cs/lukopren-n/43-silikonovy-kaucuk-lukopren-n-5221.html}{N5221}, který používám na formy různých dílů.
N8200 je totiž ve srovnání s~N5221 odolnější, ale má vyšší viskozitu, a~tak nezateče do úzkých skulin.
Proto na některé formy používám N8200 a~na jiné N5221, abych formu byl vůbec sto vyrobit. 
Forma na odlévání těla dveří dokonce kombinuje oba dva, jelikož je dvoudílná a~jeden z~dílů je tvarově náročný.

Abych ale mohl formu vyrobit, potřebuji mít vyrobené 
kopyto\footnote{Forma určená pro výrobu další formy.} \obr{fig:E4-kopito}. Abych jej získal, využil jsem znovu 3D tisku.
Tentokráte 3D tisk SLA, který umožňuje dosažení daleko vyšší přesnosti než technologie FDM. %todo mám popisovat i specifické technologické úpravy pro SLA?
Do vyrobeného kopyta se naleje dobře promíchaná a~vyvakuovaná směs kaučukové pasty a~katalyzátoru.
Ta se následně nechá zvulkanizovat a~vyjme se z~kopyta.

Při odlévání vícedílné formy, ve chvíli, kdy se jeden díl odlévá podle již hotového dílu, je třeba natřít kontaktní plochy na hotovém díle separátorem, aby se výsledek neslepil.
Pro tento účel jsem používal \href{https://www.lucebni.cz/cs/pomocne-pripravky/54-lukopren-parafinovy-separator.html}{parafínový separátor}.

\paragraph*{Odlévání polyuretanu}
Konkrétně jsem zvolil polyuretan \href{https://www.smooth-on.com/tb/files/TASK_4_TB.pdf}{TASK 4} s~vyvá\-že\-ný\-mi mechanickými i~technologickými vlastnostmi.
Složky polyuretanu se smísí a~ve vakuové komoře řádně promíchají. Následně se polyuretan nalije do formy a~vloží do přetlakové komory,
protože polyuretany jsou sto, za zvýšeného tlaku, pohltit nějaké množství vzduchu.
Zbylé vzduchové bublinky, které se do polyuretanu dostaly, se tak v~polyuretanu rozpustí.